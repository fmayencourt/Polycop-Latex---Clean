\begin{exercice}
\'Etudier toutes les caractéristiques des droites suivantes (formes implicites).
	Tracer au maximum deux droites sur un même système d'axes, unité des axes : un carré.
\begin{multicols}{2}
\begin{enumerate}
\item $x+y=5$
\item $x-y=-3$
\item $2x+y=6$
\item $x-3y=12$
\item $3x+2y=6$
\item $-2x+3y=6$
\item $-3x-5y=30$
\item $5x+2y=-20$
\item $7x-2y=-28$
\item $3x-4y=9$
\item $-4x+5y=-15$
\item $2x+3y=5$
\item $y=-4$
\item $5y=35$
\item $x=6$
\item $-3x=12$
\end{enumerate}
\end{multicols}
\end{exercice}

\begin{exercice}
Calculer les équations des droites suivantes, connaissant la pente p et les coordonnées d'un point A de la droite. Donner les formes explicites et implicites des équations de droites.
\begin{multicols}{2}
\begin{enumerate}
\item $p=1\text{     }A(1;2)$
\item $p=3\text{     }A(-1;3)$
\item $p=-4\text{     }A(5;-5)$
\item $p=-7\text{     }A(-4;-3)$
\item $p=\frac{1}{2}\text{     }A(4;12)$
\item $p=\frac{4}{5}\text{     }A(-9;5)$
\item $p=-\frac{8}{3}\text{     }A(0;3)$
\item $p=-\frac{11}{13}\text{     }A(-5;0)$
\item $p=\frac{127}{19}\text{     }A(0;0)$
\item $p=\frac{34}{11}\text{     }A(-7;15)$
\item $p=0\text{     }A(4;5)$
\item $p=\infty \text{     }A(4;5)$
\end{enumerate}
\end{multicols}
\end{exercice}

\begin{exercice}
Calculer les équations des droites suivantes, connaissant les coordonnées de deux points A et B de la droite. Donner les formes explicites et implicites des équations de droites.
\begin{multicols}{2}
\begin{enumerate}
\item $$ \left| \begin{array}{l}
 A(2;3) \\ 
B\left( 1;5 \right) \\ 
\end{array} \right.$$
\item $$\left| \begin{array}{l}
 A(12;7) \\ 
B\left( -2;5 \right) \\ 
\end{array} \right.$$
\item $$\left| \begin{array}{l}
 A(-11;-7) \\ 
B\left( 1;1 \right) \\ 
\end{array} \right.$$
\item $$\left| \begin{array}{l}
 A(5;7) \\ 
B\left( -3;-9 \right) \\ 
\end{array} \right.$$
\item $$\left| \begin{array}{l}
 A(-6;4) \\ 
B\left( 7;-2 \right) \\ 
\end{array} \right.$$
\item $$\left| \begin{array}{l}
 A(10;-12) \\ 
B\left( -4;5 \right) \\ 
\end{array} \right.$$
\item $$\left| \begin{array}{l}
 A(3,5;5,5) \\ 
B\left( 2,5;7,5 \right) \\ 
\end{array} \right.$$
\item $$\left| \begin{array}{l}
 A(75;-37) \\ 
B\left( -1;8 \right) 
 \end{array} \right.$$
\item $$\left| \begin{array}{l}
 A(5;5) \\ 
B\left( 12;12 \right) \\ 
\end{array} \right.$$
\item $$\left| \begin{array}{l}
 A(2;-6) \\ 
B\left( 6;-18 \right) \\ 
\end{array} \right.$$
\item $$\left| \begin{array}{l}
 A(3;7) \\ 
B\left( 15;7 \right) \\ 
\end{array} \right.$$
\item $$\left| \begin{array}{l}
 A(-2;-3) \\ 
B\left( -2;6 \right) \\ 
\end{array} \right.$$
\end{enumerate}
\end{multicols}
\end{exercice}

\begin{exercice}
Calculer les coordonnées du point M, milieu du segment AB.
\begin{multicols}{2}
\begin{enumerate}
\item $$\left| \begin{array}{l}
 A\left( 6;8 \right) \\ 
B\left( 2;12 \right) \\ 
\end{array} \right.$$
\item $$\left| \begin{array}{l}
 A\left( -4;16 \right) \\ 
B\left( 6;-10 \right) \\ 
\end{array} \right.$$
\item $$\left| \begin{array}{l}
 A\left( 5;11 \right) \\ 
B\left( 3;-9 \right) \\ 
\end{array} \right.$$
\item $$\left| \begin{array}{l}
 A\left( -18;77 \right) \\ 
B\left( 25;34 \right) \\ 
\end{array} \right.$$
\item $$\left| \begin{array}{l}
 A\left( \frac{13}{2};-\frac{3}{4} \right) \\ 
B\left( -\frac{7}{3};\frac{3}{5} \right) \\ 
\end{array} \right.$$
\item $$\left| \begin{array}{l}
 A\left( 0;0 \right) \\ 
B\left( -16;24 \right) \\ 
\end{array} \right.$$
\end{enumerate}
\end{multicols}
\end{exercice}

\begin{exercice}
Calculer la longueur du segment AB.
\begin{multicols}{3}
\begin{enumerate}
\item $$\left| \begin{array}{l}
 A\left( 2;10 \right) \\ 
B\left( 5;12 \right) \\ 
\end{array} \right.$$
\item $$\left| \begin{array}{l}
 A\left( -8;5 \right) \\ 
B\left( 6;-2 \right) \\ 
\end{array} \right.$$
\item $$\left| \begin{array}{l}
 A\left( 0;0 \right) \\ 
B\left( -5;0 \right) \\ 
\end{array} \right.$$
\end{enumerate}
\end{multicols}
\end{exercice}

\begin{exercice}
Calculer l'équation de la droite passant par le point A et parallèle à la droite d.
\begin{multicols}{3}
\begin{enumerate}
\item $$\left| \begin{array}{l}
 A\left( 5;9 \right) \\ 
d:y=3x-5 \\ 
\end{array} \right.$$
\item $$\left| \begin{array}{l}
 A\left( -3;4 \right) \\ 
d:y=-\frac{5x}{2}+1 \\ 
\end{array} \right.$$
\item $$\left| \begin{array}{l}
 A\left( -7;-3 \right) \\ 
d:3x+7y=2 \\ 
\end{array} \right.$$
\end{enumerate}
\end{multicols}
\end{exercice}

\begin{exercice}
Calculer l'équation de la droite passant par le point A et perpendiculaire à la droite d.
\begin{multicols}{3}
\begin{enumerate}
\item $$\left| \begin{array}{l}
 A\left( 3;-2 \right) \\ 
d:y=2x-5 \\ 
\end{array} \right.$$
\item $$\left| \begin{array}{l}
 A\left( -6;0 \right) \\ 
d:y=\frac{3x}{7} \\ 
\end{array} \right.$$
\item $$\left| \begin{array}{l}
 A\left( -3;8 \right) \\ 
d:5x+2y=2 \\ 
\end{array} \right.$$
\end{enumerate}
\end{multicols}
\end{exercice}