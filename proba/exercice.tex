\subsection{Dénombrements}

\begin{exercice}
Douze joueurs d’échecs participent à un tournoi dans lequel chaque joueur joue une fois contre chacun des autres joueurs. Combien y a-t-il de parties disputées ?
\end{exercice}

\begin{exercice}
Combien de mots différents peut-on former avec le mot GENOVA ?
\end{exercice}

\begin{exercice}
Cinq personnes désirent s’asseoir dans un compartiment de 8 places numérotées. Quel est le nombre de possibilités?
Même question avec huit personnes.
\end{exercice}

\begin{exercice}
Dans une société de 25 personnes, on doit en désigner 4 qui formeront le comité. Combien de comités différents peut-on constituer ?
\end{exercice}

\begin{exercice}
Dans une société de 25 personnes, on doit désigner un président, un vice-président, un trésorier et un secrétaire ; ces quatre personnes forment le comité. Combien de comités différents peut-on constituer ?
\end{exercice}

\begin{exercice}
On distribue les 36 cartes d’un jeu à quatre joueurs. Chacun reçoit 9 cartes. Quel est le nombre de distributions différentes ?
\end{exercice}

\begin{exercice}
De combien de façons peut-on choisir 5 cartes à jouer dans un jeu de 36 cartes, de manière que ces 5 cartes comprennent :
\begin{enumerate}
\item les 4 as ?
\item 2 as et 2 rois ?
\item au moins un as ?
\end{enumerate}
\end{exercice}

\begin{exercice}
De combien de façons peut-on remplir une feuille de loterie à numéros ?
Combien, parmi toutes ces possibilités, permettent-elles de réaliser 6 points ? 0 point ?
\end{exercice}

\begin{exercice}
Avec les chiffres de l'ensemble $E=\left\{ 0,1,2,3,4,5,6,7,8,9 \right\}$
\begin{enumerate}
\item Combien peut-on former de nombres composés de 6 chiffres différents ?
\item Parmi les nombres précédents, combien se terminent par 1 ?
\item Combien sont impairs ? 
\item Combien sont strictement inférieurs à 300000 ? 
\end{enumerate}
\end{exercice}

\begin{exercice}
Avec les lettres du mot "MARTIGNY"
\begin{enumerate}
\item Combien peut-on former de mots de 5 lettres ?
\item Parmi ceux-ci, combien se terminent par Y ?
\item Parmi ceux-ci, combien commencent par M et se terminent par Y ?
\item Parmi ceux-ci, combien contiennent 3 voyelles (Y est une voyelle) ?
\item Parmi ceux-ci, combien contiennent exactement 2 voyelles ?
\item Parmi ceux-ci, combien contiennent au moins deux voyelles ?
\end{enumerate}
\end{exercice}

\begin{exercice}
Un jeu de cartes contient 36 cartes, 4 couleurs ($\clubsuit, \lozenge, \spadesuit, \heartsuit $) et 9 symboles (6, 7, 8, 9, 10, valet, dame, roi, as)
\begin{enumerate}
\item Combien de mains différentes de 9 cartes peut-on obtenir ?
\item Parmi ces mains, combien contiennent 4 cœurs
\item Parmi ces mains, combien contiennent 4 cœurs, 3 piques, deux trèfles ?
\item Combien ne contiennent pas d'as ?
\item Combien contiennent exactement 1 as ?
\item Combien contiennent exactement 2 as, 3 as, 4 as, au moins 3 as ?
\end{enumerate}
\end{exercice}

\begin{exercice}
Pour mettre sur pied la course d'école d'une classe de 20 élèves, six organisateurs doivent être choisis parmi les élèves.
\begin{enumerate}
\item Combien y a-t-il de possibilités ?
\item Sachant qu'il y a 9 filles dans la classe, combien de possibilités seront-elles mixtes ?
\item Seulement composées de filles ?
\end{enumerate}
\end{exercice}

\begin{exercice}
Un étudiant doit résoudre 8 problèmes sur 10 lors d'une épreuve écrite.
\begin{enumerate}
\item Combien de choix différents peut-il faire ?
\item Même question en supposant qu'il doive obligatoirement résoudre :
\begin{itemize}
\item les 3 premiers problèmes
\item 4 exactement des 5 premiers problèmes
\end{itemize}
\end{enumerate}
\end{exercice}

\subsection{Probabilités}


\begin{exercice}Une seule carte est tirée d’un jeu de 52 cartes, calculer la probabilité que la carte soit :
\begin{enumerate}
\item Un as
\item Un as de couleur rouge
\item Un roi ou une reine
\item Un roi ou une reine ou un valet
\item Un trèfle ou un carreau
\item Un cœur ou un carreau ou un pique
\end{enumerate}
\end{exercice}

\begin{exercice}
Un seul dé est lancé, calculer la probabilité que le résultat soit :
\begin{enumerate}
\item un 5
\item un 2 ou un 3
\item un nombre pair
\end{enumerate}
\end{exercice}

\begin{exercice}
Une urne contient cinq balles rouges, six balles vertes et quatre balles blanches. Si une seule balle est extraite de l’urne, calculer la probabilité que la balle soit :
\begin{enumerate}
\item rouge
\item verte
\item rouge ou blanche
\item verte ou blanche
\end{enumerate}
\end{exercice}

\begin{exercice}
Un jeu de cartes contient 36 cartes, 4 couleurs ($\clubsuit, \lozenge, \spadesuit, \heartsuit $) et 9 symboles (6, 7, 8, 9, 10, valet, dame, roi, as). De ce jeu on tire une main de 9 cartes, calculer la probabilité que parmi ces différentes mains :
\begin{enumerate}
\item Une contienne exactement 6 trèfles
\item Une contienne les quatre valets
\item Une ne contienne pas de cœur
\item Une contienne exactement deux rois et deux reines
\end{enumerate}
\end{exercice}

\begin{exercice}Avec 10 députés et 6 sénateurs, on veut former une commission de 7 membres. Quelle est la probabilité que cette commission comprenne 5 députés ?
\end{exercice}

\begin{exercice}
Calculer la probabilité que l’on tire d’un jeu de 36 cartes :
\begin{enumerate}
\item un as ou un roi
\item un as ou un cœur	
\item un as ou une carte rouge
\item un as ou un roi ou un cœur
\item un as ou un roi ou une carte rouge
\end{enumerate}
\end{exercice}

\begin{exercice}
Dans une assemblée de 25 dames et 15 messieurs, il est décidé de nommer un comité de 5 personnes.
\begin{enumerate}
\item Quelle est la probabilité que ce comité comprennent 3 dames ?
\item Quelle est la probabilité que ce comité comprennent au moins 3 dames ?
\end{enumerate}
\end{exercice}

\begin{exercice}L'Euromillions consiste à cocher 5 numéros parmi 50 et 2 étoiles parmi 9.
Quelle est la probabilité de cocher 3 bons numéros et une bonne étoile ?
\end{exercice}

\begin{exercice}
Avec les lettres du mot ZURICH, on forme des mots de 4 lettres.
\begin{enumerate}
\item Quelle est la probabilité que ces mots commencent par R ?
\item Quelle est la probabilité que ces mots commencent par CH ?
\item Quelle est la probabilité que ces mots contiennent exactement les 2 voyelles ?
\end{enumerate}
\end{exercice}