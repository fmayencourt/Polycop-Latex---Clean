\chapter{Probabilités}

\section{Un peu d'histoire}

Bien que l'histoire des jeux de hasard remonte à la nuit des temps, on ne retrouve la trace de calculs de probabilités (à savoir : quelles sont les chances de gagner ou de perdre) qu'au Moyen-Âge. Le premier problème nous vient d'une correspondance entre Blaise Pascal~\index{Pascal, Blaise} et Pierre de Fermat\index{Fermat, Pierre} et porte sur l'histoire suivante (quelque peu simplifiée et tirée de \cite{godfroy2000pascal}) :
\begin{quote}
Deux joueurs jouent l'un contre l'autre à une partie de dés. Ils ont chacun misé dix pièces d'or. Le premier qui gagne trois manches remporte la totalité de la mise, c'est-à-dire les vingt pièces. Or ils doivent arrêter le jeu avant que l'un des deux soit vainqueur. Comment se répartir alors les vingt pièces pour respecter l'état actuel du jeu ?
\end{quote}

Ce problème peut paraître simple, voir simpliste, mais il a inspiré aux mathématiciens une nouvelle branche : celle du hasard. Il a même eu des répercutions jusqu'au XXe siècle\cite{coumet1970theorie} : Georges-Théodule Guilbaud a relié ce problème à celui, plus juridique, des ruptures de contrats aléatoires.

\section{Et un peu de math}

Nous allons nous concentrer sur un problème plus parlant : Dans une classe de vingt élèves avec vingt places, le titulaire décide de changer les places. Or Jules souhaite savoir combien il a de chance de se trouver à côté d'une fenêtre. Il a une idée : il commence par compter le nombre de possibilité de placement qu'a le titulaire, puis le nombre de placement où il se trouve à côté d'une fenêtre. Formalisons un peu cette pensée :

\begin{definition}
Dans une situation aléatoire discrète, la probabilité qu'un évènement $A$ surgisse, notée $P(A)$ est donnée par
$$
P(A) = \frac{\mbox{nombre de fois où }A\mbox{ arrive}}{\mbox{nombre d'issues possibles}}
$$
\end{definition}

Éclaircissons un peu cette définition, en particulier le terme de "situation aléatoire discrète" : il s'agit d'une situation dont on ne peut pas calculer l'issue car le nombre de variables est trop grand ou font appel à un élément humain, mais le nombre d'issues, même s'il est grand, peut être calculé. Il s'agit par exemple d'un lancé de dés ou de tirer une carte dans un jeu. A contrario, on va parler de situation aléatoire continue si le nombre d'issues ne peut pas être déterminé.

Reprenons notre exemple de plan de classe. Pour compter le nombre de possibilités de plans de classe, Jules procède de manière inductive : il commence par placer un élève, puis un deuxième, un troisième, etc. Il se rend vite compte qu'il ne va pas pouvoir faire tous les plans de classe. Mais heureusement, dans sa méthode il suffit de les compter. Voilà son raisonnement : il attribue à chaque place un numéro (de $1$ à $20$) et se dit 
\begin{quotation}
À la place numéro $1$, j'ai vingt possibilités (vingt élèves). J'en prends un au hasard et le mets là. Ensuite, pour la place numéro $2$, je n'ai plus que dix-neuf possibilités, car il ne reste plus que dix-neuf élèves à placer. Pour la place numéro $3$, il m'en reste dix-huit, etc.

Si je représente la situation en forme d'arbre, la première division se fait en $20$, puis en $19$. A la deuxième étape, j'ai donc déjà $20\cdot 19$ possibilités de placement. Ainsi, le nombre de placements différents doit être donné par 
$$
20\cdot 19\cdot 18\cdot 17 \cdot \dots \cdot 3 \cdot 2 \cdot 1 
$$
\end{quotation}

En généralisant le raisonnement de Jules, on obtient :

\begin{definition}
Le nombre de possibilités de placer $n$ objets à $n$ places distinctes est donné par 
$$
n! = n \cdot (n-1) \cdot \dots \cdot 2 \cdot 1
$$
On appelle ce calcul $n$ factoriel.
\end{definition}

Jules sait à présent que le nombre total de plans est de $20!$. Il doit encore compter le nombre de plans où il se trouve à côté d'une fenêtre. Il raisonne de la manière suivante :
\begin{quote}
Il y a quatre places au bord d'une fenêtre. Une fois que je m'y suis assis, il reste 19 places pour les 19 élèves restants. Ainsi le nombre de plans où je suis au bord d'une fenêtre est de 
$$
4 \cdot 19!
$$
Ainsi la probabilité que je sois à côté d'une fenêtre est de 
$$
P = \frac{4\cdot 19!}{20!} = 0.2 = 20 \%
$$
\end{quote}

Or voilà que deux élèves décident de quitter l'école. Il ne reste donc plus que dix-huit élèves à placer sur vingt places. Jules raisonne ainsi :
\begin{quote}
Cette fois, je ne suis pas sûr que la place $1$ soit occupée. Par contre, je sais que chaque élève a une place. Je prend donc le premier élève et le place dans la classe ; il y a vingt possibilités. Puis, je fais de même avec le deuxième élève qui lui n'en a plus que dix-neuf possibilités et ainsi de suite jusqu'au dernier élève qui n'en a que trois. Ainsi le nombre de plans différents est de 
$$
20\cdot 19 \cdot \cdot \dots \cdot 4 \cdot 3
$$
\end{quote}
Jules a raisonné en faisant un arrangement de dix-huit personnes dans vingt places, qu'on formalise de la manière suivante :
\begin{definition}
Le nombre de possibilités de placer $p$ objets dans $n$ places (avec $p>n$) est donné par 
$$
A^p_n = \frac{n!}{(n-p)!}
$$
On appelle cela un \emph{arrangement de $p$ parmi $n$}.
\end{definition}

Jules doit maintenant déterminer le nombre de possibilités d'être à côté d'une fenêtre :
\begin{quote}
J'ai toujours quatre possibilités d'être à côté d'une fenêtre. L'élève suivant a dix-neuf possibilités de placement, etc. jusqu'au dernier qui n'en a plus que trois. Ainsi le nombre de places où je suis à côté d'une fenêtre est de 
$$
4\cdot 19 \cdot \dots \cdot 3
$$
Ainsi la probabilité que je sois à côté d'une fenêtre est de 
$$
P = \frac{4\cdot 19 \cdot \dots \cdot 3}{20\cdot 19 \cdot \cdot \dots \cdot 4 \cdot 3} = \frac{^4\cdot A^{19}_3}{A^{20}_3} = 0.2 = 20 \%
$$
\end{quote}

\section{Exercices}

\subsection{Dénombrements}

\begin{exercice}
Douze joueurs d’échecs participent à un tournoi dans lequel chaque joueur joue une fois contre chacun des autres joueurs. Combien y a-t-il de parties disputées ?
\end{exercice}

\begin{exercice}
Combien de mots différents peut-on former avec le mot GENOVA ?
\end{exercice}

\begin{exercice}
Cinq personnes désirent s’asseoir dans un compartiment de 8 places numérotées. Quel est le nombre de possibilités?
Même question avec huit personnes.
\end{exercice}

\begin{exercice}
Dans une société de 25 personnes, on doit en désigner 4 qui formeront le comité. Combien de comités différents peut-on constituer ?
\end{exercice}

\begin{exercice}
Dans une société de 25 personnes, on doit désigner un président, un vice-président, un trésorier et un secrétaire ; ces quatre personnes forment le comité. Combien de comités différents peut-on constituer ?
\end{exercice}

\begin{exercice}
On distribue les 36 cartes d’un jeu à quatre joueurs. Chacun reçoit 9 cartes. Quel est le nombre de distributions différentes ?
\end{exercice}

\begin{exercice}
De combien de façons peut-on choisir 5 cartes à jouer dans un jeu de 36 cartes, de manière que ces 5 cartes comprennent :
\begin{enumerate}
\item les 4 as ?
\item 2 as et 2 rois ?
\item au moins un as ?
\end{enumerate}
\end{exercice}

\begin{exercice}
De combien de façons peut-on remplir une feuille de loterie à numéros ?
Combien, parmi toutes ces possibilités, permettent-elles de réaliser 6 points ? 0 point ?
\end{exercice}

\begin{exercice}
Avec les chiffres de l'ensemble $E=\left\{ 0,1,2,3,4,5,6,7,8,9 \right\}$
\begin{enumerate}
\item Combien peut-on former de nombres composés de 6 chiffres différents ?
\item Parmi les nombres précédents, combien se terminent par 1 ?
\item Combien sont impairs ? 
\item Combien sont strictement inférieurs à 300000 ? 
\end{enumerate}
\end{exercice}

\begin{exercice}
Avec les lettres du mot "MARTIGNY"
\begin{enumerate}
\item Combien peut-on former de mots de 5 lettres ?
\item Parmi ceux-ci, combien se terminent par Y ?
\item Parmi ceux-ci, combien commencent par M et se terminent par Y ?
\item Parmi ceux-ci, combien contiennent 3 voyelles (Y est une voyelle) ?
\item Parmi ceux-ci, combien contiennent exactement 2 voyelles ?
\item Parmi ceux-ci, combien contiennent au moins deux voyelles ?
\end{enumerate}
\end{exercice}

\begin{exercice}
Un jeu de cartes contient 36 cartes, 4 couleurs ($\clubsuit, \lozenge, \spadesuit, \heartsuit $) et 9 symboles (6, 7, 8, 9, 10, valet, dame, roi, as)
\begin{enumerate}
\item Combien de mains différentes de 9 cartes peut-on obtenir ?
\item Parmi ces mains, combien contiennent 4 cœurs
\item Parmi ces mains, combien contiennent 4 cœurs, 3 piques, deux trèfles ?
\item Combien ne contiennent pas d'as ?
\item Combien contiennent exactement 1 as ?
\item Combien contiennent exactement 2 as, 3 as, 4 as, au moins 3 as ?
\end{enumerate}
\end{exercice}

\begin{exercice}
Pour mettre sur pied la course d'école d'une classe de 20 élèves, six organisateurs doivent être choisis parmi les élèves.
\begin{enumerate}
\item Combien y a-t-il de possibilités ?
\item Sachant qu'il y a 9 filles dans la classe, combien de possibilités seront-elles mixtes ?
\item Seulement composées de filles ?
\end{enumerate}
\end{exercice}

\begin{exercice}
Un étudiant doit résoudre 8 problèmes sur 10 lors d'une épreuve écrite.
\begin{enumerate}
\item Combien de choix différents peut-il faire ?
\item Même question en supposant qu'il doive obligatoirement résoudre :
\begin{itemize}
\item les 3 premiers problèmes
\item 4 exactement des 5 premiers problèmes
\end{itemize}
\end{enumerate}
\end{exercice}

\subsection{Probabilités}


\begin{exercice}Une seule carte est tirée d’un jeu de 52 cartes, calculer la probabilité que la carte soit :
\begin{enumerate}
\item Un as
\item Un as de couleur rouge
\item Un roi ou une reine
\item Un roi ou une reine ou un valet
\item Un trèfle ou un carreau
\item Un cœur ou un carreau ou un pique
\end{enumerate}
\end{exercice}

\begin{exercice}
Un seul dé est lancé, calculer la probabilité que le résultat soit :
\begin{enumerate}
\item un 5
\item un 2 ou un 3
\item un nombre pair
\end{enumerate}
\end{exercice}

\begin{exercice}
Une urne contient cinq balles rouges, six balles vertes et quatre balles blanches. Si une seule balle est extraite de l’urne, calculer la probabilité que la balle soit :
\begin{enumerate}
\item rouge
\item verte
\item rouge ou blanche
\item verte ou blanche
\end{enumerate}
\end{exercice}

\begin{exercice}
Un jeu de cartes contient 36 cartes, 4 couleurs ($\clubsuit, \lozenge, \spadesuit, \heartsuit $) et 9 symboles (6, 7, 8, 9, 10, valet, dame, roi, as). De ce jeu on tire une main de 9 cartes, calculer la probabilité que parmi ces différentes mains :
\begin{enumerate}
\item Une contienne exactement 6 trèfles
\item Une contienne les quatre valets
\item Une ne contienne pas de cœur
\item Une contienne exactement deux rois et deux reines
\end{enumerate}
\end{exercice}

\begin{exercice}Avec 10 députés et 6 sénateurs, on veut former une commission de 7 membres. Quelle est la probabilité que cette commission comprenne 5 députés ?
\end{exercice}

\begin{exercice}
Calculer la probabilité que l’on tire d’un jeu de 36 cartes :
\begin{enumerate}
\item un as ou un roi
\item un as ou un cœur	
\item un as ou une carte rouge
\item un as ou un roi ou un cœur
\item un as ou un roi ou une carte rouge
\end{enumerate}
\end{exercice}

\begin{exercice}
Dans une assemblée de 25 dames et 15 messieurs, il est décidé de nommer un comité de 5 personnes.
\begin{enumerate}
\item Quelle est la probabilité que ce comité comprennent 3 dames ?
\item Quelle est la probabilité que ce comité comprennent au moins 3 dames ?
\end{enumerate}
\end{exercice}

\begin{exercice}L'Euromillions consiste à cocher 5 numéros parmi 50 et 2 étoiles parmi 9.
Quelle est la probabilité de cocher 3 bons numéros et une bonne étoile ?
\end{exercice}

\begin{exercice}
Avec les lettres du mot ZURICH, on forme des mots de 4 lettres.
\begin{enumerate}
\item Quelle est la probabilité que ces mots commencent par R ?
\item Quelle est la probabilité que ces mots commencent par CH ?
\item Quelle est la probabilité que ces mots contiennent exactement les 2 voyelles ?
\end{enumerate}
\end{exercice}