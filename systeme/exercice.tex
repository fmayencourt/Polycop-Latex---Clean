\subsection{Systèmes $2\times 2$}

\begin{exercice}
Résoudre les systèmes suivants :
\begin{multicols}{2}
\begin{enumerate}
\item $$\left\{ \begin{array} {l}
    3x+4y=24 \\ 
   5y=15 \\ 
\end{array} \right.$$ 
\item $$\left\{ \begin{array} {l}
    x+y=19 \\ 
   2x-y=2 \\ 
\end{array} \right.$$ 
\item $$\left\{ \begin{array} {l}
    12x-5y=29 \\ 
   4x-3y=11 \\ 
\end{array} \right.$$ 
\item $$\left\{ \begin{array} {l}
    x+y=28 \\ 
   3x-11y=8y-48 \\ 
\end{array} \right.$$ 
\item $$\left\{ \begin{array} {l}
    x+2y=22 \\ 
   5(x-5)=y-3 \\ 
\end{array} \right.$$ 
\item $$\left\{ \begin{array} {l}
    x-12y=4 \\ 
   41y+7=3x \\ 
\end{array} \right.$$ 
\item $$\left\{ \begin{array} {l}
    x+3y=11 \\ 
   5y-68=3(x-1) \\ 
\end{array} \right.$$ 
\item $$\left\{ \begin{array} {l}
    12x-5y=57 \\ 
   15y+2x=19 \\ 
\end{array} \right.$$ 
\item $$\left\{ \begin{array} {l}
    24x+7y=69 \\ 
   7(9x-5y)=21 \\ 
\end{array} \right.$$ 
\item $$\left\{ \begin{array} {l}
    8x-9y=53 \\ 
   5x+18y=41 \\ 
\end{array} \right.$$ 
\item $$\left\{ \begin{array} {l}
    5x=y \\ 
   12x-2y=10 \\ 
\end{array} \right.$$
\item $$\left\{ \begin{array} {l}
    2x+3y=4 \\ 
   3y+10x=44 \\ 
\end{array} \right.$$
\item $$\left\{ \begin{array} {l}
    2x-3y+35=0 \\ 
   4x-y-25=0 \\ 
\end{array} \right.$$
\item $$\left\{ \begin{array} {l}
    12x+11y=6 \\ 
   3y-2x=28 \\ 
\end{array} \right.$$
\item $$\left\{ \begin{array} {l}
    2x+5y=69 \\ 
   y-4(x-7)=67-3x \\ 
\end{array} \right.$$
\item $$\left\{ \begin{array} {l}
    72x+14y=330 \\ 
   63x+7y=273 \\ 
\end{array} \right.$$
\item $$\left\{ \begin{array} {l}
    21x+8y+66=0 \\ 
   28x-23y-13=0 \\ 
\end{array} \right.$$
\item $$\left\{ \begin{array} {l}
    3y-4x=1 \\ 
   6(2x+y)=17 \\ 
\end{array} \right.$$
\item $$\left\{ \begin{array} {l}
    123x-234y=-99 \\ 
   345x-456y=123 \\ 
\end{array} \right.$$
\item $$\left\{ \begin{array} {l}
    999x+y=1001 \\ 
   x+999y=1999 \\ 
\end{array} \right.$$
\end{enumerate}
\end{multicols}
\end{exercice}

\begin{exercice} 
Résoudre les systèmes suivants :
\begin{multicols}{2}
\begin{enumerate}
\item $$\left\{ \begin{array} {l}
    x+\frac{3y}{7}=17 \\ 
   y-\frac{5x}{8}=16 \\ 
\end{array} \right.$$ 
\item $$\left\{ \begin{array} {l}
    \frac{x}{3}+\frac{y}{4}=9 \\ 
   \frac{x}{4}+\frac{y}{5}=7 \\ 
\end{array} \right.$$ 
\item $$\left\{ \begin{array} {l}
    \frac{4x-5}{2y-3}=3 \\ 
   \frac{3x+5}{y+1}=4 \\ 
\end{array} \right.$$ 
\item $$\left\{ \begin{array} {l}
    \frac{x-1}{8}+\frac{y-2}{5}=2 \\ 
   2x+\frac{2y-5}{3}=21 \\ 
\end{array} \right.$$ 
\item $$\left\{ \begin{array} {l}
    \frac{x-4}{3}-\frac{3y+4}{10}=x-y \\ 
   \frac{2x-5}{5}-\frac{2y-4}{4}=x-12 \\ 
\end{array} \right.$$ 
\item $$\left\{ \begin{array} {l}
    \frac{4x+15}{3}-\frac{3y-5}{5}=x \\ 
   \frac{2y+3x}{4}+\frac{y+15}{5}=y \\ 
\end{array} \right.$$
\item $$\left\{ \begin{array} {l}
    \frac{x-y}{3}-\frac{1}{4}\left( x-\frac{10-2y}{3} \right)=3 \\ 
   \frac{x-5y}{5}+\frac{x+2}{2}=x-4 \\ 
\end{array} \right.$$
\item $$\left\{ \begin{array} {l}
    \frac{3x}{5}+\frac{4y}{10}=\frac{x-y}{5} \\ 
   \frac{10(2x+3)}{11}-2\left( y-\frac{3x-5}{8} \right)=60 \\ 
\end{array} \right.$$
\item $$\left\{ \begin{array} {l}
    \frac{13}{3+x+2y}+\frac{3}{6+4x-5y}=0 \\ 
   \frac{6x-5y+4}{3}=\frac{3x+2y+1}{19} \\ 
\end{array} \right.$$
\item $$\left\{ \begin{array} {l}
    \frac{x+5}{x+1}=\frac{y-9}{y+7}+\frac{112}{(x+1)(y+7)} \\ 
   2x+10=3y+1 \\ 
\end{array} \right.$$
\end{enumerate}
\end{multicols}
\end{exercice}

\subsection{Systèmes $n\times n$}

\begin{exercice}
Résoudre les systèmes suivants (algorithme de Gauss-Jordan) : 
\begin{multicols}{2}
\begin{enumerate}
\item $$\left\{ \begin{array}{l}
    4x+3y+6z=41 \\ 
   8x+5y=31 \\ 
   7y=21 \\ 
\end{array} \right.$$

\item $$\left\{ \begin{array}{l}
    2x-3y+2z=41 \\ 
   5x+3y+z=10 \\ 
   9x=27 \\ 
\end{array} \right.$$

\item $$\left\{ \begin{array}{l}
    7x-4y-5z=56 \\ 
   3y-2z=13 \\ 
   5x-3y=22 \\ 
\end{array} \right.$$

\item $$\left\{ \begin{array}{l}
    6x-y+3z=38 \\ 
   5x-2y+z=24 \\ 
   3x-y+5z=24 \\ 
\end{array} \right.$$

\item $$\left\{ \begin{array}{l}
    x+y+z=25 \\ 
   x-y+z=5 \\ 
   x+2z=2y-10 \\ 
\end{array} \right.$$

\item $$\left\{ \begin{array}{l}
    x-y-z=6 \\ 
   x-2y-3z=10 \\ 
   5x+6y+z=2 \\ 
\end{array} \right.$$

\item $$\left\{ \begin{array}{l}
    -x-2y+3z=18 \\ 
   -2x+2y+3z=36 \\ 
   5x+2y-z=10 \\ 
\end{array} \right.$$

\item $$\left\{ \begin{array}{l}
    3x-y+z=29 \\ 
   x+3y+30z=6 \\ 
   x-y+z=17 \\ 
\end{array} \right.$$ 
\end{enumerate}
\end{multicols}
\end{exercice}

\begin{exercice}
Résoudre les systèmes suivants (méthode des déterminants, règle de Sarrus) : 
\begin{multicols}{2}
\begin{enumerate}
\item $$\left\{ \begin{array}{l}
    2x+3y+4z=53 \\ 
   3x+5y-4z=2 \\ 
   4x+7y-2z=31 \\ 
\end{array} \right.$$

\item $$\left\{ \begin{array}{l}
    4x+5y-3z=-1 \\ 
   2x-10y+z=2 \\ 
   -3x+15y+4z=8 \\ 
\end{array} \right.$$
\item $$\left\{ \begin{array}{l}
    14x-13y+12z=61 \\ 
   -11x+15y+18z=168 \\ 
   10x+17y-16z=3 \\ 
\end{array} \right.$$

\item $$\left\{ \begin{array}{l}
    15x-13y+18z=-45 \\ 
   17x+21y-31z=206 \\ 
   10x+12y-2z=40 \\ 
\end{array} \right.$$ 
\end{enumerate}
\end{multicols}
\end{exercice}

\begin{exercice}
Résoudre les systèmes suivants : 
\begin{multicols}{2}
\begin{enumerate}
\item $$\left\{ \begin{array}{l}
    x-\frac{4y-3}{3}+\frac{z-2}{2}=2 \\ 
   \frac{x}{5}-\frac{3y}{2}+3z=22 \\ 
   \frac{x-1}{4}-\frac{y-1}{5}=\frac{z-10}{10} \\ 
\end{array} \right.$$

\item $$\left\{ \begin{array}{l}
    \frac{4x-7}{3}-\frac{2(y-2)}{3}=z \\ 
   \frac{3x}{7}+\frac{2y-1}{7}-5z=1 \\ 
   \frac{2x+y}{5}+\frac{z}{4}=\frac{21}{4} \\ 
\end{array} \right.$$
 

\item $$\left\{ \begin{array}{l}
    \frac{x+2y-3z}{13}-4z=3(x+2) \\ 
   \frac{5x-1}{7}+2y-z=33 \\ 
   x+\frac{2y+7}{5}-z=3x-7 \\ 
\end{array} \right.$$

\item $$\left\{ \begin{array}{l}
    x+y+z=a+b \\ 
   \frac{1}{x-y}=\frac{1}{2b} \\ 
   \frac{x}{y-z}-\frac{1}{2}=\frac{b}{a-b} \\ 
\end{array} \right.$$ 
\end{enumerate}
\end{multicols}
\end{exercice}

\begin{exercice}
Résoudre les systèmes suivants avec des artifices de calcul : 
\begin{multicols}{2}
\begin{enumerate}
\item $$\left\{ \begin{array}{l}
    x+y=10 \\ 
   x+z=19 \\ 
   y+z=23 \\ 
\end{array} \right.$$

\item $$\left\{ \begin{array}{l}
    x+y+z=3 \\ 
   y+z+v=-2 \\ 
   x+z+v=6 \\ 
   x+y+v=2 \\ 
\end{array} \right.$$

\item $$\left\{ \begin{array}{l}
    -2x+y+z+v=-1 \\ 
   x-2y+z+v=11 \\ 
   x+y-2z+v=5 \\ 
   x+y+z-2v=17 \\ 
\end{array} \right.$$

\item $$\left\{ \begin{array}{l}
    x+y-z=17 \\ 
   x-y+z=13 \\ 
   -x+y+z=7 \\ 
\end{array} \right.$$

\item $$\left\{ \begin{array}{l}
    \frac{1}{x}+\frac{1}{y}=\frac{9}{20} \\ 
   \frac{1}{y}+\frac{1}{z}=\frac{13}{15} \\ 
   \frac{1}{x}+\frac{1}{z}=\frac{11}{12} \\ 
\end{array} \right.$$

\item $$\left\{ \begin{array}{l}
    x+y=18 \\ 
   y+z=14 \\ 
   z+u=10 \\ 
   u+v=6 \\ 
   x+v=12 \\ 
\end{array} \right.$$
\end{enumerate}
\end{multicols}
\end{exercice}

\subsection{Problèmes}

\begin{exercice}
Comment peut-on payer la somme de Fr. 99.— avec 27 pièces, les unes de Fr. 5.— les autres de Fr. 2.— ?
\end{exercice}

\begin{exercice}
Il y a 4 ans, l’âge d’un père était le quadruple de celui de son fils; dans 10 ans, il n’en sera plus que le double. Quels sont les âges actuels ?
\end{exercice}

\begin{exercice}
Il y a 7 ans, la moitié de l’âge de mon oncle surpassait le mien de 2 ans. Aujourd’hui mon âge surpasse de 5 ans les $\frac{2}{5}$ de celui de mon oncle. Quels sont les âges ?
\end{exercice}

\begin{exercice}
Deux sommes placées l’une à $4 \%$ et l’autre à $5 \%$ produisent ensemble un revenu annuel de Fr. 400.—. Si l’une était placée au taux de l’autre et inversement, elles donneraient alors Fr. 410.—. Quelles sont ces deux sommes ?
\end{exercice}

\begin{exercice}
Deux sommes placées à $5 \%$ donnent Fr. 550.— d’intérêts par an; en diminuant le taux de la première et en augmentant celui de la seconde, chacun de $\frac{1}{4} \%$, l’intérêt serait augmenté de Fr. 2.50. Quelles sont les deux sommes ?
\end{exercice}

\begin{exercice}
Deux sommes, l’une de Fr. 5’000.— et l’autre de Fr. 6’000.—, rapportent ensemble Fr. 525.— par an. En plaçant l’une au taux de l’autre et inversement, l’intérêt ne serait que de Fr. 520.—. Quels sont les deux taux ?
\end{exercice}

\begin{exercice}
Deux capitaux A et B ont été placés comme il suit: le $\frac{1}{4}$ de A et les $\frac{3}{5}$ de B à $4 \%$; les restes à $5  \%$. Le premier placement donne Fr. 2’160.— d’intérêts simples en 3 ans et l’autre Fr. 5’200.— en 4 ans. Trouver les deux capitaux.
\end{exercice}

\begin{exercice}
Jean à placé Fr. 12’600.— de plus que Louis et à $1 \%$ de plus; aussi retire-t-il Fr. 730.— d’intérêts de plus par an. Julien place Fr. 3’000.— de plus que Louis et à $2 \%$ de plus; son revenu annuel surpasse de Fr. 380.— celui de Louis. Déterminer les capitaux placés et les taux.
\end{exercice}

\begin{exercice}
Deux capitaux ont comme somme Fr. 6’000.—. Le 1er est placé à $1 \%$ de plus que le 2e et ils produisent ensemble Fr. 264.— d’intérêts par an. Si le premier était placé au taux du second, et réciproquement, ils produiraient Fr. 276.— d’intérêts. Quelles sont les deux sommes et à quels taux sont-elles placées ?
\end{exercice}

\begin{exercice}
Une somme d’argent a été partagée également entre un certain nombre de personnes. S’il y avait eu 6 personnes de plus, chacun eût reçu Fr. 2.— de moins. Au contraire, s’il avait eu 3 personnes de moins, chacune aurait reçu Fr. 2.— de plus. Déterminer le nombre de personnes, la part de chacun et la somme partagée.
\end{exercice}

\begin{exercice}
Un contremaître distribue une gratification à ses ouvriers; quand chaque ouvrier prend Fr. 1’400.— il ne reste plus que Fr. 700.—; mais si chaque ouvrier prenait Fr. 1’500.— il manquerait Fr. 2’600.—. Quel est le montant de cette gratification et combien y a-t-il d’ouvriers ?
\end{exercice}

\begin{exercice}
Jean-Paul dit à son camarade : “ Donne-moi 5 de tes billes et nous en auront autant l’un que l’autre ”. L’autre répond : “ Donne-moi 10 de tes billes et j’en aurai alors le double de ce qu’il te restera ”. Combien ont-ils de billes chacun ?
\end{exercice}

\begin{exercice}
En ajoutant 36 à un nombre de deux chiffres, on obtient le nombre renversé ; le chiffre des dizaines augmenté de 2, vaut les $\frac{3}{4}$ du chiffre des unités. Quel est ce nombre ?
\end{exercice}

\begin{exercice}
Un nombre de deux chiffres est tel qu’en y ajoutant 9, on obtient le nombre renversé et qu’en le diminuant de 9, le reste égale 4 fois la somme des chiffres. Quel est ce nombre ?
\end{exercice}

\begin{exercice}
On demandait à quelqu’un son âge, ainsi que celui de son père et de son grand-père. Il répondit : mon âge et celui de mon père font ensemble 56 ans; mon père et mon grand-père ont ensemble 100  ans, enfin mon âge et celui de mon grand-père font ensemble 80 ans. Déterminer les trois âges.
\end{exercice}

\begin{exercice}
Trois artilleurs A, B, C ont tiré des coups de canon. A et B ont tiré ensemble 20 coups de plus que C ; B et C, 32 coups de plus que A ; A et C, 28 coups de plus que B. Calculer le nombre de coups tirés par chaque artilleur.
\end{exercice}

\begin{exercice}
Un nombre de trois chiffres a 16 pour somme de ses chiffres ; en y ajoutant le nombre renversé, on obtient 1211; en retranchant du nombre le nombre renversé, on obtient 297. Quel est ce nombre ?
\end{exercice}