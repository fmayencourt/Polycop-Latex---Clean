\chapter{Limites}

\section{Limite vers un nombre}

Rappelons que certaines fonctions ont un domaine de définition différent de $\R$. C'est par exemple le cas des fonctions homographiques car le dénominateur ne peut pas être nul.

Ce chapitre s'intéresse à ce qui se passe quand on fait  rentrer dans la fonction des valeurs qui s'approchent de la valeur problématique.

\begin{definition}
Soit $f(x)$ une fonction numérique et $a$ un nombre réel. On note
$$
\lim_{x\rightarrow a} f(x)
$$
la \emph{limite de $f$ en $a$}. Il s'agit de ce vers quoi tend la fonction quand $x$ prend des valeurs de plus en plus proches de $a$.
\end{definition}

Cette définition n'est pas très "mathématique" ; la définition standard utilise une notion un peu trop avancée pour notre cours :
$$
\lim_{x\rightarrow a} f(x) = l \ssi \forall \epsilon >0 \exists \delta >0 \mbox{ tel que si }  \vaabs{x-a}< \epsilon \mbox{ alors } \vaabs{f(x) - l}<\epsilon
$$

\begin{exemple}
Nous voyons en exemple les trois cas auxquels nous serons confrontés en exercice :
\begin{enumerate}
\item La limite est un nombre
$$
\lim_{x\rightarrow 2} \frac{2x-5}{x+3}
$$
On remplace $x$ par $2$ et on obtient $\frac{-1}{5}$. Ainsi 
$$
\lim_{x\rightarrow 2} \frac{2x-5}{x+3} = \frac{-1}{5}
$$
\item La limite est indéterminée $\left( \frac{0}{0}\right)$
$$
\lim_{x\rightarrow 2} \frac{x^2 - 3 x + 2}{x^2-4}
$$
On remplace $x$ par $2$ et on obtient $\frac{0}{0}$. On factorise donc le numérateur et le dénominateur :
$\frac{(x-2)(x-1)}{(x-2)(x+2)}$. On remarque que le facteur problématique $x-2$ peut être simplifié au numérateur et au dénominateur :
$$
\lim_{x\rightarrow 2} \frac{x^2 - 3 x + 2}{x^2-4} = \lim_{x\rightarrow 2}\frac{x-1}{x+2} = \frac{1}{4}
$$
\item La limite est asymptotique $\left( \frac{\mbox{nombre}}{0}\right)$
$$
\lim_{x\rightarrow 2}\frac{x+3}{x-2}
$$
On remplace $x$ par $2$ et on obtient $\frac{5}{0}$. La limite est donc infinie, on a une asymptote verticale en $x=2$. On va de plus déterminer ce qui se passe lorsque l'on approche $2$ par des nombres plus petits et plus grands : $f(1.9) = -49$ et $f(2.1) = +51$. On a donc
$$
\lim_{x\rightarrow 2}\frac{x+3}{x-2} = \mp \infty
$$
\end{enumerate}
\end{exemple}

\section{Limite à l'infini}

On cherche maintenant à décrire le comportement de fractions lorsque $x$ devient très grand (en positif et en négatif : $\pm \infty$). 

\begin{theoreme}
Soit $f(x)$ une fonction polynomiale. Alors la limite à l'infini de cette fonction ne dépend que du monôme du plus grand degré.
\end{theoreme}

Nous ne démontrerons pas ce théorème, mais l'idée est la suivante :

Regardons par exemple la fonction $$f(x) = 2x^2 -3x+4$$

Lorsque $x$ prend des valeurs de plus en plus grande, par exemple $1'000$, on peut calculer $f(1'000) = 2'000'000 - 3'000 + 4$. Par rapport aux autres nombres, le $+4$ est insignifiant. On peut donc le négliger lorsque l'on calcule une limite. Il en va de même pour le $-3x$ par rapport au $2x^2$ : lorsque $x$ devient très grand, le $-3x$ devient négligeable par rapport au $2x$. Ainsi pour une fonction polynomiale, seul le monôme du plus haut degré est important.

\begin{exemple}
Voilà les différents types d'exemples que l'on peut rencontrer dans les exercices :
\begin{enumerate}
\item $$ \lim_{x\rightarrow \pm \infty} -3x^4+7x^3-2x+1 = \lim_{x\rightarrow \pm \infty} -3x^4$$
On va regarder séparément ce qui se passe pour $+\infty$ et $-\infty$ :
	\begin{itemize}
	\item $-3\cdot (+\infty)^4 = -3 \cdot +\infty = -\infty$, selon la règle des signes $-\cdot + = -$
	\item $-3\cdot (-\infty)^4 = -3 \cdot +\infty = -\infty$, car $(-)^4 = +$
	\end{itemize}
On a donc 
$$
\lim_{x\rightarrow \pm \infty} -3x^4+7x^3-2x+1 = -\infty
$$
\item 
$$
\lim_{x\rightarrow \pm \infty} \frac{3x^2-5x+2}{x^3-8} = \lim_{x\rightarrow \pm \infty} \frac{3x^2}{x^3}
$$
On peut simplifier la fraction : $\frac{3x^2}{x^3} = \frac{3}{x}$. Ainsi 
$$
\lim_{x\rightarrow \pm \infty} \frac{3x^2-5x+2}{x^3-8} = \lim_{x\rightarrow \pm \infty} \frac{3}{x} \left( = \frac{3}{\infty} \right) = 0
$$
On dit alors que \emph{$f$ a une asymptote horizontale en $y=0$}.
\item 
$$
\lim_{x\rightarrow \pm \infty} \frac{3x^5-2x+4}{3-7x^5} = \lim_{x\rightarrow \pm \infty} \frac{3x^5}{-7x^5}
$$
On peut simplifier la fraction $\frac{3x^5}{-7x^5} = -\frac{3}{7}$. Ainsi
$$
\lim_{x\rightarrow \pm \infty} \frac{3x^5-2x+4}{3-7x^5} = \lim_{x\rightarrow \pm \infty} -\frac{3}{7} = -\frac{3}{7}
$$
On dit alors que \emph{$f$ a une asymptote horizontale en $y=-\frac{3}{7}$}.
\end{enumerate}
\end{exemple}

\section{Exercices}

\begin{exercice}
Calculer les limites suivantes :
\begin{multicols}{2}
\begin{enumerate}
\item $\underset{x\to 0}{\mathop{\lim }}\,\frac{{{x}^{3}}-3{{x}^{2}}+2x-2}{-{{x}^{2}}+3x+2}$
\item $\underset{x\to 2}{\mathop{\lim }}\,\left( 2{{x}^{2}}+4x-5 \right)$
\item $\underset{x\to -3}{\mathop{\lim }}\,\frac{2x-1}{{{x}^{2}}}$
\item $\underset{x\to -1}{\mathop{\lim }}\,\frac{{{x}^{3}}-3{{x}^{2}}+4}{3{{x}^{3}}-18{{x}^{2}}+36x-24}$
\end{enumerate}
\end{multicols}
\end{exercice}

\begin{exercice}
Calculer les limites suivantes et préciser si la fonction admet une asymptote verticale :
\begin{multicols}{2}
\begin{enumerate}
\item $\underset{x\to 1}{\mathop{\lim }}\,\frac{{{x}^{3}}+2{{x}^{2}}-x-2}{{{x}^{2}}+x-2}$
\item $\underset{x\to 1}{\mathop{\lim }}\,\frac{{{x}^{3}}-3x+2}{{{x}^{2}}-6x+5}$
\item $\underset{x\to 5}{\mathop{\lim }}\,\frac{{{x}^{3}}-3x+2}{{{x}^{2}}-6x+5}$
\item $\underset{x\to 2}{\mathop{\lim }}\,\frac{{{x}^{3}}-3{{x}^{2}}+4}{3{{x}^{3}}-18{{x}^{2}}+36x-24}$
\item $\underset{x\to \frac{1}{2}}{\mathop{\lim }}\,\frac{8{{x}^{3}}+2{{x}^{2}}-5x+1}{8{{x}^{3}}+10{{x}^{2}}-11x+2}$
\item $\underset{x\to \frac{1}{4}}{\mathop{\lim }}\,\frac{8{{x}^{3}}+2{{x}^{2}}-5x+1}{8{{x}^{3}}+10{{x}^{2}}-11x+2}$
\item $\underset{x\to 1}{\mathop{\lim }}\,\frac{{{x}^{4}}-2{{x}^{3}}+2{{x}^{2}}-2x+1}{{{x}^{4}}-2{{x}^{3}}+{{x}^{2}}}$
\item $\underset{x\to 0}{\mathop{\lim }}\,\frac{{{x}^{4}}-2{{x}^{3}}+2{{x}^{2}}-2x+1}{{{x}^{4}}-2{{x}^{3}}+{{x}^{2}}}$
\item $\underset{{}}{\mathop{\underset{x\to 0}{\mathop{\lim }}\,}}\,\frac{{{x}^{4}}-5{{x}^{3}}+6{{x}^{2}}+4x-8}{{{x}^{3}}-4{{x}^{2}}+4x}$
\item $\underset{x\to -2}{\mathop{\lim }}\,\frac{{{x}^{4}}-8{{x}^{2}}+16}{{{x}^{3}}+2{{x}^{2}}-4x-8}$
\end{enumerate}
\end{multicols}
\end{exercice}

\begin{exercice}
Calculer les limites suivantes et préciser si la fonction admet une asymptote horizontale :
\begin{multicols}{2}
\begin{enumerate}
\item $\underset{x\to \pm \infty }{\mathop{\lim }}\,\left( 3{{x}^{2}}-5x-4 \right)$
\item $\underset{x\to \pm \infty }{\mathop{\lim }}\,\left( 2{{x}^{3}}-8{{x}^{2}}+4x+7 \right)$
\item $\underset{x\to \pm \infty }{\mathop{\lim }}\,\left( -{{x}^{2}}+1 \right)$
\item $\underset{x\to \pm \infty }{\mathop{\lim }}\,\left( -2{{x}^{5}}+{{x}^{4}}+4{{x}^{3}}+3{{x}^{2}}-5 \right)$
\item $\underset{x\to \pm \infty }{\mathop{\lim }}\,\frac{{{x}^{3}}+2{{x}^{2}}-2}{{{x}^{2}}+x-3}$
\item $\underset{x\to \pm \infty }{\mathop{\lim }}\,\frac{-2{{x}^{3}}+{{x}^{2}}-1}{5{{x}^{2}}-3x+4}$
\item $\underset{x\to \pm \infty }{\mathop{\lim }}\,\frac{5{{x}^{3}}+{{x}^{2}}+4x+7}{5{{x}^{3}}+3{{x}^{2}}-x+4}$
\item $\underset{x\to \pm \infty }{\mathop{\lim }}\,\frac{{{x}^{2}}+2x-4}{{{x}^{3}}-x+5}$
\item $\underset{x\to \pm \infty }{\mathop{\lim }}\,\frac{2{{x}^{4}}+{{x}^{3}}-5{{x}^{2}}-2x+3}{5{{x}^{4}}+3{{x}^{3}}-{{x}^{2}}+x+7}$
\item $\underset{x\to \pm \infty }{\mathop{\lim }}\,\frac{-2{{x}^{3}}-4{{x}^{2}}+1}{3{{x}^{3}}-5{{x}^{2}}+2x-7}$
\item $\underset{x\to \pm \infty }{\mathop{\lim }}\,\frac{-{{x}^{4}}+2{{x}^{3}}-5{{x}^{2}}-2x+1}{{{x}^{3}}+{{x}^{2}}-3}$
\item $\underset{x\to \pm \infty }{\mathop{\lim }}\,\frac{1}{{{x}^{2}}}$
\end{enumerate}
\end{multicols}
\end{exercice}