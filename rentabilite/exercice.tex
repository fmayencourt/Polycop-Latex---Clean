\begin{exercice}
Francine gère un comptoir où elle vend des soupes. Elle a calculé que son coût de production quotidien est constitué de Fr. 42.— de frais fixes plus Fr. 0,60 par soupe qu’elle prépare. Elle vend ses soupes Fr. 1.— chacune. Trouver le nombre de soupes qu’elle doit vendre dans une journée pour atteindre le seuil de rentabilité.
\end{exercice}

\begin{exercice}
Un maraîcher sait qu’il peut vendre toute sa production de navets en les vendant Fr. 0,40 chacun. Il estime qu’il a des frais fixes de Fr. 100.— par jour et qu’il lui en coûte Fr. 0,20 pour produire chaque navet. Trouver son seuil de rentabilité.
Un vendeur de machines agricoles propose à ce maraîcher l’achat d’une machine qui réduira le coût de production à Fr. 0,10 par navet mais qui augmentera ses frais fixes à Fr. 180.— par jour. Analyser le seuil de rentabilité et les coûts de production pour aider le producteur à prendre une décision.
\end{exercice}

\begin{exercice}
Un marchand achète d’un grossiste des bas pour hommes au prix de Fr. 2.— chaque paire. Si les frais fixes de fonctionnement de la mercerie sont de Fr. 148.— par jour, à quel prix devra-t-il vendre chaque paire de bas pour avoir un seuil de rentabilité de 37 paires par jour ?
\end{exercice}

\begin{exercice}
Un éditeur décide de publier un ouvrage de mathématiques. Les coûts qu’il doit assumer sont formés de frais fixes (composition, montage) s’élevant à Fr. 12'240.— et de frais variables (impression, droits d’auteurs) s’élevant à Fr. 9.— par volume. S’il vend ses livres Fr. 21.— chacun, trouver son seuil de rentabilité.
Un nouveau procédé de composition permet de baisser les frais fixes à Fr. 9'660.—. Par contre, dans ce cas, l’impression fait grimper les coûts variables à Fr. 11.— par livre. Analyser le seuil de rentabilité.
\end{exercice}

\begin{exercice}
On désire donner de la fabrication en sous-traitance ; trois entreprises font les propositions suivantes :
A : Fr. 150.— la pièce
B : Fr.   75.— la pièce + un investissement unique de Fr. 1'000.—
C : Fr.   50.— la pièce + un investissement unique de Fr. 2'000.—
Illustrer ces trois propositions par un graphique, calculer les seuils de rentabilité et enfin analyser concrètement les trois propositions.
\end{exercice}

\begin{exercice}
Un propriétaire encaveur produit chaque année 25'000 bouteilles. Il vend Fr. 145.–– les dix bouteilles.
Il paie annuellement les frais fixes suivants : 
amortissement des installations	Fr. 12'100.—
assurances	Fr.   9'500.—
chauffage	Fr.   5'300.––
divers	Fr.   5'200.––
 
 
Ainsi que les frais variables suivants par bouteille : 
bouteille	Fr. 0.45
bouchon	Fr. 0.10
étiquette	Fr. 0.30
acide sulfurique	Fr. 0.05
petit matériel	Fr. 0.55
vin	Fr. 7.05
 

Quel doit être le nombre de bouteilles à vendre pour couvrir ses frais fixes ?
S'il vend les 25'000 bouteilles, quel bénéfice fera-t-il ?
Après deux années d'exploitation, il décide de baisser le prix de vente de 10 %. Quel sera le nouveau seuil de rentabilité ?
Un pépiniériste prépare une plantation de thuyas ; pour cela, il loue un terrain agricole d’un hectare sur lequel il plantera ses jeunes arbustes ; la location annuelle du terrain s’élève à Fr. 0.25 le m2 (1 hectare = 10'000 m2).
\end{exercice}

\begin{exercice}
L’entretien des jeunes arbres (heures de travail, engrais, machines, …) s’élève à Fr. 0.75.–– par arbuste et par année.

Il s’écoule exactement quatre années entre la date de la plantation et la date de la vente des thuyas.

Le prix de vente d’un thuya étant de Fr. 5.––, calculer le nombre de thuyas que devra vendre le pépiniériste pour qu’il puisse rentrer dans ses frais.

Sachant qu’il a planté et cultivé 60'000 thuyas et qu’après 4 ans il ne pourra en vendre que le $90 \%$, calculer le revenu annuel moyen que lui procurera la vente des arbres ?
\end{exercice}

\begin{exercice}
Les CFF proposent 3 possibilités de paiement :

Payer le trajet en plein tarif 
Acheter un abonnement demi-tarif de Fr. 150.— et payer le trajet à moitié prix
Acheter un abonnement général de Fr. 1050.— et ne pas payer le trajet 

Monsieur Blanc prend le train fréquemment en direction de Sion. 
Son trajet aller-retour lui coûte Fr. 12,50 en plein tarif. Il hésite entre payer son billet toutes les semaines ou opter pour un abonnement (demi-tarif ou général). 

Que conseiller à Monsieur Blanc ?

Représenter la situation graphiquement.
\end{exercice}