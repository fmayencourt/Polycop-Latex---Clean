\begin{exercice}
Quel est le capital qui, placé à intérêts composés et à $5 \%$ pendant 10 ans, est devenu Fr. 12'640.— ?
\end{exercice}

\begin{exercice}
Une somme de Fr. 10'000.— placée à intérêts composés est devenue Fr. 20'360.— après 15 ans. Calculer le taux des intérêts composés.
\end{exercice}

\begin{exercice}
Une somme de Fr. 40'000.— placée à intérêts composés à $4.5 \%$ est devenue Fr. 67'835.70, pendant combien d’années a-t-elle été placée ?		
\end{exercice}

\begin{exercice}
A quel taux faudrait-il placer une somme de Fr. 10'000.—, à intérêts simples, pour qu’après 6 ans elle eût rapporté autant que cette même somme placée à intérêts composés à $4 \%$ pendant le même temps ?	
\end{exercice}

\begin{exercice}
Pendant combien d’années faudrait-il placer un capital à intérêts composés à $4.5 \%$ pour tripler ce capital ?
\end{exercice}

\begin{exercice}
En payant une dette douze ans avant son échéance, on a obtenu une réduction de Fr. 2'570.62. Calculer la valeur nominale de la dette et la somme payée pour l’acquitter. Le taux des intérêts composés est de 5 %.
\end{exercice}

\begin{exercice}
Un capital de Fr. 1'025'000.— a été placé à $4 \%$ pendant 5 ans. A ce moment-là, son propriétaire retire pour ses besoins personnels Fr. 222'069.20. Le montant restant est replacé au taux de $6 \%$ pendant 10 ans. 
Quel est le capital après ce temps de placement ?
\end{exercice}

\begin{exercice}
Un de mes ancêtres a placé Fr. 100.—. Ce montant est devenu Fr. 49'410'548.—. Combien de temps l’ai-je placé, si le taux de capitalisation a été de $6 \%$ ?
\end{exercice}

\begin{exercice}
Mon père a remis une certaine somme dont il disposait le jour de ma naissance à sa banque. Le 01.01.2003 je me retrouve avec un capital de Fr. 30'000.— et j’ai 30 ans. Le taux moyen du placement a été de $5,5 \%$.
Quel montant mon père a-t-il placé ?
\end{exercice}

\begin{exercice}
Le taux moyen de renchérissement du coût de la vie a été dans un certain pays au cours des dix dernières années de $5.8 \%$ annuellement. Un ouvrier gagnait Fr. 4'267.80 au 1er janvier 1993 dans une entreprise qui a toujours indexé intégralement ses salaires. Quel est son salaire au 1er janvier 2003 ?
\end{exercice}

\begin{exercice}
Tu disposes de deux capitaux. Le premier atteint un montant de Fr. 2'198'852.— et le second se monte à Fr. 2’000'000.—. Tu places le premier à $5 \%$ et le second à $6 \%$. Après combien de temps le deuxième capital sera-t-il d’un montant égal au premier ?
\end{exercice}

\begin{exercice}
Un capital de Fr. 80'000.— a été placé comme suit :
\begin{itemize}
\item pendant  5 ans,  à $7 \%$
\item puis, pendant 6 ans, à $7.5 \%$
\item puis, pendant 11 ans, à $8 \%$
\item puis enfin, pendant 3 ans, à $4 \%$
\end{itemize}
Combien retirera-t-on après ces 25 ans ? Si pendant tout le placement, on avait voulu avoir un taux unique de rendement et toucher en finale le même montant, quel aurait dû être ce taux ?
\end{exercice}

\begin{exercice}
Un capital de Fr. 1'000'000.— est devenu Fr. 2'000'000.— après 10 ans. Quel a été le taux de placement ?
\end{exercice}

\begin{exercice}
On place à intérêts composés une somme de Fr. 250'000.— à $8 \%$ pendant 5 ans; puis, on déplace le montant obtenu auprès d’une autre banque qui a trouvé un placement rétribué à $10 \%$. Après 6 nouvelles années, nous retirons Fr. 250'751.—. Que reste-t-il sur notre compte après ce retrait ?
\end{exercice}

\begin{exercice}
Si, le 1er janvier de l'an 1, un Romain avait placé pour toi à $0.5 \%$ l’équivalent de Fr. 100.—, de quelle somme disposerais-tu le 1er janvier 2001 ?
\end{exercice}

\begin{exercice}
En remboursant une dette de Fr. 25'000.— trois ans avant son échéance on a obtenu un escompte de Fr. 4'303.75. Calculer le taux des intérêts composés.
\end{exercice}

\begin{exercice}
On dispose de Fr. 500'000.— à placer. Deux banquiers contactés te proposent ceci :
\begin{itemize}
\item 1e banquier : Placement de  Fr. 500'000.–– pendant 10 ans à $5 \%$, puis pendant 5 ans à $7 \%$
\item 2e banquier :  1e  tranche de 200'000.–– pendant 15 ans à $5 \%$
2e  tranche de 300'000.–– pendant 10 ans à $6 \%$, puis pendant 5 ans à $5 \%$
\end{itemize}
Quelle proposition choisirais-tu si tu veux que ton capital te rapporte le maximum ? Quel sera ton gain si tu choisis la meilleure solution ?
\end{exercice}

\begin{exercice}
Une somme de Fr. 100'000.— est payable dans 10 ans. Déterminer sa valeur :
\begin{itemize}
\item aujourd’hui
\item dans 3 ans
\item dans 15 ans
\item dans 12 ans et 7 mois
\end{itemize}
Taux annuel de $3.5 \%$.
\end{exercice}

\begin{exercice}
Le 1er janvier 1990 une personne a placé la somme de Fr. 10'000.— sur compte d'épargne. Le 1er janvier 1996 elle effectue un deuxième placement sur ce même compte. Calculer le montant de ce deuxième versement sachant qu'au 31 décembre 2002 le montant du compte est de Fr. 36'774.—.
\end{exercice}



