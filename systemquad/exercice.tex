\begin{exercice}
Pour chaque système :
\begin{itemize}
\item Résoudre algébriquement le système.
\item Étudier toutes les caractéristiques des deux fonctions.
\item Tracer les deux courbes sur un même système d’axes (unités des axes : 2 carrés).
\item Vérifier graphiquement les solutions obtenues algébriquement.
\end{itemize}
\begin{enumerate}
\item $$\left\{ \begin{array}{l}
    y=x+1 \\ 
   y={{x}^{2}}-2x-3 \\ 
	\end{array} \right.$$
\item $$\left\{ \begin{array}{l}
    y=-x-\frac{3}{2} \\ 
   y=\frac{{{x}^{2}}}{2}+x-4 \\ 
	\end{array} \right.$$
\item $$\left\{ \begin{array}{l}
    y=-x+2 \\ 
   y=\frac{x+2}{-x+4} \\ 
	\end{array} \right.$$
\item $$\left\{ \begin{array}{l}
    y=\frac{{{x}^{2}}}{2}+2x+5 \\ 
   y=-\frac{{{x}^{2}}}{2}-4x \\ 
	\end{array} \right.$$
\item $$\left\{ \begin{array}{l}
    y=-2x-1 \\ 
   y={{x}^{2}}+2x+3 \\ 
	\end{array} \right.$$
\item $$\left\{ \begin{array}{l}
    y=\frac{{{x}^{2}}}{2}+3x \\ 
   y=\frac{-x}{x+3} \\ 
	\end{array} \right.$$
\end{enumerate}
\end{exercice}

