\chapter{Factorisation}

\begin{definition}
Un polynôme est dit \emph{factorisé} s'il est écrit sous forme d'un produit de polynômes de degré plus petit.
\end{definition}

\begin{exemple}
Le polynôme $2(x+1)(x+3)$ est la forme factorisée du polynôme $2x^2 + 8x + 6$ car
$$
2(x+1)(x+3) = 2(x^2 + x + 3x + 3) = 2x^2 + 8x + 6.
$$

Le polynôme $2(x^2 +3) + 8x$ n'est pas une forme factorisée du polynôme $2x^2 + 8x + 6$ car il s'agit de l'addition de $2(x^2 + 3)$ et de $8x$ et non d'une multiplication.
\end{exemple}

\section{Mise en évidence}\index{factorisation!mise en évidence}

Dans le chapitre précédent, nous avons vu comment multiplier un monôme et un polynôme (définition~\ref{distribuer}). \'A présent nous allons voir comment effectuer l'opération dans l'autre sens.

\begin{exemple}
Factoriser par mise en évidence le polynôme
$$
4x^3y^2 - 6 x^2y^3 + 42xy^4.
$$

Commençons par regarder les diviseurs en communs des parties numériques :
$$
\underbrace{4}_{2\cdot 2}x^3y^2 - \underbrace{6}_{2\cdot 3} x^2y^3 + \underbrace{42}_{2\cdot 21}xy^4
=2\left(2x^3y^2 - 3 x^2y^3 + 21xy^4\right),
$$
Puis les lettres communes aux parties littérales. On va prendre chaque fois le plus petit exposant :
$$
2\left(2\underbrace{x^3y^2}_{xy^2 \cdot x} - 3 \underbrace{x^2y^3}_{xy^2 \cdot xy} + 21\underbrace{xy^4}_{xy^2 \cdot y^3}\right) = 2xy^2\left(2x-3xy+21y^3\right).
$$
En effet, si l'on prend un exposant plus grand, on ne va pas pouvoir écrire tous les termes avec cet exposant : $x^2y^2 \cdot x^?y^2 = xy^4$.
\end{exemple}

\begin{remarque}
Dans certains cas (qui arriveront dans les exercices et les examens), on peut aussi mettre en évidence des polynômes.
$$
\underbrace{2x(x+1)}_{(x+1)\cdot 2x} - \underbrace{(x+1)^2}_{(x+1)\cdot(x+1)} = (x+1)\left[2x - (x+1) \right]
$$
\end{remarque}

\section{Identités remarquables}\index{factorisation!identités remarquables}

Nous avons vu que les identités remarquables sont un raccourci pour développer un polynôme (section~\ref{identites}). Elles sont donc naturellement, si on les prend dans l'autre sens, un raccourci pour factoriser un polynôme.

\begin{exemple}
$$
\begin{array}{rcccccccccc}
9x^2 + 30 x + 25 & = & (3x)^2 & +2 &3x& 5 &+ & (5)^2 & = & (3x &+ 5)^2    \\
& &\downarrow & &\downarrow & \downarrow & & \downarrow& &\uparrow & \uparrow \\
& & a^2 & +2 & a & b &+ & b^2 & = & (a & +b)^2   \\
\end{array}
$$
\end{exemple}

\subsection{Comment déterminer l'identité à utiliser ?}

Nous avons plusieurs sortes d'identités : les carrées et les cubiques et plusieurs sortes de développement : deux, trois, quatre ou six termes. Si l'on conna\^it par coeur les différentes identités, ces indices sont suffisants pour savoir laquelle pourrait être utilisée.

\begin{exemple}
\begin{enumerate}
\item $x^4 - 9y^2$, il y a deux termes en soustraction. Ainsi il s'agit soit de l'identité $a^2-b^2$, soit de l'identité $a^3 - b^3$. Puisque $x^4 = (x^2)^2$ et $9y^2 = (3y)^2$, il doit s'agir de l'identité $a^2 - b^2$. Ainsi
$$
x^4 - 9y^2 = (x^2-3y)(x^2+ 3y).
$$
\item $4x^2 - 12xy + 9y^2$, il y a trois termes. Ainsi il s'agit soit de l'identité $a^2 + 2ab + b^2$, soit de $a^2 - 2ab + b^2$. Au vue des signes, il nous faut utiliser l'identité avec un $-$. On calcule que $4x^2 = (2x)^2$, que $9y^2 = (3y)^2$ et que $12xy = 2 \cdot 2x \cdot 3y$. Ainsi
$$
4x^2 - 12xy + 9y^2 = (2x-3y)^2.
$$
\item $8x^3 + 6xy^2 - y^3 - 12x^2$ y, il y a quatre termes. Ainsi il s'agit d'une des deux identités du cube. Grâce aux signes, on comprend qu'il doit s'agir de celle du type $(a-b)^3$. Les deux cubes présents sont $8x^3 = (2x)^3$ et $y^3$. De plus on vérifie que $3\cdot (2x)^2 \cdot y$ et $3 \cdot 2x \cdot y^2$ sont bien les autres termes et donc
$$
x^3 + 6xy^2 - y^3 - 12x^2 = (2x-y)^3
$$
\end{enumerate}
\end{exemple}

\section{Trinôme du second degré}\index{factorisation!sommme-produit}

Dans cette partie, on s'intéressera à la factorisation d'un polynôme du type
$$
ax^2 + bx + c,
$$
comme par exemple $2x^2 + 7x + 5$ ou $x^2 -2x - 35$.

Mais on peut séparer ces factorisations en deux familles :

\subsection{Coefficient de $x^2$ vaut $1$}

On factorise ce genre de polynômes grâce au théorème suivant :

\begin{theoreme}
Le polynôme 
$
x^2 + bx + c
$
se factorise en 
$
(x+m)(x+n)
$
où 
$$
\left\{
\begin{array}{lcl}
m+n &=& b \\
m\cdot n &=& c
\end{array}
\right.
$$
\end{theoreme}

\begin{proof}
Développons 
$$
(x+m)(x+n) = x^2 + mx + nx + mn = x^2 + (m+n)x + mn.
$$
Ainsi si
$$
\left\{
\begin{array}{lcl}
m+n &=& b \\
m\cdot n &=& c
\end{array}
\right.
$$
il s'agit bien de la factorisation du polynôme 
$x^2 + bx + c$.
\end{proof}

\subsection{Coefficient de $x^2$ est différent de $1$}\label{factorisertrinome}

La factorisation de ce genre de polynôme est plus complexe, pour le comprendre, mieux vaut passer par un exemple :

\begin{exemple}
Factoriser 
$$
6x^2 +x -1.
$$
Il faut à nouveau trouver $m$ et $n$, mais cette fois le produit utilise le coefficient de $x^2$ :
$$
\left\{
\begin{array}{lcl}
m+n &=& 1 \mbox{ car }1\cdot x  \\
m\cdot n &=& 6\cdot (-1) = -6 \mbox{ car } 6\cdot x^2 \mbox{ et } -1
\end{array}
\right.
\Leftrightarrow
\left\{
\begin{array}{lcl}
m &=& 3 \\
n &=& -2
\end{array}
\right.
$$
Une fois $m$ et $n$ trouvés, il faut encore travailler un peu pour avoir la factorisation :
$$
\begin{array}{cccccc}
6x^2& + & x && -1 &=\\
&&\swarrow\searrow &&&\\
6x^2 & +3x && -2x &-1 &=\\
(2x+1) \cdot & 3x & + & (2x+1)&\cdot (-1) &=\\
\downarrow &&\searrow&&\swarrow & \\
(2x+1) \cdot &&& (3x-1) && \\
\end{array}
$$
\end{exemple}

\section{Factorisation par Hörner}\index{factorisation!Hörner}

Lorsque toutes les autres méthodes ne fonctionnent pas, on se tourne vers une factorisation à l'aide du schéma de Hörner :

\begin{enumerate}
\item trouver une valeur $a$ de $x$ qui annule le polynôme, c'est-à-dire trouver une valeur qui, lorsqu'on remplace $x$, donne zéro comme réponse au calcul. On ne regarde que les diviseurs positifs et négatifs du terme sans~$x$
\item effectuer la division par $(x-a)$
\item la factorisation est donnée par la réponse de la division et $(x-a)$
\end{enumerate}

\begin{exemple}
Factoriser $6x^3 - 47 x^2 + 90 x - 25$ :
\begin{enumerate}
\item On cherche une valeur de $x$ qui annule le polynôme
	\begin{itemize}
	\item $x=1$ : $6 \cdot 1^3 - 47 \cdot 1^2 + 90 \cdot 1 - 25 \neq 0$
	\item $x=5$ : $6 \cdot 5^3 - 47 \cdot 5^2 + 90 \cdot 5 - 25 = 0$
	\end{itemize}
\item On effectue la division par $(x-5)$ grâce au schéma de Hörner et on trouve $(6x^2 - 17 x + 5)$
\item la factorisation est donc
$$
(x-5)(6x^2 - 17 x + 5)
$$
\end{enumerate}
\end{exemple}

\section{Exercices}


\begin{exercice}Mettre en évidence les facteurs communs :
\begin{multicols}{2}
\begin{enumerate}
\item $10a{{c}^{2}}+25ac+15{{a}^{2}}c$
\item $12{{x}^{2}}{{y}^{2}}-18x{{y}^{3}}+24{{x}^{3}}y$
\item $12{{a}^{2}}{{x}^{3}}-30{{a}^{3}}{{x}^{2}}+18a{{x}^{4}}$
\item $a(x+y)+b(x+y)+(x+y)$
\item $(a-b)+x(a-b)-y(b-a)$
\item $44a{{x}^{n}}-286{{a}^{2}}{{x}^{n+1}}+66{{a}^{3}}{{x}^{n+2}}$
\item ${{x}^{m+n}}{{y}^{m}}-{{x}^{2n}}{{y}^{m+n}}-{{x}^{n}}{{y}^{2m}}$
\item $7{{x}^{m+3}}{{y}^{n-2}}+14{{x}^{m}}{{y}^{n+1}}+21{{x}^{m-3}}{{y}^{n+4}}$
\end{enumerate}
\end{multicols}
\end{exercice}

\begin{exercice} Factoriser en utilisant les identités remarquables :
\begin{multicols}{2}
\begin{enumerate}
\item ${{a}^{2}}-9$
\item ${{b}^{2}}-{{a}^{2m}}$
\item ${{x}^{3}}y-x{{y}^{3}}$
\item ${{a}^{2}}-16{{b}^{2}}$
\item ${{a}^{4}}-9{{b}^{2}}$
\item ${{a}^{2}}-25{{x}^{2}}$
\item $32{{a}^{2}}-2{{b}^{4}}$
\item ${{a}^{2}}{{x}^{2}}-{{b}^{2}}{{x}^{2}}$
\item $4{{x}^{2}}-16{{a}^{2}}$
\item ${{a}^{2}}{{b}^{2}}{{c}^{2}}-{{m}^{2}}$
\item $50{{x}^{4}}-2{{y}^{2}}$
\item $256{{x}^{2}}-64{{a}^{4}}$
\item ${{a}^{2}}{{x}^{2}}-81{{x}^{2}}$
\item $16{{x}^{2}}{{y}^{2}}-121{{y}^{4}}$
\item ${{x}^{4}}{{y}^{2}}-{{x}^{2}}{{y}^{4}}$
\item $3{{a}^{3}}x-3a{{x}^{3}}$
\item $150{{a}^{6}}{{b}^{2}}-24{{a}^{2}}{{b}^{2}}$
\item $37{{a}^{5}}x-333{{a}^{3}}x$
\item ${{x}^{4}}-81$
\item $81{{x}^{4}}-625{{a}^{4}}$
\item $32{{x}^{4}}-2{{a}^{4}}$
\item $3a{{x}^{4}}-3a{{y}^{4}}$
\item $3{{x}^{5}}-48x{{y}^{8}}$
\item ${{x}^{11}}{{y}^{4}}-{{x}^{5}}{{y}^{10}}$
\item ${{m}^{3}}+{{n}^{6}}$
\item $27{{a}^{9}}{{b}^{6}}-8{{c}^{3}}$
\item $6x{{y}^{3}}-6x$
\item $125{{x}^{3}}-1$
\item $192{{x}^{6}}{{y}^{6}}-2187{{z}^{6}}$
\item ${{a}^{7}}b-a{{b}^{7}}$
\item ${{x}^{10}}y-x{{y}^{10}}$
\item ${{a}^{5m}}-9{{a}^{3m}}{{b}^{2n}}$
\item ${{a}^{3m+3}}+{{b}^{3n+6}}$
\item $343{{x}^{3}}-512{{y}^{6}}$
\item $64{{x}^{6}}-1$
\item $729{{a}^{6}}-64$
\item ${{(a+b)}^{2}}-{{c}^{2}}$
\item ${{(a+b)}^{2}}-{{(x-y)}^{2}}$
\item ${{(5a+2b)}^{2}}-{{(2b-5a)}^{2}}$
\item ${{(x+a)}^{2}}-{{(3x-2a)}^{2}}$
\item ${{(4x-y)}^{2}}-{{(4y-x)}^{2}}$
\item ${{(a+b+c)}^{2}}-{{(a-b-c)}^{2}}$
\item ${{(x+1)}^{2}}-{{(x-1)}^{2}}$
\item ${{(a+b)}^{3}}+{{(a-b)}^{3}}$
\item ${{(2x-3y)}^{3}}-{{(x+2y)}^{3}}$
\end{enumerate}
\end{multicols}
\end{exercice}

\begin{exercice}
Factoriser en utilisant les identités remarquables :
\begin{multicols}{2}
\begin{enumerate}
\item ${{a}^{2}}+4ab+4{{b}^{2}}$
\item $9{{a}^{2}}-12ab+4{{b}^{2}}$
\item $4{{a}^{2}}-4a+1$
\item ${{a}^{2}}-a+\frac{1}{4}$
\item ${{x}^{4}}+2{{x}^{2}}+1$
\item ${{x}^{6}}+6{{x}^{3}}+9$
\item $a{{b}^{2}}-2abc+a{{c}^{2}}$
\item $\frac{{{x}^{2}}}{16}-\frac{3xy}{2}+9{{y}^{2}}$
\item $4{{x}^{4}}+{{x}^{2}}y+\frac{{{y}^{2}}}{16}$
\item $9{{a}^{4}}{{b}^{2}}-6{{a}^{2}}bc+{{c}^{2}}$
\item $\frac{9{{a}^{4}}b}{4}-{{a}^{3}}{{b}^{2}}+\frac{{{a}^{2}}{{b}^{3}}}{9}$
\item $9{{a}^{2m+2}}-\frac{3{{a}^{m+2}}{{y}^{n}}}{2}+\frac{{{a}^{2}}{{y}^{2n}}}{16}$
\item $50{{a}^{6}}{{b}^{2}}{{c}^{2}}+72{{a}^{2}}{{b}^{8}}{{c}^{2}}+120{{a}^{4}}{{b}^{5}}{{c}^{2}}$
\item $\frac{4}{3}{{a}^{7}}x+8{{a}^{4}}{{x}^{5}}+12a{{x}^{9}}$
\item $144{{x}^{5}}y+324{{x}^{3}}{{y}^{3}}+432{{x}^{4}}{{y}^{2}}$
\item $24{{a}^{6}}b{{c}^{3}}+54{{a}^{4}}{{b}^{3}}{{c}^{3}}-72{{a}^{5}}{{b}^{2}}{{c}^{3}}$
\item $175{{a}^{2}}{{x}^{2m}}+280{{a}^{2}}{{x}^{m}}{{y}^{n}}+112{{a}^{2}}{{y}^{2n}}$
\item $49{{x}^{2}}{{y}^{8}}+25{{a}^{6}}{{b}^{4}}-70{{a}^{3}}{{b}^{2}}x{{y}^{4}}$
\end{enumerate}
\end{multicols}
\end{exercice}

\begin{exercice} Factoriser en utilisant les identités remarquables :
\begin{multicols}{2}
\begin{enumerate}
\item ${{a}^{3}}-6{{a}^{2}}b+12a{{b}^{2}}-8{{b}^{3}}$
\item ${{x}^{3}}+9{{x}^{2}}+27x+27$
\item ${{a}^{6}}{{x}^{3}}+3x-3{{a}^{3}}{{x}^{2}}-\frac{1}{{{a}^{3}}}$
\item $150x{{y}^{2}}-60{{x}^{2}}y+8{{x}^{3}}-125{{y}^{3}}$
\item $27{{a}^{6}}{{x}^{3}}-108{{a}^{4}}b{{x}^{2}}+144{{a}^{2}}{{b}^{2}}x-64{{b}^{3}}$
\item $108{{x}^{3}}+432{{x}^{2}}+576x+256$
\item $1000{{a}^{3}}-1200{{a}^{2}}b+480a{{b}^{2}}-64{{b}^{3}}$
\item $250{{x}^{6}}{{y}^{9}}+150{{x}^{4}}{{y}^{7}}{{z}^{2}}+30{{x}^{2}}{{y}^{5}}{{z}^{4}}+2{{y}^{3}}{{z}^{6}}$
\end{enumerate}
\end{multicols}
\end{exercice}

\begin{exercice} Factoriser en utilisant les identités remarquables :
\begin{multicols}{2}
\begin{enumerate}
\item ${{a}^{2}}+{{b}^{2}}+{{c}^{2}}-2ab-2ac+2bc$
\item ${{x}^{2}}+{{y}^{2}}-2xy+2x-2y+1$
\item ${{x}^{8}}-2{{x}^{6}}-2{{x}^{5}}+{{x}^{4}}+2{{x}^{3}}+{{x}^{2}}$
\item ${{x}^{4}}+2{{x}^{3}}+3{{x}^{2}}+2x+1$
\end{enumerate}
\end{multicols}
\end{exercice}

\begin{exercice} Factoriser les trinômes suivants, (coefficient de $x^2  = 1$) :
\begin{multicols}{2}
\begin{enumerate}
\item ${{x}^{2}}-8x+12$
\item ${{x}^{2}}-14x+13$
\item ${{x}^{2}}-22x+85$
\item ${{x}^{2}}-4x-5$
\item ${{x}^{2}}+10x+16$
\item ${{x}^{2}}+115x+1500$
\item ${{x}^{2}}-4x-32$
\item ${{x}^{2}}+5x-14$
\item ${{x}^{2}}+20x+19$
\item ${{x}^{2}}-4x-12$
\item ${{x}^{2}}-9999x-10000$
\item ${{x}^{2}}+x-132$
\end{enumerate}
\end{multicols}
\end{exercice}

\begin{exercice} Factoriser les trinômes suivants, (coefficient de $x^2\neq 1$) :
\begin{multicols}{2}
\begin{enumerate}
\item $2{{x}^{2}}+9x+7$
\item $2{{x}^{2}}+5x+2$
\item $9{{x}^{2}}-25x+16$
\item $4{{x}^{2}}+x-5$
\item $3{{x}^{2}}+x-2$
\item $4{{x}^{2}}+9x+2$
\item $5{{x}^{2}}-27x+10$
\item $4{{x}^{2}}+4x+1$
\item $7{{x}^{2}}-8x+1$
\item $7{{x}^{2}}-9x-10$
\item $2{{x}^{2}}-13x+15$
\item $5{{x}^{2}}-3x-2$
\item $5{{x}^{2}}+6x+1$
\item $4{{x}^{2}}-5x-6$
\item $7{{x}^{2}}-34x-5$
\item $3{{x}^{2}}+x-2$
\item $9{{x}^{2}}+6x-3$
\item $4{{x}^{2}}-x-5$
\item $11{{x}^{2}}+28x-15$
\item $6{{x}^{4}}+5{{x}^{2}}+1$
\item $21{{x}^{4}}-8{{x}^{2}}-5$
\item $45{{x}^{2}}-39xy-6{{y}^{2}}$
\item $12{{x}^{2}}+34xy+10{{y}^{2}}$
\item $2{{x}^{4}}+{{x}^{2}}{{y}^{2}}-3{{y}^{4}}$
\end{enumerate}
\end{multicols}
\end{exercice}

\begin{exercice}
Effectuer les divisions algébriques suivantes (divisions sans reste) :
\begin{multicols}{2}
\begin{enumerate}
\item $(35{{x}^{3}}+47{{x}^{2}}+13x+1)\div (5x+1)$
\item $(6{{x}^{3}}-17{{x}^{2}}+14x-3)\div (2x-3)$
\item $({{a}^{7}}-3{{a}^{6}}+{{a}^{5}}-4{{a}^{2}}+12a-4)\div ({{a}^{5}}-4)$
\item $(10{{a}^{3}}{{b}^{2}}+{{a}^{5}}-5{{a}^{4}}b-10{{a}^{2}}{{b}^{3}}-{{b}^{5}}+5a{{b}^{4}})\div (a-b)$
\item $(3{{a}^{4}}-7{{a}^{3}}-18{{a}^{2}}+28a+24)\div (3{{a}^{2}}+8a+4)$
\item $(14{{a}^{4}}-27{{a}^{3}}b-3a{{b}^{3}}+21{{a}^{2}}{{b}^{2}}-2{{b}^{4}})\div (2{{a}^{2}}-3ab+2{{b}^{2}})$
\item $(-25{{a}^{3}}x-2{{a}^{2}}{{x}^{2}}+12{{a}^{4}}-10{{x}^{4}}+7a{{x}^{3}})\div (4{{a}^{2}}-3ax+2{{x}^{2}})$
\item $(8{{a}^{5}}-17{{a}^{3}}{{b}^{2}}-22{{a}^{4}}b-8{{b}^{5}}+48{{a}^{2}}{{b}^{3}}+26a{{b}^{4}})\div (2{{a}^{2}}-3ab-4{{b}^{2}})$
\item $(9{{x}^{8}}-130{{x}^{6}}+497{{x}^{4}}-520{{x}^{2}}+144)\div ({{x}^{3}}-2{{x}^{2}}-9x+18)$
\item $(3{{a}^{7}}-11{{a}^{6}}+7{{a}^{5}}+11{{a}^{4}}-2{{a}^{3}}+{{a}^{2}}-28a+15)\div (3{{a}^{3}}-2{{a}^{2}}-5a+3)$
\item $({{a}^{5}}-{{a}^{2}}{{b}^{3}})\div (a-b)$
\item $({{x}^{6}}-1)\div ({{x}^{3}}+2{{x}^{2}}+2x+1)$
\item $({{a}^{8}}+{{a}^{4}}+1)\div ({{a}^{2}}-a+1)$
\item $(6{{a}^{4m}}-{{a}^{3m}}-82{{a}^{2m}}+81{{a}^{m}}+36)\div (2{{a}^{m}}-3)$
\item $(4{{a}^{2m+4}}+6{{a}^{2m+3}}-{{a}^{2m+2}}+5{{a}^{2m+1}}-2{{a}^{2m}})\div ({{a}^{m+1}}+2{{a}^{m}})$
\end{enumerate}
\end{multicols}
\end{exercice}

\begin{exercice}
Effectuer les divisions algébriques suivantes (divisions avec reste) :
\begin{multicols}{2}
\begin{enumerate}
\item $(6{{x}^{5}}+5{{x}^{4}}-25{{x}^{3}}+31{{x}^{2}}-12x+5)\div (2{{x}^{2}}-3x+2)$
\item $(120{{x}^{4}}+154{{x}^{3}}+71{{x}^{2}}+14x+8)\div (6{{x}^{2}}+5x+1)$
\item $({{a}^{7}}-4{{a}^{6}}+2{{a}^{5}}+{{a}^{4}}-3{{a}^{2}}+2a-6)\div ({{a}^{5}}-3)$
\item $({{a}^{4}}-{{a}^{3}}b-{{a}^{2}}{{b}^{2}}-a{{b}^{3}}-{{b}^{4}})\div (a-2b)$
\item $({{a}^{8}}-2{{a}^{5}}+{{a}^{4}}+2{{a}^{3}}+1)\div ({{a}^{6}}+{{a}^{5}}-{{a}^{3}}+a+1)$
\item $(3{{x}^{6}}+27{{x}^{5}}-9{{x}^{4}}-68{{x}^{3}}+36{{x}^{2}}+10x+70)\div (3{{x}^{5}}-8{{x}^{3}}+4{{x}^{2}}+8)$
\item $(24{{a}^{5}}+16{{a}^{4}}b-12{{a}^{3}}{{b}^{2}}+4{{a}^{2}}{{b}^{3}}+10a{{b}^{4}}+4{{b}^{5}})\div (2{{a}^{2}}-2ab+{{b}^{2}})$
\item $(16{{x}^{8}}-32a{{x}^{6}}+20{{a}^{2}}{{x}^{4}}-10{{a}^{3}}{{x}^{2}}+2{{a}^{4}})\div (8{{x}^{6}}-12a{{x}^{4}}+6{{a}^{2}}{{x}^{2}}-{{a}^{3}})$
\end{enumerate}
\end{multicols}
\end{exercice}

\begin{exercice}
Effectuer les divisions algébriques suivantes (schéma de Hörner) :
\begin{multicols}{2}
\begin{enumerate}
\item $({{x}^{3}}-4{{x}^{2}}+4x-1)\div (x-1)$
\item $({{x}^{2}}-7x+6)\div (x-1)$
\item $(2{{a}^{3}}+7{{a}^{2}}-6a-5)\div (a+1)$
\item $({{x}^{4}}-7{{x}^{2}}-x+6)\div (x+3)$
\item $({{x}^{3}}-40x-63)\div (x-7)$
\item $(2{{x}^{6}}-3{{x}^{5}}-9{{x}^{4}}-5{{x}^{3}}+12{{x}^{2}}+4x-40)\div (x+2)$
\item $({{x}^{6}}-1)\div (x-1)$
\item $({{x}^{6}}-1)\div (x+1)$
\item $({{x}^{5}}+{{y}^{5}})\div (x+y)$
\item $({{x}^{8}}-{{y}^{8}})\div (x-y)$
\item $({{x}^{7}}+2187)\div (x+3)$
\item $({{x}^{8}}-{{y}^{8}})\div ({{x}^{2}}+{{y}^{2}})$
\end{enumerate}
\end{multicols}
\end{exercice}

\begin{exercice}
Factoriser les polynômes suivants (division par (x-a), schéma de Hörner) :
\begin{multicols}{2}
\begin{enumerate}
\item ${{x}^{3}}+9{{x}^{2}}+11x-21$
\item ${{x}^{3}}+2{{x}^{2}}-5x-6$
\item ${{x}^{4}}+2{{x}^{3}}-16{{x}^{2}}-2x+15$
\item ${{x}^{4}}-7{{x}^{3}}+17{{x}^{2}}-17x+6$
\item ${{x}^{5}}+3{{x}^{4}}-16x-48$
\item $6{{x}^{4}}+13{{x}^{3}}-13x-6$
\item $6{{x}^{4}}-5{{x}^{3}}-23{{x}^{2}}+20x-4$
\item $6{{x}^{4}}+4{{x}^{3}}-26{{x}^{2}}-16x+8$
\end{enumerate}
\end{multicols}
\end{exercice}

\section{Corrigé}

\begin{solution}
 Mettre en évidence les facteurs communs : 
\begin{enumerate}
\item $5ac(2c+5+3a)$
\item $6xy(2xy-3{{y}^{2}}+4{{x}^{2}})$
\item $6a{{x}^{2}}(2ax-5{{a}^{2}}+3{{x}^{2}})$
\item $(a+b+1)(x+y)$
\item $(a-b)(1+x-y)$
\item $22a{{x}^{n}}(2-13ax+3{{a}^{2}}{{x}^{2}})$
\item ${{x}^{n}}{{y}^{m}}({{x}^{m}}-{{x}^{n}}{{y}^{n}}-{{y}^{m}})$
\item $7{{x}^{m-3}}{{y}^{n-2}}({{x}^{6}}+2{{x}^{3}}{{y}^{3}}+3{{y}^{6}})$ 
\end{enumerate}
\end{solution}

\begin{solution}
Factoriser en utilisant les identités remarquables : 
\begin{multicols}{2}
\begin{enumerate}
\item $(a+3)(a-3)$
\item $(b+{{a}^{m}})(b-{{a}^{m}})$
\item $xy(x+y)(x-y)$
\item $(a+4b)(a-4b)$
\item $({{a}^{2}}+3b)({{a}^{2}}-3b)$
\item $(a+5x)(a-5x)$
\item $2(4a+{{b}^{2}})(4a-{{b}^{2}})$
\item ${{x}^{2}}(a+b)(a-b)$
\item $4(x+2a)(x-2a)$
\item $(abc+m)(abc-m)$
\item $2(5{{x}^{2}}+y)(5{{x}^{2}}-y)$
\item $64(2x+{{a}^{2}})(2x-{{a}^{2}})$
\item ${{x}^{2}}(a+9)(a-9)$
\item ${{y}^{2}}(4x+11y)(4x-11y)$
\item ${{x}^{2}}{{y}^{2}}(x+y)(x-y)$
\item $3ax(a+x)(a-x)$
\item $6{{a}^{2}}{{b}^{2}}(5{{a}^{2}}+2)(5{{a}^{2}}-2)$
\item $37{{a}^{3}}x(a+3)(a-3)$
\item $({{x}^{2}}+9)(x+3)(x-3)$
\item $(9{{x}^{2}}+25{{a}^{2}})(3x+5a)(3x-5a)$
\item $2(4{{x}^{2}}+{{a}^{2}})(2x+a)(2x-a)$
\item $3a({{x}^{2}}+{{y}^{2}})(x+y)(x-y)$
\item $3x({{x}^{2}}+4{{y}^{4}})(x+2{{y}^{2}})(x-2{{y}^{2}})$
\item ${{x}^{5}}{{y}^{4}}(x+y)({{x}^{2}}-xy+{{y}^{2}})(x-y)({{x}^{2}}+xy+{{y}^{2}})$
\item $(m+{{n}^{2}})({{m}^{2}}-m{{n}^{2}}+{{n}^{4}})$
\item $(3{{a}^{3}}{{b}^{2}}-2c)(9{{a}^{6}}{{b}^{4}}+6{{a}^{3}}{{b}^{2}}c+4{{c}^{2}})$
\item $6x(y-1)({{y}^{2}}+y+1)$
\item $(5x-1)(25{{x}^{2}}+5x+1)$
\item $ \begin{array}{ll}
  & 3(2xy+3z)(4{{x}^{2}}{{y}^{2}}-6xyz+9{{z}^{2}}) \\ 
 & (2xy-3z)(4{{x}^{2}}{{y}^{2}}+6xyz+9{{z}^{2}}) \\ 
\end{array} $
\item $ab(a+b)({{a}^{2}}-ab+{{b}^{2}})(a-b)({{a}^{2}}+ab+{{b}^{2}})$
\item $xy(x-y)({{x}^{2}}+xy+{{y}^{2}})({{x}^{6}}+{{x}^{3}}{{y}^{3}}+{{y}^{6}})$
\item ${{a}^{3m}}({{a}^{m}}+3{{b}^{n}})(a-3{{b}^{n}})$
\item $({{a}^{m+1}}+{{b}^{n+2}})({{a}^{2m+2}}-{{a}^{m+1}}{{b}^{n+2}}+{{b}^{n+4}})$
\item $(7x-8{{y}^{2}})(49{{x}^{2}}+56x{{y}^{2}}+64{{y}^{4}})$
\item $(2x+1)(4{{x}^{2}}-2x+1)(2x-1)(4{{x}^{2}}+2x+1)$
\item $(3a+2)(9{{a}^{2}}-6a+4)(3a-2)(9{{a}^{2}}+6a+4)$
\item $(a+b+c)(a+b-c)$
\item $(a+b+x-y)(a+b-x+y)$
\item $40ab$
\item $(4x-a)(3a-2x)$
\item $15(x+y)(x-y)$
\item $4a(b+c)$
\item $4x$
\item $2a({{a}^{2}}+3{{b}^{2}})$
\item $(x-5y)(7{{x}^{2}}-7xy+7{{y}^{2}})$ 
\end{enumerate}
\end{multicols}
\end{solution}

\begin{solution}
Factoriser en utilisant les identités remarquables : 
\begin{multicols}{2}
\begin{enumerate}
\item ${{(a+2b)}^{2}}$
\item ${{(3a-2b)}^{2}}$
\item ${{(2a-1)}^{2}}$
\item ${{\left( a-\frac{1}{2} \right)}^{2}}$
\item ${{\left( {{x}^{2}}+1 \right)}^{2}}$
\item ${{\left( {{x}^{3}}+3 \right)}^{2}}$
\item $a{{\left( b-c \right)}^{2}}$
\item ${{\left( \frac{x}{4}-3y \right)}^{2}}$
\item ${{\left( 2{{x}^{2}}+\frac{y}{4} \right)}^{2}}$
\item ${{\left( 3{{a}^{2}}b-c \right)}^{2}}$
\item ${{a}^{2}}b{{\left( \frac{3a}{2}-\frac{b}{3} \right)}^{2}}$
\item ${{a}^{2}}{{\left( 3{{a}^{m}}-\frac{{{y}^{n}}}{4} \right)}^{2}}$
\item $2{{a}^{2}}{{b}^{2}}{{c}^{2}}{{\left( 5{{a}^{2}}+6{{b}^{3}} \right)}^{2}}$
\item $\frac{4}{3}ax{{\left( {{a}^{3}}+3{{x}^{4}} \right)}^{2}}$
\item $36{{x}^{3}}y{{\left( 2x+3y \right)}^{2}}$
\item $6{{a}^{4}}b{{c}^{3}}{{\left( 2a-3b \right)}^{2}}$
\item $7{{a}^{2}}{{\left( 5{{x}^{m}}+4{{y}^{n}} \right)}^{2}}$
\item ${{\left( 7x{{y}^{4}}-5{{a}^{3}}{{b}^{2}} \right)}^{2}}$ 
\end{enumerate}
\end{multicols}
\end{solution}

\begin{solution}
Factoriser en utilisant les identités remarquables : 
\begin{multicols}{2}
\begin{enumerate}
\item ${{(a-2b)}^{3}}$
\item ${{\left( x+3 \right)}^{3}}$
\item ${{\left( {{a}^{2}}x-\frac{1}{a} \right)}^{3}}$
\item ${{\left( 2x-5y \right)}^{3}}$
\item ${{\left( 3{{a}^{2}}x-4b \right)}^{3}}$
\item $4{{\left( 3x+4 \right)}^{3}}$
\item ${{\left( 10a-4b \right)}^{3}}\text{ ou 8}{{\left( \text{5a-2b} \right)}^{3}}$
\item $2{{y}^{3}}{{\left( 5{{x}^{2}}{{y}^{2}}+{{z}^{2}} \right)}^{3}}$ 
\end{enumerate}
\end{multicols}
\end{solution}

\begin{solution}
Factoriser en utilisant les identités remarquables : 
\begin{multicols}{2}
\begin{enumerate}
\item ${{(a-b-c)}^{2}}$
\item ${{(x-y+1)}^{2}}$
\item ${{x}^{2}}{{({{x}^{3}}-x-1)}^{2}}$
\item ${{\left( {{x}^{2}}+x+1 \right)}^{2}}$ 
\end{enumerate}
\end{multicols}
\end{solution}

\begin{solution}
Factoriser les trinômes suivants, (coefficient de $x^2 = 1$) : 
\begin{multicols}{2}
\begin{enumerate}
\item  $(x-2)(x-6)$
\item $(x-1)(x-13)$
\item $(x-5)(x-17)$
\item $(x+1)(x-5)$
\item $(x+2)(x+8)$
\item $(x+15)(x+100)$
\item $(x+4)(x-8)$
\item $(x-2)(x+7)$
\item $(x+1)(x+19)$
\item $(x+2)(x-6)$
\item $(x-10'000)(x+1)$
\item $(x+12)(x-11)$ 
\end{enumerate}
\end{multicols}
\end{solution}

\begin{solution}
Factoriser les trinômes suivants, (coefficient de $x^2 \neq 2$ : 
\begin{multicols}{2}
\begin{enumerate}
\item $(x+1)(2x+7)$
\item $(2x+1)(x+2)$
\item $(9x-16)(x-1)$
\item $(4x+5)(x-1)$
\item $(3x-2)(x+1)$
\item $(4x+1)(x+2)$
\item $(5x-2)(x-5)$
\item $(2x+1)^2$
\item $(7x-1)(x-1)$
\item  $(7x+5)(x-2)$
\item  $(x-5)(2x-3)$
\item  $(5x+2)(x-1)$
\item $(5x+1)(x+1)$
\item $(4x+3)(x-2)$
\item $(7x+1)(x-5)$
\item $(3x-2)(x+1)$
\item $3(x+1)(3x-1)$
\item  $(4x-5)(x+1)$
\item  $(11x-5)(x+3)$
\item  $(2x^2+1)(3x^2+1)$
\item $(3x^2+1)(7x^2-5)$
\item $3(x-y)(15x+2y)$
\item $2(3x+y)(2x+5y)$
\item $(x+y)(x-y)(2x^2+3y^2$)
\end{enumerate}
\end{multicols}
\end{solution}

\begin{solution}
Effectuer les divisions algébriques (divisions sans reste) :
\begin{enumerate}
\item $(35{{x}^{3}}+47{{x}^{2}}+13x+1)\div (5x+1)=7{{x}^{2}}+8x+1$
\item $(6{{x}^{3}}-17{{x}^{2}}+14x-3)\div (2x-3)=3{{x}^{2}}-4x+1$
\item $({{a}^{7}}-3{{a}^{6}}+{{a}^{5}}-4{{a}^{2}}+12a-4)\div ({{a}^{5}}-4)={{a}^{2}}-3a+1$
\item $(10{{a}^{3}}{{b}^{2}}+{{a}^{5}}-5{{a}^{4}}b-10{{a}^{2}}{{b}^{3}}-{{b}^{5}}+5a{{b}^{4}})\div (a-b)={{a}^{4}}-4{{a}^{3}}b+6{{a}^{2}}{{b}^{2}}-4a{{b}^{3}}+{{b}^{4}}$
\item $(3{{a}^{4}}-7{{a}^{3}}-18{{a}^{2}}+28a+24)\div (3{{a}^{2}}+8a+4)={{a}^{2}}-5a+6$
\item $(14{{a}^{4}}-27{{a}^{3}}b-3a{{b}^{3}}+21{{a}^{2}}{{b}^{2}}-2{{b}^{4}})\div (2{{a}^{2}}-3ab+2{{b}^{2}})=7{{a}^{2}}-3ab-{{b}^{2}}$
\item $(-25{{a}^{3}}x-2{{a}^{2}}{{x}^{2}}+12{{a}^{4}}-10{{x}^{4}}+7a{{x}^{3}})\div (4{{a}^{2}}-3ax+2{{x}^{2}})=3{{a}^{2}}-4ax-5{{x}^{2}}$
\item $\begin{array}{ll}
  & (8{{a}^{5}}-17{{a}^{3}}{{b}^{2}}-22{{a}^{4}}b-8{{b}^{5}}+48{{a}^{2}}{{b}^{3}}+26a{{b}^{4}})\div (2{{a}^{2}}-3ab-4{{b}^{2}}) \\ 
 & =4{{a}^{3}}-5{{a}^{2}}b-8a{{b}^{2}}+2{{b}^{3}} \\ 
\end{array}$
\item $\begin{array}{ll}
  & (9{{x}^{8}}-130{{x}^{6}}+497{{x}^{4}}-520{{x}^{2}}+144)\div ({{x}^{3}}-2{{x}^{2}}-9x+18) \\ 
 & =9{{x}^{5}}+18{{x}^{4}}-13{{x}^{3}}-26{{x}^{2}}+4x+8 \\ 
\end{array}$
\item $\begin{array}{ll}
  & (3{{a}^{7}}-11{{a}^{6}}+7{{a}^{5}}+11{{a}^{4}}-2{{a}^{3}}+{{a}^{2}}-28a+15)\div (3{{a}^{3}}-2{{a}^{2}}-5a+3) \\ 
 & ={{a}^{4}}-3{{a}^{3}}+2{{a}^{2}}-a+5 \\ 
\end{array}$
\item $({{a}^{5}}-{{a}^{2}}{{b}^{3}})\div (a-b)={{a}^{4}}+{{a}^{3}}b+{{a}^{2}}{{b}^{2}}$		
\item $({{x}^{6}}-1)\div ({{x}^{3}}+2{{x}^{2}}+2x+1)={{x}^{3}}-2{{x}^{2}}+2x-1$	
\item $({{a}^{8}}+{{a}^{4}}+1)\div ({{a}^{2}}-a+1)={{a}^{6}}+{{a}^{5}}-{{a}^{3}}+a+1$		
\item $(6{{a}^{4m}}-{{a}^{3m}}-82{{a}^{2m}}+81{{a}^{m}}+36)\div (2{{a}^{m}}-3)=3{{a}^{3m}}+4{{a}^{2m}}-35{{a}^{m}}-12$	
\item $(4{{a}^{2m+4}}+6{{a}^{2m+3}}-{{a}^{2m+2}}+5{{a}^{2m+1}}-2{{a}^{2m}})\div ({{a}^{m+1}}+2{{a}^{m}})=4{{a}^{m+3}}-2{{a}^{m+2}}+3{{a}^{m+1}}-{{a}^{m}}$
\end{enumerate}
\end{solution}

\begin{solution}
Effectuer les divisions algébriques suivantes (divisions avec reste) :
\begin{multicols}{2}
\begin{enumerate}
\item $3{{x}^{3}}+7{{x}^{2}}-5x+1$	$x+3$
\item $20{{x}^{2}}+9x+1$
\item ${{a}^{2}}-4a+2$	${{a}^{4}}-10a$
\item ${{a}^{3}}+{{a}^{2}}b+a{{b}^{2}}+{{b}^{3}}$	${{b}^{4}}$	
\item ${{a}^{2}}-a+1$	 $-2{{a}^{5}}+2{{a}^{3}}$
\item $x+9$	$-{{x}^{4}}+2x-2$
\item $12{{a}^{3}}+20{{a}^{2}}b+8a{{b}^{2}}$	$2a{{b}^{4}}+4{{b}^{5}}$
\item $2{{x}^{2}}-a$	$-4{{a}^{2}}{{x}^{4}}-2{{a}^{3}}{{x}^{2}}+{{a}^{4}}$
\end{enumerate}
\end{multicols}
\end{solution}

\begin{solution}
Effectuer les divisions algébriques suivantes (schéma de Hörner) :
\begin{enumerate}
\item $({{x}^{3}}-4{{x}^{2}}+4x-1)\div (x-1)={{x}^{2}}-3x+1$
\item $({{x}^{2}}-7x+6)\div (x-1)=x-6$
\item $(2{{a}^{3}}+7{{a}^{2}}-6a-5)\div (a+1)=2{{a}^{2}}+5a-11\text{ reste 6}$
\item $({{x}^{4}}-7{{x}^{2}}-x+6)\div (x+3)={{x}^{3}}-3{{x}^{2}}+2x-7\text{ reste 27}$
\item $({{x}^{3}}-40x-63)\div (x-7)={{x}^{2}}+7x+9$
\item $\begin{array}{ll}
  & (2{{x}^{6}}-3{{x}^{5}}-9{{x}^{4}}-5{{x}^{3}}+12{{x}^{2}}+4x-40)\div (x+2) \\ 
 & =2{{x}^{5}}-7{{x}^{4}}+5{{x}^{3}}-15{{x}^{2}}+42x-80\text{ reste 120} \\ 
\end{array}$
\item $({{x}^{6}}-1)\div (x-1)={{x}^{5}}+{{x}^{4}}+{{x}^{3}}+{{x}^{2}}+x+1$
\item $({{x}^{6}}-1)\div (x+1)={{x}^{5}}-{{x}^{4}}+{{x}^{3}}-{{x}^{2}}+x-1$
\item $({{x}^{5}}+{{y}^{5}})\div (x+y)={{x}^{4}}-{{x}^{3}}y+{{x}^{2}}{{y}^{2}}-x{{y}^{3}}+{{y}^{4}}$
\item $({{x}^{8}}-{{y}^{8}})\div (x-y)={{x}^{7}}+{{x}^{6}}y+{{x}^{5}}{{y}^{2}}+{{x}^{4}}{{y}^{3}}+{{x}^{3}}{{y}^{4}}+{{x}^{2}}{{y}^{5}}+x{{y}^{6}}+{{y}^{7}}$
\item $({{x}^{7}}+2187)\div (x+3)={{x}^{6}}-3{{x}^{5}}+9{{x}^{4}}-27{{x}^{3}}+81{{x}^{2}}-243x+729$
\item $({{x}^{8}}-{{y}^{8}})\div ({{x}^{2}}+{{y}^{2}})={{x}^{6}}-{{x}^{4}}{{y}^{2}}+{{x}^{2}}{{y}^{4}}-{{y}^{6}}$
\end{enumerate}
\end{solution}

\begin{solution}
Factoriser les polynômes suivants (division par (x-a), schéma de Hörner) :
\begin{enumerate}
\item ${{x}^{3}}+9{{x}^{2}}+11x-21=(x-1)(x+3)(x+7)$
\item ${{x}^{3}}+2{{x}^{2}}-5x-6=(x+1)(x-2)(x+3)$
\item ${{x}^{4}}+2{{x}^{3}}-16{{x}^{2}}-2x+15=(x+1)(x-1)(x-3)(x+5)$
\item ${{x}^{4}}-7{{x}^{3}}+17{{x}^{2}}-17x+6={{(x-1)}^{2}}(x-2)(x-3)$
\item ${{x}^{5}}+3{{x}^{4}}-16x-48=(x-2)(x+2)(x+3)({{x}^{2}}+4)$
\item $6{{x}^{4}}+13{{x}^{3}}-13x-6=(x+1)(x-1)(2x+3)(3x+2)$
\item $6{{x}^{4}}-5{{x}^{3}}-23{{x}^{2}}+20x-4=(x-2)(x+2)(2x-1)(3x-1)$
\item $6{{x}^{4}}+4{{x}^{3}}-26{{x}^{2}}-16x+8=2(x+1)(x-2)(x+2)(3x-1)$
\end{enumerate}
\end{solution}