\subsection{Suites arithmétiques}
\begin{exercice}
Quel est le 110ème terme de la suite arithmétique suivante : 3; 5; 7; …
\end{exercice}

\begin{exercice}
Calculer le premier terme ${{t}_{1}}$et la raison r d'une suite arithmétique dont le quatrième terme est 40 et le douzième 52.
\end{exercice}

\begin{exercice}
Calculer le vingt-deuxième terme ${{t}_{22}}$ et la somme des vingt-deux premiers termes ${{S}_{22}}$ de la suite arithmétique suivante : 4; 6; 8; … 
\end{exercice}

\begin{exercice}
Calculer le premier terme ${{t}_{1}}$et la raison r d'une suite arithmétique sachant que la somme du huitième terme et du quatorzième terme est égale à 50 et que le cinquième terme est égal à 13.
\end{exercice}

\begin{exercice}
La somme des n premiers nombres entiers est 496. Trouver n.
\end{exercice}

\begin{exercice}
Dans une suite arithmétique on connaît : ${{t}_{1}}=8,\text{ }n=11,\text{ }{{S}_{11}}=253$ . Calculer r et ${{t}_{11}}$.
\end{exercice}

\begin{exercice}
Dans une suite arithmétique on connaît : ${{t}_{1}}=35,\text{ }n=11,\text{ }{{S}_{11}}=0$ . Calculer r et ${{t}_{11}}$.
\end{exercice}

\begin{exercice}
Dans une suite arithmétique on connaît : ${{t}_{3}}=11,\text{ }{{\text{t}}_{\text{11}}}=43,\text{ }n=13$ . Calculer r et ${{t}_{1}}$.
\end{exercice}

\begin{exercice}
Développer les dix premiers termes d'une suite arithmétique dont on connaît : ${{t}_{3}}=13\text{ et }{{t}_{7}}=29$.
\end{exercice}

\begin{exercice}
Calculer la somme des 50 premiers multiples de 11.
\end{exercice}

\begin{exercice}
Trouver trois nombres en suite arithmétique dont la somme est 105 et le produit 39'375.
\end{exercice}

\begin{exercice}
Trouver trois nombres en suite arithmétique dont la somme est 33 et le produit 1'287.
\end{exercice}

\begin{exercice}
Démontrer que les trois nombres suivants sont en suite arithmétique :  $\frac{a}{a+1};\text{  }\frac{2a+1}{a+1};\text{  }\frac{3a+2}{a+1}$
 \end{exercice}
 
\subsection{Suites géométriques}

\begin{exercice}
Dans une suite géométrique, on donne : ${{t}_{1}}=3,\text{ }q=2,\text{ }n=7$. Calculer ${{t}_{7}}\text{ }et\text{ }{{S}_{7}}$
\end{exercice}

\begin{exercice}
Dans une suite géométrique, on donne : ${{t}_{5}}=32,\text{ }q=-2,\text{ }n=5$. Calculer ${{t}_{1}}\text{ }et\text{ }{{S}_{5}}$
\end{exercice}

\begin{exercice}
Dans une suite géométrique, on donne : ${{t}_{2}}=2\text{ }et\text{ }{{t}_{6}}=8$. Calculer ${{t}_{1}}\text{ }et\text{ }q$
\end{exercice}

\begin{exercice}
Dans une suite géométrique, on donne : ${{t}_{3}}=0.01\text{ et }{{t}_{7}}=100$. Calculer ${{t}_{1}}\text{ }et\text{ }q$
\end{exercice}

\begin{exercice}
Dans une suite géométrique, on donne : ${{t}_{1}}=8,\text{ }{{t}_{3}}=18$. Calculer ${{t}_{13}}\text{ }et\text{ }{{\text{t}}_{\text{14}}}$
\end{exercice}

\begin{exercice}
Dans une suite géométrique, le premier terme est égal à 32 et le produit du troisième par le sixième est égal à 17'496. Calculer la raison de cette suite.
\end{exercice}

\begin{exercice}
Calculer la somme des termes de la suite géométrique illimitée suivante : ${{t}_{1}}=6,\text{ }q=\frac{1}{4}$
\end{exercice}

\subsection{Exercices divers sur les suites et séries}

\begin{exercice}
Une compagnie pétrolière britannique engage des ingénieurs pour la recherche de gisements pétroliers en Mer du Nord. Elle propose deux types de contrats salariaux aux ingénieurs engagés.
\end{exercice}

\begin{exercice}
Un premier salaire annuel de Fr. 110'000.— plus chaque année une augmentation de Fr. 5’500.— sur le salaire annuel de l’année précédente.
Un premier salaire annuel de Fr. 100'000.— plus chaque année une augmentation de 5,5% du salaire annuel de l’année précédente.

Indiquer quel est le contrat qui propose, pour une durée de 12 ans, 
\begin{enumerate}
\item Le meilleur dernier salaire annuel.
\item La meilleure rémunération totale (sur les 12 ans).
\end{enumerate}

Justifier vos deux réponses par l’utilisation de formules mathématiques.
\end{exercice}

\begin{exercice}
On a acheté une voiture pour le prix de Fr. 28'500.––. Quel sera le prix de cette voiture cinq années après l'achat, si en moyenne, elle a perdu chaque année le $15 \%$ de sa valeur.
\end{exercice}

\begin{exercice}
Après 14 années d’exploitation, le patron de l’entreprise «Florax» est un commerçant satisfait. Il constate que le bénéfice de son entreprise a augmenté chaque année en  suite géométrique de raison  $q=\tfrac{8}{7}$jusqu’à la fin de la huitième année d’exploitation; dès le début de la neuvième année, le bénéfice augmente en suite arithmétique de raison $r=$24'500.
	Si le bénéfice de la neuvième année s’est élevé à Fr. 85'616.—, déterminer : 
\begin{enumerate}
\item le bénéfice de la première année d’exploitation
\item le bénéfice total des 14 années réalisé par «Florax».
\end{enumerate}
\end{exercice}

\begin{exercice}
Une entreprise spécialisée dans la production en sous-traitance de pièces détachées pour l'industrie automobile reçoit une nouvelle commande.
	Une chaîne de production est alors mise en place qui permet à l'entreprise de réaliser la première semaine 12'000 pièces. Cette production hebdomadaire étant insuffisante pour satisfaire aux exigences des clients, l'entreprise doit améliorer le rendement de sa chaîne de production.
	La deuxième semaine, la production est alors augmentée de 1'250 pièces et, ainsi de suite, chaque semaine la production augmentent de 1'250 unités par rapport à la production de la semaine précédente.
\begin{enumerate}
\item Combien de semaines faudra-t-il à l'entreprise pour que sa production hebdomadaire atteigne 32'000 pièces ?
\item Combien de semaines de production seront-elles nécessaires à l'entreprise si elle doit produire au total 675'000 pièces ?
\end{enumerate}
\end{exercice}

\begin{exercice}
Le journal « PME Magazine » a sorti un graphique relatif à la vente des albums  « Titeuf ».
Par des méthodes mathématiques, nous trouvons que la vente des « Titeuf » a augmenté de manière géométrique au cours des années. 
La suite nous est donnée par la formule : \[{{t}_{n}}=92195\cdot {{(1.3718)}^{n-1}}\]
\begin{enumerate}
\item Calculer combien d’albums seront probablement vendus la 10e année de commercialisation si la suite des ventes se poursuit comme pour les années précédentes ?
\item A combien se monte le nombre total des albums vendus pendant ces 10 années ?
\item Après combien d’années aurons-nous atteint la vente cumulée de 10 millions d’albums annuels (à arrondir à l’année la plus proche) ?
\end{enumerate}
\end{exercice}