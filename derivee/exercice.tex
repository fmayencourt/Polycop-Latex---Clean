\begin{exercice}
Calculer la dérivée des fonctions suivantes :
\begin{multicols}{3}
\begin{enumerate} 
\item $x+1$
\item $2x$
\item $-3x+5$
\item ${{x}^{2}}$
\item $4{{x}^{2}}-5x+6$
\item $2{{x}^{3}}+2x+1$
\item $0$
\item ${{x}^{2}}-4$
\item $\frac{1}{2}{{x}^{2}}-3x+4$
\item $\left( x+5 \right)\left( x-3 \right)$
\item $\left( 3{{x}^{2}}+5 \right)\left( {{x}^{2}}-1 \right)$
\item $\frac{x+5}{x-1}$
\item $\frac{{{x}^{2}}-x+5}{{{x}^{2}}-2x+1}$
\item $\left( 2x-1 \right)\left( 7-3{{x}^{2}} \right)$
\item $\frac{2x}{{{x}^{2}}+1}$
\item $\sqrt{x}$
\item $\sqrt[3]{x}$
\item $\sqrt[3]{{{x}^{2}}}$
\item $-x+2$
\item ${{x}^{3}}-3x+2$
\item $3$
\item $–12x$
\item $3{{x}^{2}}$
\item $4{{x}^{2}}+x-6$
\item $\frac{1}{3}{{x}^{3}}-{{x}^{2}}+x-3$
\item $\frac{1}{4}{{x}^{4}}-\frac{5}{2}{{x}^{2}}+3x-\frac{3}{4}$
\item $–5$
\item ${{x}^{8}}-16$
\item ${{\left( x-1 \right)}^{2}}\left( x+2 \right)$
\item $\frac{{{x}^{3}}}{x+1}$
\item $\frac{2x}{{{\left( x-1 \right)}^{2}}}$
\item $\frac{{{x}^{3}}-4}{{{x}^{2}}}$
\item $\frac{{{x}^{2}}-2x-3}{2x+3}$
\end{enumerate}
\end{multicols}
\end{exercice} 

\begin{exercice}
\'Etudier les variations des fonctions suivantes par la méthode des dérivées. Pour chacune de ces fonctions, faire un croquis du graphique.
\begin{multicols}{2}
\begin{enumerate}
\item $f(x)=3{{x}^{2}}$
\item $f(x)={{x}^{2}}+3x$
\item $f(x)=x\left( 4-x \right)$
\item $f(x)={{x}^{2}}+2x+2$
\item $f(x)={{x}^{2}}-4x+4$
\item $f(x)=\left( x+1 \right)\left( 5-x \right)$
\item $f(x)=x\left( 9-{{x}^{2}} \right)$
\item $f(x)=\frac{1}{4}\left( {{x}^{3}}+x-10 \right)$
\item $f(x)={{x}^{3}}-3{{x}^{2}}+2$
\item $f(x)=3x-4{{x}^{3}}$
\item $f(x)=3{{x}^{2}}-{{x}^{3}}$
\item $f(x)={{x}^{4}}-10{{x}^{2}}+9$
\item $f(x)=\frac{x-1}{x+1}$
\item $f(x)=\frac{1-x}{2x-1}$
\item $f(x)=\frac{1}{{{x}^{2}}-3x+2}$
\item $f(x)=\frac{{{x}^{2}}-1}{4x+5}$
\item $f(x)=\frac{{{x}^{2}}+x-2}{{{x}^{2}}-1}$
\end{enumerate}
\end{multicols}
\end{exercice}  


\begin{exercice}
Indiquer sur quel sous-ensemble de $\mathbb{R}$ les fonctions ci-dessous sont strictement croissantes. 
\begin{multicols}{2}
\begin{enumerate}
\item $f(x)={{x}^{2}}+5x+1$
\item $f(x)={{x}^{3}}+3x$
\item $f(x)=\frac{{{x}^{3}}}{3}+\frac{5{{x}^{2}}}{2}+6x+1$
\item $f(x)=2{{x}^{4}}-9{{x}^{2}}$
\item $f(x)=\frac{4x+5}{2x-3}$
\item $f(x)={{x}^{3}}-3{{x}^{2}}+6x-1$
\item $f(x)={{x}^{5}}-5{{x}^{4}}+5{{x}^{3}}+1$ 
\end{enumerate}
\end{multicols}
\end{exercice}

\begin{exercice}
Pour les fonctions suivantes, donner l'équation de la tangente au graphe de $f$ 
en $x$. 
\begin{enumerate}
\item $f(x)=5{{x}^{2}}-6x+2\text{  ;  en }x=1$
\item $f(x)=\sqrt{x}\text{  ;  en }x=4$
\item $f(x)=\frac{3x-2}{5x+1}\text{  ;  en }x=0$
\end{enumerate}
\end{exercice}