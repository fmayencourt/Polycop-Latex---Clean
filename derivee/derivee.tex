\chapter{La fonction dérivée}

\section{Dérivée en $a$}

Soit $f(x)$ une fonction et $a$ un nombre réel. On cherche la pente de la fonction au point $\left(a;f(a)\right)$. Construisons ensemble cette notion :

Soit $h$ un nombre réel. On peut alors calculer avec la fonction le point $\left(a+h; f(a+h)\right)$. En reliant ces deux points on obtient une droite dont la pente est donnée par
$$
p = \frac{d_v}{d_h} = \frac{f(a+h)-f(a)}{(a+h)-a} = \frac{f(a+h)-f(a)}{h}
$$
Cette pente n'est pas exactement celle du point $\left(a;f(a)\right)$, mais en faisant tendre $h$ vers $0$ on obtient bien cette pente. On définit donc :

\begin{definition}
Soit $f(x)$ une fonction et $a$ un nombre réel. La \emph{dérivée de $f$ en $a$} est donnée par
$$
f'(a) =  \lim_{h\rightarrow 0}\frac{f(a+h)-f(a)}{h}
$$
\end{definition}

\begin{remarque}
La dérivée de $f$ en $a$ n'existe pas toujours. Par exemple la fonction $f(x) = \vaabs{x}$ n'admet pas de dérivée en $x=0$ car la pente depuis la gauche n'est pas la même que la pente depuis la droite.
\end{remarque}

\section{La fonction dérivée}

\begin{definition}
Soit $f$ une fonction dérivable en tout $a\in D_f$. On définit la fonction dérivée par 
$$
f'(x) =  \lim_{h\rightarrow 0}\frac{f(x+h)-f(x)}{h}
$$
\end{definition}

Dans le cadre de nos exercices, nous aurons systématiquement des fonctions dérivables.

\begin{proposition}
\begin{enumerate}
\item Si $f(x) = c$ une fonction constante, $f'(x) = 0$
\item Si $f(x) = x^n$, alors $f'(x) = n\cdot x^{n-1}$
\item Si $f(x) = \lambda \cdot g(x)$, alors $f'(x) = \lambda \cdot g'(x)$
\item Si $f(x) = g(x) + h(x)$, alors $f'(x) = g'(x) + h'(x)$
\item Si $f(x) = g(x) \cdot h(x)$, alors $f'(x) = g'(x) \cdot h(x) + g(x) \cdot h'(x)$
\item Si $f(x) = \frac{g(x)}{h(x)}$, alors $f'(x) = \frac{g'(x) \cdot h(x)- g(x) \cdot h'(x)}{h(x)^2}$
\item Si $f(x) = g\left(h(x)\right)$, alors $f'(x) = g'(h(x)) \cdot h'(x)$
\end{enumerate}
\end{proposition}

\begin{proof}
Nous ne ferons la démonstration que pour certains cas assez faciles. Par exemple :
\begin{enumerate}
\item $f(x) = c$, alors 
$$
f'(x) = \lim_{h\rightarrow 0} \frac{f(x+h) - f(x)}{h} = \lim_{x\rightarrow 0} \frac{c-c}{h} = \lim_{x\rightarrow 0} 0 = 0 
$$
\item $f(x) = x^2$, alors
$$
\begin{array}{l}
f'(x) = \lim_{x\rightarrow 0} \frac{f(x+h)-f(x)}{h} = \lim_{x\rightarrow 0} \frac{(x+h)^2 - x^2}{h} \\
\\ 
= \lim_{x\rightarrow 0} \frac{x^2 + 2xh + h^2 - x^2}{h} 
= \lim_{x\rightarrow 0} \frac{2xh+h^2}{h} = \lim_{x\rightarrow 0}2x + h = 2x\\
\end{array}
$$
\end{enumerate}
\end{proof}

\begin{exemple}
\begin{enumerate}
\item $f(x) = 3x^2 - 2x +3$, alors
$$
f'(x) = \left(3x^2\right)' - \left(2x\right)' + \left(3\right)'
= 3 \left(x^2\right)' -2\left(x\right)' + 0 = 3\cdot 2x^1 - 2\cdot 1x^0 = 6x-2
$$
\item $f(x) = (x-1)(2x+3)$, on commence par dériver chacune des parties :
$$
\begin{array}{l}
(x-1)' = (x)'-(1)' = 1-0 = 1\\
(2x+3)' = (2x)' + (3)' = 2(x)'+(3)' = 2\cdot 1 + 0 = 2\\
\end{array}
$$
On a donc :
$$
\begin{array}{l}
f'(x) = (x-1)'(2x+3)+(x-1)(2x+3)' = 1\cdot (2x+3)+(x-1)\cdot 2\\ \\ = (2x+3)+(2x-2) = 4x+1\\
\end{array}
$$
\item $f(x) = \frac{x-1}{2x+3}$, on commence par dériver chacune des parties :
$$
\begin{array}{l}
(x-1)' = (x)'-(1)' = 1-0 = 1\\
(2x+3)' = (2x)' + (3)' = 2(x)'+(3)' = 2\cdot 1 + 0 = 2\\
\end{array}
$$
On a donc :
$$
\begin{array}{l}
f'(x) = \frac{(x-1)'(2x+3)-(x-1)(2x+3)'}{(2x+3)^2} = \frac{1\cdot (2x+3)-(x-1)\cdot 2}{(2x+3)^2}\\
\\
 =\frac{2x+3-x+1}{(2x+3)^2}= \frac{x+4}{(2x+3)^2}\\
\end{array}
$$
\end{enumerate}
\end{exemple}

\section{Exercices}

\begin{exercice}
Calculer la dérivée des fonctions suivantes :
\begin{multicols}{3}
\begin{enumerate} 
\item $x+1$
\item $2x$
\item $-3x+5$
\item ${{x}^{2}}$
\item $4{{x}^{2}}-5x+6$
\item $2{{x}^{3}}+2x+1$
\item $0$
\item ${{x}^{2}}-4$
\item $\frac{1}{2}{{x}^{2}}-3x+4$
\item $\left( x+5 \right)\left( x-3 \right)$
\item $\left( 3{{x}^{2}}+5 \right)\left( {{x}^{2}}-1 \right)$
\item $\frac{x+5}{x-1}$
\item $\frac{{{x}^{2}}-x+5}{{{x}^{2}}-2x+1}$
\item $\left( 2x-1 \right)\left( 7-3{{x}^{2}} \right)$
\item $\frac{2x}{{{x}^{2}}+1}$
\item $\sqrt{x}$
\item $\sqrt[3]{x}$
\item $\sqrt[3]{{{x}^{2}}}$
\item $-x+2$
\item ${{x}^{3}}-3x+2$
\item $3$
\item $–12x$
\item $3{{x}^{2}}$
\item $4{{x}^{2}}+x-6$
\item $\frac{1}{3}{{x}^{3}}-{{x}^{2}}+x-3$
\item $\frac{1}{4}{{x}^{4}}-\frac{5}{2}{{x}^{2}}+3x-\frac{3}{4}$
\item $–5$
\item ${{x}^{8}}-16$
\item ${{\left( x-1 \right)}^{2}}\left( x+2 \right)$
\item $\frac{{{x}^{3}}}{x+1}$
\item $\frac{2x}{{{\left( x-1 \right)}^{2}}}$
\item $\frac{{{x}^{3}}-4}{{{x}^{2}}}$
\item $\frac{{{x}^{2}}-2x-3}{2x+3}$
\end{enumerate}
\end{multicols}
\end{exercice} 

\begin{exercice}
\'Etudier les variations des fonctions suivantes par la méthode des dérivées. Pour chacune de ces fonctions, faire un croquis du graphique.
\begin{multicols}{2}
\begin{enumerate}
\item $f(x)=3{{x}^{2}}$
\item $f(x)={{x}^{2}}+3x$
\item $f(x)=x\left( 4-x \right)$
\item $f(x)={{x}^{2}}+2x+2$
\item $f(x)={{x}^{2}}-4x+4$
\item $f(x)=\left( x+1 \right)\left( 5-x \right)$
\item $f(x)=x\left( 9-{{x}^{2}} \right)$
\item $f(x)=\frac{1}{4}\left( {{x}^{3}}+x-10 \right)$
\item $f(x)={{x}^{3}}-3{{x}^{2}}+2$
\item $f(x)=3x-4{{x}^{3}}$
\item $f(x)=3{{x}^{2}}-{{x}^{3}}$
\item $f(x)={{x}^{4}}-10{{x}^{2}}+9$
\item $f(x)=\frac{x-1}{x+1}$
\item $f(x)=\frac{1-x}{2x-1}$
\item $f(x)=\frac{1}{{{x}^{2}}-3x+2}$
\item $f(x)=\frac{{{x}^{2}}-1}{4x+5}$
\item $f(x)=\frac{{{x}^{2}}+x-2}{{{x}^{2}}-1}$
\end{enumerate}
\end{multicols}
\end{exercice}  


\begin{exercice}
Indiquer sur quel sous-ensemble de $\mathbb{R}$ les fonctions ci-dessous sont strictement croissantes. 
\begin{multicols}{2}
\begin{enumerate}
\item $f(x)={{x}^{2}}+5x+1$
\item $f(x)={{x}^{3}}+3x$
\item $f(x)=\frac{{{x}^{3}}}{3}+\frac{5{{x}^{2}}}{2}+6x+1$
\item $f(x)=2{{x}^{4}}-9{{x}^{2}}$
\item $f(x)=\frac{4x+5}{2x-3}$
\item $f(x)={{x}^{3}}-3{{x}^{2}}+6x-1$
\item $f(x)={{x}^{5}}-5{{x}^{4}}+5{{x}^{3}}+1$ 
\end{enumerate}
\end{multicols}
\end{exercice}

\begin{exercice}
Pour les fonctions suivantes, donner l'équation de la tangente au graphe de $f$ 
en $x$. 
\begin{enumerate}
\item $f(x)=5{{x}^{2}}-6x+2\text{  ;  en }x=1$
\item $f(x)=\sqrt{x}\text{  ;  en }x=4$
\item $f(x)=\frac{3x-2}{5x+1}\text{  ;  en }x=0$
\end{enumerate}
\end{exercice}