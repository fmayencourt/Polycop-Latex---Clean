\chapter{Racines}

\section{Racines carrées}

\subsection{L'histoire}

L'extraction d'une racine carrée a de tout temps joué un grand rôle en mathématique (du moins jusqu'à l'apparition de machines à calculer). En effet, les mathématiques ont pris une place vraiment importante dans la société dès que l'on a chercher à s'attribuer des biens: il faut savoir compter le nombre d'unités de blé qu'il me reste, sachant combien j'en avais après la récolte et sachant combien j'en ai vendu; il faut savoir combien d'unités d'aire de champs j'ai pour faire un partage (équitable ou non) de mon héritage, etc. Ces exemples mettent en évidence deux aspects des mathématiques : l'algèbre et la géométrie. Comme le disait Descartes \cite{Descartes}:
\begin{quotation}
(. . . ) toute l’arithmétique n’est composée que de quatre ou cinq opérations, qui
sont : l’addition, la soustraction, la multiplication, la division et l’extraction des
racines, qu’on peut prendre pour une espèce de division (. . . )
\end{quotation}
Si la racine carrée joue un rôle mineur dans l'algèbre de base, il en est tout autrement dans la géométrie, où la notion d'aire lui est intimement lié.

L'origine de la notation $\sqrt{a}$ est flou, mais quelques auteurs \cite{Euler} pensent qu'elle provient de 
$$
Ra \rightarrow ra \rightarrow \sqrt{a}.
$$

\subsection{La théorie}

On définit la racine carrée de la manière suivante :

\begin{definition}
Soit $a$ un nombre réel positif. Le nombre réel positif $x$ est appelé \emph{racine carrée de $a$} et noté $\sqrt{a}$ s'il satisfait la relation suivante :
$$
x = \sqrt{a} \ssi x^2 = a.
$$
\end{definition}

Interprétons de manière géométrique cette définition: 
\begin{quotation}
Soit un champ de forme carrée et d'aire $a$. Quelle longueur à chacun de ses côtés ?\\
Posons $l$ la longueur d'un des côtés. Par géométrie, on sait que l'aire du carré est donné par la formule \emph{aire = côté $\cdot$ côté}. On obtient donc
$$
l^2 = a,
$$
d'où, par la définition que nous venons de poser, $l = \sqrt{a}$.
\end{quotation}

La racine carrée possède des propriétés directement héritées de celles des puissances carrées:

\begin{propriete}
Soient $a$ et $b$ deux nombres réels positifs et $c$ un nombre réel quelconque. On a alors:
\begin{indice}
	\item $\sqrt{a \cdot b} = \sqrt{a} \cdot \sqrt{b}$; \label{racinei}
	\item $\sqrt{\frac{a}{b}} = \frac{\sqrt{a}}{\sqrt{b}}$;\label{racineii}
\end{indice}
\end{propriete}

\begin{proof}
Soient $a$ et $b$ deux réels positifs. Montrons la propriété \ref{racinei} : posons $x = \sqrt{a}$ et $y = \sqrt{b}$. Par la définition, on a donc
$$
\left.
\begin{array}{lcr}
x^2 &=& a\\
y^2 &=& b\\
\end{array}
\right\} \Rightarrow x^2 y^2 = ab.
$$
Par la distributivité du carré pour la multiplication, on a donc $(xy)^2 = ab$, d'où, par la définition:
$$
\sqrt{a \cdot b} = \sqrt{a} \cdot \sqrt{b}.
$$
Toujours sous les mêmes conditions, montrons la propriété \ref{racineii}. Par la définition, on a:
$$
\left.
\begin{array}{lcr}
x^2 &=& a\\
y^2 &=& b\\
\end{array}
\right\} \Rightarrow \frac{x^2}{y^2} = \frac{a}{b}.
$$
Par la distributivité du carré pour la multiplication, on a donc $(\frac{x}{y})^2 = \frac{a}{b}$, d'où, par la définition:
$$
\sqrt{\frac{a}{b}} = \frac{\sqrt{a}}{\sqrt{b}},
$$
ce qui achève la démonstration.
\end{proof}

\begin{remarque}
\textbf{Attention !} En général, on a pour $x$ et $y$ deux nombres réels positifs et $z$ un nombre réel quelconque:
\begin{itemize}
	\item $\sqrt{x^2 + y^2} \neq x + y$;
	\item $\sqrt{x+y} \neq \sqrt{x} + \sqrt{y}$;
\end{itemize}
\end{remarque}

Ces deux premières formules découlent principalement de l'identité remarquable :
$$
(x+y)^2 = x^2 + 2xy + y^2.
$$


Introduisons maintenant une notion importante dans le calcul de racines:

\begin{definition}
On appelle \emph{carré parfait} les entiers $n$ dont la racine carrée est elle-même entière.
\end{definition}

\begin{enumerate}
\item Suite à cette définition, la notion d'\emph{extraction de carrés parfaits} prend tout son sens: il s'agit de 'sortir' de la racines $\sqrt{n}$ ses diviseurs qui sont des carrés parfaits. Par exemple:
$$
\sqrt{175} = \sqrt{25 \cdot 7} = 5\sqrt{7}.
$$
Cette notation simplifie le calcul 'manuel' de racines : il est beaucoup plus facile d'approcher numériquement $\sqrt{7}$ plutôt que $\sqrt{175}$.
\item Une autre simplification se situe au niveau des fractions : la fraction $\frac{\sqrt{3}}{\sqrt{5}}$ est très difficile à cerner, car il faut diviser un nombre approché par un autre nombre approché; sans compter que la division par un nombre non-entier est très difficile. La méthode consiste donc à amplifier la fraction par le dénominateur de façon à le rendre entier. Par exemple :
$$
\frac{\sqrt{3}}{\sqrt{5}} = \frac{\sqrt{3}}{\sqrt{5}} \cdot \frac{\sqrt{5}}{\sqrt{5}} = \frac{\sqrt{3 \cdot 5}}{(\sqrt{5})^2} = \frac{\sqrt{15}}{5} \simeq \frac{4}{5}.
$$
\end{enumerate}


\section{Racines $n$-èmes}

On peut étendre la notion de racine carrée à une racine $n$-ème selon la définition suivante :

\begin{definition}
Soient $a$ un nombre réel positif et $n$ un nombre entier positif. Le nombre réel $x$ (positif si $n$ est pair) est appelé \emph{racine n-ème de $a$} et noté $\sqrt[n]{a}$ s'il satisfait la relation suivante :
$$
x = \sqrt[n]{a} \ssi x^n = a.
$$
\end{definition}

On applique les mêmes règles pour les extractions de racines $n$-èmes.

\begin{theoreme}
La notion de racine $n$-ème (et donc aussi de racine carrée) peut s'entendre sous la forme de puissance selon la formule :
$$
\sqrt[n]{a} = a^{\frac{1}{n}}.
$$ 
\end{theoreme}

\begin{proof}
Notons $x$ la puissance de $a$ telle que 
$$
\sqrt[n]{a} = a^x.
$$
Alors selon la définition de la racine :
$$
\begin{array}{ll}
&\left(\sqrt[n]{a}\right)^n = a \\
\ssi & \left(a^x\right)^n = a\\
\ssi & a^{n\cdot x} = a^1
\end{array}
$$
Ainsi $n\cdot x = 1$ et donc $x=\frac{1}{n}$. On a donc démontré
$$
\sqrt[n]{a} = a^\frac{1}{n}.
$$
\end{proof}

\section{Exercice}

\subsection{Racines carrées}
\begin{exercice}
Simplifier les racines suivantes :
\begin{enumerate}
\item $\sqrt{12},\text{  }\sqrt{18},\text{  }\sqrt{32},\text{  }\sqrt{160},\text{  }\sqrt{300},\text{  }\sqrt{2000}$
\item $\sqrt{{{a}^{5}}},\text{  }\sqrt{{{a}^{4}}{{x}^{5}}},\text{  }\sqrt{8{{x}^{4}}},\text{  }\sqrt{72{{x}^{4m}}},\text{  }\sqrt{12a{{b}^{5}}},\text{  }\sqrt{18{{x}^{4n}}{{y}^{4n+1}}}$
\item $\sqrt{162{{a}^{2}}},\text{  }\sqrt{90{{x}^{2}}},\text{  }\sqrt{{{x}^{4}}{{y}^{3}}},\text{  }\sqrt{2{{a}^{3}}{{x}^{2}}},\text{  }\sqrt{20{{a}^{5}}{{x}^{2n}}},\text{  }\sqrt{504{{x}^{2n-1}}}$
\item $\sqrt{\frac{1}{9}},\text{  }\sqrt{\frac{5}{3}},\text{  }\sqrt{\frac{9}{8}},\text{  }\sqrt{\frac{7}{27}},\text{  }\sqrt{\frac{1}{2}},\text{  }\sqrt{\frac{5}{7}}$
\item $\sqrt{\frac{a{{x}^{5}}}{20}},\text{  }\sqrt{\frac{320{{a}^{5}}}{{{x}^{3}}}},\text{  }\sqrt{\frac{12{{a}^{9}}}{5{{b}^{8}}}},\text{  }\sqrt{\frac{{{a}^{2}}x}{3}},\text{ }\sqrt{\frac{2{{a}^{3}}}{{{b}^{4}}}},\text{  }\sqrt{\frac{{{a}^{4}}{{x}^{2n}}}{2}}$
\end{enumerate}
\end{exercice}

\begin{exercice}
Effectuer les opérations suivantes :
\begin{multicols}{2}
\begin{enumerate}
\item $5\sqrt{2}-\frac{1}{2}\sqrt{2}+\frac{2}{3}\sqrt{2}-2\sqrt{2}$ 
\item $\sqrt{50}-2\sqrt{8}+3\sqrt{18}-7\sqrt{2}$ 
\item $2\sqrt{54}-2\sqrt{24}-\sqrt{150}+\sqrt{6}$ 
\item $2\sqrt{\frac{1}{2}}-\sqrt{18}+\sqrt{\frac{2}{9}}-\sqrt{\frac{9}{8}}$ 
\item $\sqrt{48}-\sqrt{\frac{12}{25}}+\sqrt{\frac{1}{3}}+3\sqrt{75}$
\item $2\sqrt{28}-6\sqrt{\frac{7}{4}}+14\sqrt{\frac{1}{7}}$
\item $\sqrt{72}+3-\sqrt{50}-\sqrt{25}$
\item $5\sqrt{12}-2\sqrt{\frac{3}{4}}+2\sqrt{27}-8\sqrt{\frac{3}{16}}$
\item $-\sqrt{\frac{3}{5}}+2\sqrt{\frac{5}{3}}-\sqrt{60}+\sqrt{\frac{1}{15}}$
\item $2\sqrt{5}-\frac{1}{2}\sqrt{\frac{1}{5}}+\frac{1}{20}\sqrt{45}$
\end{enumerate}
\end{multicols}
\end{exercice}

\begin{exercice}
Effectuer les opérations suivantes :
\begin{multicols}{2}
\begin{enumerate}
\item $\sqrt{2{{a}^{3}}b}-\sqrt{32a{{b}^{3}}}+\sqrt{18a{{b}^{3}}}$
\item $\sqrt{72{{a}^{5}}{{b}^{5}}}+\sqrt{18{{a}^{3}}{{b}^{7}}}+\sqrt{18{{a}^{7}}{{b}^{3}}}$
\item $\sqrt{{{a}^{5}}-2{{a}^{4}}}+\sqrt{a{{b}^{4}}-2{{b}^{4}}}-\sqrt{4{{a}^{3}}{{b}^{2}}-8{{a}^{2}}{{b}^{2}}}$
\item $\sqrt{\frac{{{a}^{2}}b}{c}}-\sqrt{\frac{9{{b}^{3}}}{c}}-\sqrt{bc}+\sqrt{\frac{4{{b}^{3}}}{c}}$
\item $\sqrt{\frac{a+b}{a-b}}-\sqrt{\frac{a-b}{a+b}}$
\item $a\sqrt{\frac{a-b}{a+b}}+b\sqrt{\frac{a+b}{a-b}}$
\item $\sqrt{\frac{b+x}{{{x}^{2}}}}-\sqrt{\frac{b}{{{x}^{2}}}+\frac{1}{x}}$
\item $\sqrt{\frac{{{a}^{3}}-{{a}^{2}}b}{{{b}^{2}}}}-\sqrt{\frac{a{{b}^{2}}-{{b}^{3}}}{{{a}^{2}}}}$
\end{enumerate}
\end{multicols}
\end{exercice}

\begin{exercice}
Effectuer les opérations suivantes :
\begin{multicols}{2}
\begin{enumerate}
\item $\sqrt{28}\cdot \sqrt{7}$ 
\item $\sqrt{10}\cdot \sqrt{15}$ 
\item $-\sqrt{7}\cdot \sqrt{42}$ 
\item $\sqrt{7}\cdot \sqrt{\frac{1}{7}}$ 
\item $2\sqrt{18}\cdot \sqrt{8}$
\item $\left( 4-\sqrt{3} \right)\cdot \sqrt{3}$
\item $\left( \sqrt{5}-\sqrt{3} \right)\cdot \sqrt{15}$
\item $\left( \sqrt{3}-\sqrt{2} \right)\left( -\sqrt{6} \right)$
\item $\left( 3+\sqrt{5} \right)\left( 2-\sqrt{5} \right)$
\item $\left( 7+2\sqrt{6} \right)\left( 9-5\sqrt{6} \right)$
\item $\left( 9\sqrt{12}+3 \right)\left( \sqrt{3}+8 \right)$
\item $\left( 6+12\sqrt{7} \right)\left( 3-5\sqrt{7} \right)$
\end{enumerate}
\end{multicols}
\end{exercice}

\begin{exercice}
Effectuer les opérations suivantes, en utilisant exclusivement les propriétés du produit de deux binômes  conjugués :
\begin{multicols}{2}
\begin{enumerate}
\item $\left( 9+2\sqrt{10} \right)\left( 9-2\sqrt{10} \right)$ 
\item $\left( -5-\sqrt{3} \right)\left( -5+\sqrt{3} \right)$ 
\item $\left( 5-2\sqrt{3} \right)\left( 5+2\sqrt{3} \right)$ 
\item $\left( \sqrt{x}+\sqrt{y} \right)\left( \sqrt{{{x}^{2}}}-\sqrt{xy} \right)$ 
\item $\left( a+b+\sqrt{x} \right)\left( a+b-\sqrt{x} \right)$
\item $\left( x-y+\sqrt{xy} \right)\left( x-y-\sqrt{xy} \right)$
\item $\left( 3\sqrt{2}+2\sqrt{3} \right){{\left( 3\sqrt{2}-2\sqrt{3} \right)}^{2}}$
\item $\left( \sqrt{a}-\sqrt{b}+1 \right)\left( \sqrt{a}+\sqrt{b}-1 \right)$
\end{enumerate}
\end{multicols}
\end{exercice}

\begin{exercice}
Effectuer les opérations suivantes :
\begin{enumerate}
\item $\left( 2\sqrt{8}+3\sqrt{5}-7\sqrt{2} \right)\left( \sqrt{72}-5\sqrt{20}-2\sqrt{2} \right)$ 
\item $\left( \sqrt{35}+3\sqrt{2}+\sqrt{7} \right)\left( \sqrt{35}-3\sqrt{2}-\sqrt{7} \right)$ 
\item $\sqrt{5+\sqrt{24}}\cdot \sqrt{5-\sqrt{24}}$
\item $\sqrt{3-2\sqrt{2}}\cdot \sqrt{3+2\sqrt{2}}$
\end{enumerate}
\end{exercice}

\begin{exercice}
Effectuer les opérations suivantes :
\begin{multicols}{3}
\begin{enumerate}
\item ${{\left( 3\sqrt{2} \right)}^{2}}$ 
\item ${{\left( 2\sqrt{7} \right)}^{2}}$
\item ${{\left( 3a\sqrt{b} \right)}^{2}}$
\item ${{\left( 2a\sqrt{5b} \right)}^{2}}$
\item ${{\left( -x\sqrt{{{x}^{n}}} \right)}^{2}}$
\item ${{\left( \sqrt{2}-\sqrt{6} \right)}^{2}}$
\item ${{\left( 2\sqrt{6}+\sqrt{24} \right)}^{2}}$
\item ${{\left( 2-\sqrt{5} \right)}^{2}}$
\item ${{\left( \sqrt{2}+\sqrt{3}-\sqrt{6} \right)}^{2}}$
\end{enumerate}
\end{multicols}
\end{exercice}

\begin{exercice}
Rendre rationnel le dénominateur des fractions suivantes :
\begin{multicols}{3}
\begin{enumerate}
\item $\frac{9\sqrt{2}-8\sqrt{3}+3\sqrt{6}}{\sqrt{6}}$ 
\item $\frac{\sqrt{32}-5\sqrt{2}+2\sqrt{8}}{2\sqrt{8}}$ 
\item $\frac{-\sqrt{12}+\sqrt{24}-\sqrt{48}}{-\sqrt{6}}$ 
\item $\frac{3-\sqrt{3}}{1+\sqrt{3}}$
\item $\frac{\sqrt{2}}{\sqrt{2}-\sqrt{5}}$
\item $\frac{\sqrt{6}}{\sqrt{3}+\sqrt{2}}$
\item $\frac{\sqrt{5}-\sqrt{3}}{\sqrt{5}+\sqrt{3}}$
\item $\frac{5\sqrt{3}-3\sqrt{5}}{\sqrt{5}-\sqrt{3}}$
\item $\frac{7\sqrt{5}+5\sqrt{7}}{\sqrt{7}+\sqrt{5}}$
\end{enumerate}
\end{multicols}
\end{exercice}

\subsection{Racines $n$èmes}

\begin{exercice}
Simplifier les racines suivantes :
\begin{multicols}{3}
\begin{enumerate}
\item $\sqrt[6]{{{a}^{4}}{{b}^{2}}}$ 
\item $\sqrt[4]{64{{a}^{8}}{{b}^{6}}}$ 
\item $\sqrt{\sqrt[3]{4{{a}^{2}}}}$ 
\item $\sqrt[5]{\sqrt[3]{-{{a}^{5}}}}$ 
\item $\sqrt[5]{\sqrt[3]{{{x}^{20}}}}$
\item $\sqrt[3]{2a\sqrt{2a}}$
\item ${{\left( \sqrt[12]{a{{b}^{3}}{{c}^{7}}} \right)}^{4}}$
\item ${{\left( \sqrt[7]{-a\sqrt{3a}} \right)}^{14}}$
\item $\frac{{{a}^{2}}}{b}\sqrt[4]{{{b}^{5}}x}$
\item ${{\left( \sqrt{a\sqrt[5]{{{a}^{8}}}} \right)}^{4}}$
\item ${{\left( \sqrt[3]{\sqrt[7]{-8{{a}^{3}}}} \right)}^{7}}$
\item $\sqrt[3]{12096}$
\end{enumerate}
\end{multicols}
\end{exercice}

\begin{exercice}
Effectuer les opérations suivantes :
\begin{multicols}{2}
\begin{enumerate}
\item $\sqrt[3]{56}+\sqrt[3]{189}+\sqrt[3]{448}$
\item $\sqrt[3]{54}-\sqrt[3]{16}+\sqrt{12}-\sqrt{3}$
\item $4\sqrt[3]{24}+\sqrt[3]{-3}+\sqrt[3]{-81}$ 
\item $\sqrt[6]{16}-\sqrt[4]{4}+\sqrt[3]{-4}$
\item $\sqrt{50}-\sqrt[4]{324}-\sqrt[6]{2916}+\sqrt[8]{256}$
\item $9\sqrt[3]{2{{a}^{6}}x}+3\sqrt[3]{-16{{a}^{3}}x}+\sqrt[3]{2x}$
\end{enumerate}
\end{multicols}
\end{exercice}

\begin{exercice}
Effectuer les opérations suivantes :
\begin{multicols}{2}
\begin{enumerate}
\item $5\sqrt[3]{18}\cdot \sqrt[3]{6}$
\item $\sqrt[4]{20}\cdot \sqrt{2}$ 
\item $3\sqrt[4]{a}\cdot 7\sqrt[6]{{{a}^{5}}b}$
\item $\sqrt{a}\cdot \sqrt[3]{-a}\cdot \sqrt[4]{a}$
\item $\sqrt[3]{3+2\sqrt{2}}\cdot \sqrt[3]{3-2\sqrt{2}}$
\item $\left( a\sqrt[3]{a}-\sqrt[3]{{{a}^{2}}}+1 \right)\left( \sqrt[3]{{{a}^{2}}}+1 \right)$
\end{enumerate}
\end{multicols}
\end{exercice}

\begin{exercice}
Rendre rationnel le dénominateur des fractions suivantes :
\begin{multicols}{3}
\begin{enumerate}
\item $\frac{\sqrt[3]{12}}{2\sqrt{2}}$
\item $\frac{\sqrt[4]{18}}{\sqrt[4]{8}}$
\item $\frac{4\,\sqrt[3]{-12}}{2\sqrt{2}}$ 
\item $\frac{\sqrt{2}}{\sqrt[3]{4}}$
\item $\frac{\sqrt[5]{27}}{\sqrt[4]{9}}$
\item $\frac{\sqrt{3}}{\sqrt[3]{-3}}$
\item $5\,\sqrt[3]{\frac{8}{75}}$
\item $4\,\sqrt[3]{\frac{3}{80}}$
\item $\sqrt[3]{\frac{3}{\sqrt[4]{3}}}$
\end{enumerate}
\end{multicols}
\end{exercice}

\section{Corrigés}

\subsection{Racines carrées}
\begin{solution}
Simplifier les racines suivantes :
\begin{enumerate}
\item $\left| \begin{array}{ll}
  & \sqrt{12}=\sqrt{{{2}^{2}}\cdot 3}=2\sqrt{3} \\ 
 & \sqrt{18}=\sqrt{{{3}^{2}}\cdot 2}=3\sqrt{2} \\ 
 & \sqrt{32}\text{=}\sqrt{{{\text{2}}^{\text{4}}}\cdot 2}\text{=4}\sqrt{\text{2}}\text{ } \\ 
 & \sqrt{160}=\sqrt{{{2}^{4}}\cdot 10}=4\sqrt{10} \\ 
 & \sqrt{300}=\sqrt{{{2}^{2}}\cdot {{5}^{2}}\cdot 3}=10\sqrt{3} \\ 
 & \sqrt{2000}=\sqrt{{{2}^{4}}\cdot 5{}^{2}\cdot 5}=20\sqrt{5} \\ 
\end{array} \right.$
\item $\left| \begin{array}{ll}
  & \sqrt{{{a}^{5}}}=\sqrt{{{a}^{4}}\cdot a}={{a}^{2}}\sqrt{a} \\ 
 & \sqrt{{{a}^{4}}{{x}^{5}}}=\sqrt{{{a}^{4}}\cdot {{x}^{4}}\cdot x}={{a}^{2}}{{x}^{2}}\sqrt{x} \\ 
 & \sqrt{8{{x}^{4}}}=\sqrt{{{2}^{2}}\cdot 2\cdot {{x}^{4}}}=2{{x}^{2}}\sqrt{2} \\ 
 & \sqrt{72{{x}^{4m}}}=\sqrt{{{6}^{2}}\cdot 2{{x}^{4m}}}=6{{x}^{2m}}\sqrt{2} \\ 
 & \sqrt{12a{{b}^{5}}}=\sqrt{{{2}^{2}}\cdot 3a{{b}^{4}}\cdot b}\text{=2}{{\text{b}}^{\text{2}}}\sqrt{\text{3ab}}\text{ } \\ 
 & \sqrt{18{{x}^{4n}}{{y}^{4n+1}}}=\sqrt{{{3}^{2}}\cdot 2x{}^{4n}\cdot {{y}^{4n}}\cdot y}=3{{x}^{2n}}{{y}^{2n}}\sqrt{2y} \\ 
\end{array} \right.$
\item $\left| \begin{array}{ll}
  & \sqrt{162{{a}^{2}}}=\sqrt{81\cdot 2{{a}^{2}}}=9a\sqrt{2} \\ 
 & \sqrt{90{{x}^{2}}}=\sqrt{9\cdot 10{{x}^{2}}}=3x\sqrt{10} \\ 
 & \sqrt{{{x}^{4}}{{y}^{3}}}=\sqrt{{{x}^{4}}\cdot {{y}^{2}}\cdot y}={{x}^{2}}y\sqrt{y} \\ 
 & \sqrt{2{{a}^{3}}{{x}^{2}}}=\sqrt{2{{a}^{2}}\cdot a{{x}^{2}}}=ax\sqrt{2a} \\ 
 & \sqrt{20{{a}^{5}}{{x}^{2n}}}\text{=}\sqrt{\text{4}\cdot \text{5}\cdot {{\text{a}}^{\text{4}}}\cdot a\cdot {{x}^{2n}}}\text{=2}{{\text{a}}^{\text{2}}}{{\text{x}}^{\text{n}}}\sqrt{\text{5a}}\text{ } \\ 
 & \sqrt{504{{x}^{2n-1}}}=\sqrt{{{2}^{2}}\cdot {{3}^{2}}\cdot 14\cdot {{x}^{2n-2}}\cdot x}=6{{x}^{n-1}}\sqrt{14x} \\ 
\end{array} \right.$
\item $\left| \begin{array}{ll}
  & \sqrt{\frac{1}{9}}=\frac{1}{3} \\ 
 & \sqrt{\frac{5}{3}}=\frac{\sqrt{5}}{\sqrt{3}}\cdot \frac{\sqrt{3}}{\sqrt{3}}=\frac{\sqrt{15}}{3} \\ 
 & \sqrt{\frac{9}{8}}=\frac{\sqrt{{{3}^{2}}}}{\sqrt{{{2}^{2}}\cdot 2}}=\frac{3}{2\sqrt{2}}\cdot \frac{\sqrt{2}}{\sqrt{2}}=\frac{3\sqrt{2}}{4} \\ 
 & \sqrt{\frac{7}{27}}=\frac{\sqrt{7}}{\sqrt{{{3}^{2}}\cdot 3}}=\frac{\sqrt{7}}{3\sqrt{3}}\cdot \frac{\sqrt{3}}{\sqrt{3}}=\frac{\sqrt{21}}{9} \\ 
 & \sqrt{\frac{1}{2}}=\frac{1}{\sqrt{2}}\cdot \frac{\sqrt{2}}{\sqrt{2}}=\frac{\sqrt{2}}{2} \\ 
 & \sqrt{\frac{5}{7}}=\frac{\sqrt{5}}{\sqrt{7}}\cdot \frac{\sqrt{7}}{\sqrt{7}}=\frac{\sqrt{35}}{7} \\ 
\end{array} \right.$
\item $\left| \begin{array}{ll}
  & \sqrt{\frac{a{{x}^{5}}}{20}}=\frac{\sqrt{a{{x}^{4}}\cdot x}}{\sqrt{4\cdot 5}}=\frac{{{x}^{2}}\sqrt{ax}}{2\sqrt{5}}\cdot \frac{\sqrt{5}}{\sqrt{5}}=\frac{{{x}^{2}}\sqrt{5ax}}{10} \\ 
 & \sqrt{\frac{320{{a}^{5}}}{{{x}^{3}}}}=\frac{\sqrt{{{2}^{6}}\cdot 5{{a}^{4}}\cdot a}}{\sqrt{{{x}^{2}}\cdot x}}=\frac{8{{a}^{2}}\sqrt{5a}}{x\sqrt{x}}\cdot \frac{\sqrt{x}}{\sqrt{x}}=\frac{8{{a}^{2}}\sqrt{5ax}}{{{x}^{2}}} \\ 
 & \sqrt{\frac{12{{a}^{9}}}{5{{b}^{8}}}}=\frac{\sqrt{4\cdot 3{{a}^{8}}\cdot a}}{\sqrt{5{{b}^{8}}}}=\frac{2{{a}^{4}}\sqrt{3a}}{{{b}^{4}}\sqrt{5}}\cdot \frac{\sqrt{5}}{\sqrt{5}}=\frac{2{{a}^{4}}\sqrt{15a}}{5{{b}^{4}}} \\ 
 & \sqrt{\frac{{{a}^{2}}x}{3}}=\frac{a\sqrt{x}}{\sqrt{3}}\cdot \frac{\sqrt{3}}{\sqrt{3}}=\frac{a\sqrt{3x}}{3} \\ 
 & \sqrt{\frac{2{{a}^{3}}}{{{b}^{4}}}}=\frac{\sqrt{2{{a}^{2}}\cdot a}}{\sqrt{{{b}^{4}}}}=\frac{a\sqrt{2a}}{{{b}^{2}}} \\ 
 & \sqrt{\frac{{{a}^{4}}{{x}^{2n}}}{2}}=\frac{\sqrt{{{a}^{4}}{{x}^{2n}}}}{\sqrt{2}}\cdot \frac{\sqrt{2}}{\sqrt{2}}=\frac{{{a}^{2}}{{x}^{n}}\sqrt{2}}{2} \\ 
\end{array} \right.$
\end{enumerate}
\end{solution}

\begin{solution}
Effectuer les opérations suivantes :
\begin{enumerate}
\item $5\sqrt{2}-\frac{1}{2}\sqrt{2}+\frac{2}{3}\sqrt{2}-2\sqrt{2}=\sqrt{2}\left( 5-\frac{1}{2}+\frac{2}{3}-2 \right)=\frac{19}{6}\sqrt{2}$
\item $\sqrt{50}-2\sqrt{8}+3\sqrt{18}-7\sqrt{2}=\sqrt{25\cdot 2}-2\sqrt{4\cdot 2}+3\sqrt{9\cdot 2}-7\sqrt{2}=5\sqrt{2}-4\sqrt{2}+9\sqrt{2}-7\sqrt{2}=3\sqrt{2}$
\item $2\sqrt{54}-2\sqrt{24}-\sqrt{150}+\sqrt{6}=2\sqrt{9\cdot 6}-2\sqrt{4\cdot 6}-\sqrt{25\cdot 6}+\sqrt{6}=6\sqrt{6}-4\sqrt{6}-5\sqrt{6}+\sqrt{6}=-2\sqrt{6}$
\item $2\sqrt{\frac{1}{2}}-\sqrt{18}+\sqrt{\frac{2}{9}}-\sqrt{\frac{9}{8}}=\sqrt{2}-3\sqrt{2}+\frac{\sqrt{2}}{3}-\frac{3}{2\sqrt{2}}\cdot \frac{\sqrt{2}}{\sqrt{2}}=\sqrt{2}-3\sqrt{2}+\frac{\sqrt{2}}{3}-\frac{3\sqrt{2}}{4}=-\frac{29\sqrt{2}}{12}$
\item $\sqrt{48}-\sqrt{\frac{12}{25}}+\sqrt{\frac{1}{3}}+3\sqrt{75}=\sqrt{16\cdot 3}-\sqrt{\frac{4\cdot 3}{{{5}^{2}}}}+\frac{\sqrt{3}}{3}+3\sqrt{25\cdot 3}=4\sqrt{3}-\frac{2\sqrt{3}}{5}+\frac{\sqrt{3}}{3}+15\sqrt{3}=\frac{284\sqrt{3}}{15}$
\item $2\sqrt{28}-6\sqrt{\frac{7}{4}}+14\sqrt{\frac{1}{7}}=2\sqrt{4\cdot 7}-\frac{6\sqrt{7}}{2}+14\frac{\sqrt{7}}{7}=4\sqrt{7}-3\sqrt{7}+2\sqrt{7}=3\sqrt{7}$
\item $\sqrt{72}+3-\sqrt{50}-\sqrt{25}=\sqrt{36\cdot 2}+3-\sqrt{25\cdot 2}-5=6\sqrt{2}+3-5\sqrt{2}-5=\sqrt{2}-2$
\item $5\sqrt{12}-2\sqrt{\frac{3}{4}}+2\sqrt{27}-8\sqrt{\frac{3}{16}}=5\sqrt{4\cdot 3}-\frac{2\sqrt{3}}{2}+2\sqrt{9\cdot 3}-\frac{8\sqrt{3}}{4}=10\sqrt{3}-\sqrt{3}+6\sqrt{3}-2\sqrt{3}=13\sqrt{3}$
\item $-\sqrt{\frac{3}{5}}+2\sqrt{\frac{5}{3}}-\sqrt{60}+\sqrt{\frac{1}{15}}=-\frac{\sqrt{3}}{\sqrt{5}}\cdot \frac{\sqrt{5}}{\sqrt{5}}+2\frac{\sqrt{5}}{\sqrt{3}}\cdot \frac{\sqrt{3}}{\sqrt{3}}-\sqrt{4\cdot 15}+\frac{\sqrt{15}}{15}=-\frac{\sqrt{15}}{5}+\frac{2\sqrt{15}}{3}-2\sqrt{15}+\frac{\sqrt{15}}{15}=\frac{-22\sqrt{15}}{15}
$
\item $2\sqrt{5}-\frac{1}{2}\sqrt{\frac{1}{5}}+\frac{1}{20}\sqrt{45}=2\sqrt{5}-\frac{1}{2}\cdot \frac{\sqrt{5}}{5}+\frac{1}{20}\sqrt{9\cdot 5}=2\sqrt{5}-\frac{\sqrt{5}}{10}+\frac{3\sqrt{5}}{20}=\frac{41\sqrt{5}}{20}$
\end{enumerate}
\end{solution}

\begin{solution}
Effectuer les opérations suivantes :
\begin{enumerate}
\item $\sqrt{2{{a}^{3}}b}-\sqrt{32a{{b}^{3}}}+\sqrt{18a{{b}^{3}}}=a\sqrt{2ab}-4b\sqrt{2ab}+3b\sqrt{2ab}=\sqrt{2ab}\left( a-4b+3b \right)=\sqrt{2ab}\left( a-b \right)$
\item $\sqrt{72{{a}^{5}}{{b}^{5}}}+\sqrt{18{{a}^{3}}{{b}^{7}}}+\sqrt{18{{a}^{7}}{{b}^{3}}}=6{{a}^{2}}{{b}^{2}}\sqrt{2ab}+3a{{b}^{3}}\sqrt{2ab}+3{{a}^{3}}b\sqrt{2ab}=\sqrt{2ab}\left( 6{{a}^{2}}{{b}^{2}}+3a{{b}^{3}}+3{{a}^{3}}b \right)=3ab\sqrt{2ab}\left( 2ab+{{b}^{2}}+{{a}^{2}} \right)=3ab\sqrt{2ab}{{\left( a+b \right)}^{2}}$
\item $\sqrt{{{a}^{5}}-2{{a}^{4}}}+\sqrt{a{{b}^{4}}-2{{b}^{4}}}-\sqrt{4{{a}^{3}}{{b}^{2}}-8{{a}^{2}}{{b}^{2}}}=\sqrt{{{a}^{4}}(a-2)}+\sqrt{{{b}^{4}}\left( a-2 \right)}-\sqrt{4{{a}^{2}}{{b}^{2}}(a-2)}={{a}^{2}}\sqrt{a-2}+{{b}^{2}}\sqrt{a-2}-2ab\sqrt{a-2}=\sqrt{a-2}\left( {{a}^{2}}-2ab+{{b}^{2}} \right)=\sqrt{a-2}{{\left( a-b \right)}^{2}}$
\item $\sqrt{\frac{{{a}^{2}}b}{c}}-\sqrt{\frac{9{{b}^{3}}}{c}}-\sqrt{bc}+\sqrt{\frac{4{{b}^{3}}}{c}}=a\frac{\sqrt{b}}{\sqrt{c}}\cdot \frac{\sqrt{c}}{\sqrt{c}}-3b\frac{\sqrt{b}}{\sqrt{c}}\cdot \frac{\sqrt{c}}{\sqrt{c}}-\sqrt{bc}+2b\frac{\sqrt{b}}{\sqrt{c}}\cdot \frac{\sqrt{c}}{\sqrt{c}}=a\frac{\sqrt{bc}}{c}-3b\frac{\sqrt{bc}}{c}-\sqrt{bc}+2b\frac{\sqrt{bc}}{c}=a\frac{\sqrt{bc}}{c}-3b\frac{\sqrt{bc}}{c}-\frac{c\sqrt{bc}}{c}+2b\frac{\sqrt{bc}}{c}=\frac{\sqrt{bc}}{c}\left( a-3b-c+2b \right)=\frac{\sqrt{bc}}{c}\left( a-b-c \right)$
\item $\sqrt{\frac{a+b}{a-b}}-\sqrt{\frac{a-b}{a+b}}=\frac{\sqrt{a+b}}{\sqrt{a-b}}-\frac{\sqrt{a-b}}{\sqrt{a+b}}=\frac{\sqrt{{{a}^{2}}-{{b}^{2}}}}{a-b}-\frac{\sqrt{{{a}^{2}}-{{b}^{2}}}}{a+b}=\left( \frac{1}{a-b}-\frac{1}{a+b} \right)\sqrt{{{a}^{2}}-{{b}^{2}}}=\frac{2b}{{{a}^{2}}-{{b}^{2}}}\sqrt{{{a}^{2}}-{{b}^{2}}}$
\item $a\sqrt{\frac{a-b}{a+b}}+b\sqrt{\frac{a+b}{a-b}}=a\frac{\sqrt{a-b}}{\sqrt{a+b}}+b\frac{\sqrt{a+b}}{\sqrt{a-b}}=a\frac{\sqrt{{{a}^{2}}-{{b}^{2}}}}{a+b}+b\frac{\sqrt{{{a}^{2}}-{{b}^{2}}}}{a-b}=\left( \frac{a}{a+b}+\frac{b}{a-b} \right)\sqrt{{{a}^{2}}-{{b}^{2}}}=\frac{{{a}^{2}}+{{b}^{2}}}{{{a}^{2}}-{{b}^{2}}}\sqrt{{{a}^{2}}-{{b}^{2}}}$
\item $\sqrt{\frac{b+x}{{{x}^{2}}}}-\sqrt{\frac{b}{{{x}^{2}}}+\frac{1}{x}}=\sqrt{\frac{b+x}{{{x}^{2}}}}-\sqrt{\frac{b+x}{{{x}^{2}}}}=0$
\item $\sqrt{\frac{{{a}^{3}}-{{a}^{2}}b}{{{b}^{2}}}}-\sqrt{\frac{a{{b}^{2}}-{{b}^{3}}}{{{a}^{2}}}}=\frac{\sqrt{{{a}^{2}}(a-b)}}{b}-\frac{\sqrt{{{b}^{2}}(a-b)}}{a}=\frac{a\sqrt{(a-b)}}{b}-\frac{b\sqrt{(a-b)}}{a}=\left( \frac{a}{b}-\frac{b}{a} \right)\sqrt{a-b}$
\end{enumerate}
\end{solution}

\begin{solution}
Effectuer les opérations suivantes :
\begin{enumerate}
\item $\sqrt{28}\cdot \sqrt{7}=2\sqrt{7}\cdot \sqrt{7}=14$	
\item $\sqrt{10}\cdot \sqrt{15}=\sqrt{2\cdot 5\cdot 3\cdot 5}=5\sqrt{6}$
\item $-\sqrt{7}\cdot \sqrt{42}=-\sqrt{7\cdot 7\cdot 6}=-7\sqrt{6}$
\item $\sqrt{7}\cdot \sqrt{\frac{1}{7}}=\frac{\sqrt{7}}{\sqrt{7}}=1$
\item $2\sqrt{18}\cdot \sqrt{8}=2\sqrt{9\cdot 2\cdot 4\cdot 2}=24$
\item $\left( 4-\sqrt{3} \right)\cdot \sqrt{3}=4\sqrt{3}-3$
\item $\left( \sqrt{5}-\sqrt{3} \right)\cdot \sqrt{15}=\sqrt{5\cdot 3\cdot 5}-\sqrt{3\cdot 3\cdot 5}==5\sqrt{3}-3\sqrt{5}$
\item $\left( \sqrt{3}-\sqrt{2} \right)\left( -\sqrt{6} \right)=-\sqrt{3\cdot 2\cdot 3}+\sqrt{2\cdot 2\cdot 3}=-3\sqrt{2}+2\sqrt{3}$
\item $\left( 3+\sqrt{5} \right)\left( 2-\sqrt{5} \right)=6-3\sqrt{5}+2\sqrt{5}-5=1-\sqrt{5}$
\item $\left( 7+2\sqrt{6} \right)\left( 9-5\sqrt{6} \right)=63-35\sqrt{6}+18\sqrt{6}-60=3-17\sqrt{6}$
\item $\left( 9\sqrt{12}+3 \right)\left( \sqrt{3}+8 \right)=9\sqrt{36}+72\sqrt{12}+3\sqrt{3}+24=54+144\sqrt{3}+3\sqrt{3}+24=78+147\sqrt{3}=3(26+49\sqrt{3})$\item $\left( 6+12\sqrt{7} \right)\left( 3-5\sqrt{7} \right)=18-30\sqrt{7}+36\sqrt{7}-420=6\sqrt{7}-402=6(\sqrt{7}-67)$
\end{enumerate}
\end{solution}

\begin{solution}
Effectuer les opérations suivantes, en utilisant exclusivement les propriétés du produit de deux binômes conjugués :
\begin{enumerate}
\item $\left( 9+2\sqrt{10} \right)\left( 9-2\sqrt{10} \right)=81-40=41$
\item $\left( -5-\sqrt{3} \right)\left( -5+\sqrt{3} \right)=25-3=22$
\item $\left( 5-2\sqrt{3} \right)\left( 5+2\sqrt{3} \right)=25-12=13$
\item $\left( \sqrt{x}+\sqrt{y} \right)\left( \sqrt{{{x}^{2}}}-\sqrt{xy} \right)=\sqrt{x}\left( \sqrt{x}+\sqrt{y} \right)\left( \sqrt{x}-\sqrt{y} \right)=\sqrt{x}\left( x-y \right)$
\item $\left( a+b+\sqrt{x} \right)\left( a+b-\sqrt{x} \right)=\left[ \left( a+b \right)+\sqrt{x} \right]\left[ \left( a+b \right)-\sqrt{x} \right]={{\left( a+b \right)}^{2}}-x={{a}^{2}}+2ab+{{b}^{2}}-x$
\item $\left( x-y+\sqrt{xy} \right)\left( x-y-\sqrt{xy} \right)=\left[ \left( x-y \right)+\sqrt{xy} \right]\left[ \left( x-y \right)-\sqrt{xy} \right]={{\left( x-y \right)}^{2}}-xy={{x}^{2}}-2xy+{{y}^{2}}-xy={{x}^{2}}-3xy+{{y}^{2}}$
\item $\left( 3\sqrt{2}+2\sqrt{3} \right){{\left( 3\sqrt{2}-2\sqrt{3} \right)}^{2}}=\left( 3\sqrt{2}+2\sqrt{3} \right)\left( 3\sqrt{2}-2\sqrt{3} \right)\left( 3\sqrt{2}-2\sqrt{3} \right)=(18-12)\left( 3\sqrt{2}-2\sqrt{3} \right)=18\sqrt{2}-12\sqrt{3}$
\item $\left( \sqrt{a}-\sqrt{b}+1 \right)\left( \sqrt{a}+\sqrt{b}-1 \right)=\left[ \sqrt{a}-\left( \sqrt{b}-1 \right) \right]\left[ \sqrt{a}+\left( \sqrt{b}-1 \right) \right]=a-{{\left( \sqrt{b}-1 \right)}^{2}}=a-b+2\sqrt{b}-1$
\end{enumerate}
\end{solution}

\begin{solution}
Effectuer les opérations suivantes :
\begin{enumerate}
\item $\left( 2\sqrt{8}+3\sqrt{5}-7\sqrt{2} \right)\left( \sqrt{72}-5\sqrt{20}-2\sqrt{2} \right)=\left( 4\sqrt{2}+3\sqrt{5}-7\sqrt{2} \right)\left( 6\sqrt{2}-10\sqrt{5}-2\sqrt{2} \right)=\left( 3\sqrt{5}-3\sqrt{2} \right)\left( 4\sqrt{2}-10\sqrt{5} \right)=6\left( \sqrt{5}-\sqrt{2} \right)\left( 2\sqrt{2}-5\sqrt{5} \right)6\left[ 2\sqrt{10}-25-4+5\sqrt{10} \right]=6\left( 7\sqrt{10}-29 \right)=42\sqrt{10}-174$
\item $\left( \sqrt{35}+3\sqrt{2}+\sqrt{7} \right)\left( \sqrt{35}-3\sqrt{2}-\sqrt{7} \right)=\left[ \sqrt{35}+\left( 3\sqrt{2}+\sqrt{7} \right) \right]\left[ \sqrt{35}-\left( 3\sqrt{2}+\sqrt{7} \right) \right]=35-18-6\sqrt{14}-7=10-6\sqrt{14}=2(5-3\sqrt{14})$
\item $\sqrt{5+\sqrt{24}}\cdot \sqrt{5-\sqrt{24}}=\sqrt{\left( 5+\sqrt{24} \right)\left( 5-\sqrt{24} \right)}=\sqrt{25-24}=1$
\item $\sqrt{3-2\sqrt{2}}\cdot \sqrt{3+2\sqrt{2}}=\sqrt{\left( 3-2\sqrt{2} \right)\left( 3+2\sqrt{2} \right)}=\sqrt{9-8}=1$
\end{enumerate}
\end{solution}

\begin{solution}
Effectuer les opérations suivantes :

\begin{enumerate}
\item ${{\left( 3\sqrt{2} \right)}^{2}}=18$
\item ${{\left( 2\sqrt{7} \right)}^{2}}=28$
\item ${{\left( 3a\sqrt{b} \right)}^{2}}=9{{a}^{2}}b$
\item ${{\left( 2a\sqrt{5b} \right)}^{2}}=20{{a}^{2}}b$
\item ${{\left( -x\sqrt{{{x}^{n}}} \right)}^{2}}={{x}^{2}}\cdot {{x}^{n}}={{x}^{n+2}}$
\item ${{\left( \sqrt{2}-\sqrt{6} \right)}^{2}}=2-2\sqrt{12}+6=8-4\sqrt{3}=4(2-\sqrt{3})$
\item ${{\left( 2\sqrt{6}+\sqrt{24} \right)}^{2}}={{\left( 2\sqrt{6}+2\sqrt{6} \right)}^{2}}={{\left( 4\sqrt{6} \right)}^{2}}=96$
\item ${{\left( 2-\sqrt{5} \right)}^{2}}=4-4\sqrt{5}+5=9-4\sqrt{5}$
\item ${{\left( \sqrt{2}+\sqrt{3}-\sqrt{6} \right)}^{2}}=2+3+6+2\sqrt{6}-2\sqrt{12}-2\sqrt{18}=11+2\sqrt{6}-4\sqrt{3}-6\sqrt{2}$
\end{enumerate}
\end{solution}

\begin{solution}
Rendre rationnel le dénominateur des fractions suivantes :
\begin{enumerate}
\item $\frac{9\sqrt{2}-8\sqrt{3}+3\sqrt{6}}{\sqrt{6}}=\frac{9\sqrt{2}-8\sqrt{3}+3\sqrt{6}}{\sqrt{6}}\cdot \frac{\sqrt{6}}{\sqrt{6}}=\frac{9\sqrt{12}-8\sqrt{18}+18}{6}=\frac{18\sqrt{3}-24\sqrt{2}+18}{6}=3\sqrt{3}-4\sqrt{2}+3$\item $\frac{\sqrt{32}-5\sqrt{2}+2\sqrt{8}}{2\sqrt{8}}=\frac{4\sqrt{2}-5\sqrt{2}+4\sqrt{2}}{4\sqrt{2}}=\frac{4-5+4}{4}=\frac{3}{4}$
\item $\frac{-\sqrt{12}+\sqrt{24}-\sqrt{48}}{-\sqrt{6}}=\frac{-2\sqrt{3}+2\sqrt{6}-4\sqrt{3}}{-\sqrt{6}}\cdot \frac{\sqrt{6}}{\sqrt{6}}=\frac{\left( -6\sqrt{3}+2\sqrt{6} \right)\sqrt{6}}{-6}=\frac{-6\sqrt{18}+12}{-6}=\sqrt{18}-2=3\sqrt{2}-2$\item $\frac{3-\sqrt{3}}{1+\sqrt{3}}=\frac{3-\sqrt{3}}{1+\sqrt{3}}\cdot \frac{1-\sqrt{3}}{1-\sqrt{3}}=\frac{3-3\sqrt{3}-\sqrt{3}+3}{1-3}=\frac{6-4\sqrt{3}}{-2}=-3+2\sqrt{3}$	
\item $\frac{\sqrt{2}}{\sqrt{2}-\sqrt{5}}=\frac{\sqrt{2}}{\sqrt{2}-\sqrt{5}}\cdot \frac{\sqrt{2}+\sqrt{5}}{\sqrt{2}+\sqrt{5}}=\frac{2+\sqrt{10}}{2-5}=-\frac{2+\sqrt{10}}{3}$
\item $\frac{\sqrt{6}}{\sqrt{3}+\sqrt{2}}=\frac{\sqrt{6}}{\sqrt{3}+\sqrt{2}}\cdot \frac{\sqrt{3}-\sqrt{2}}{\sqrt{3}-\sqrt{2}}=\sqrt{18}-\sqrt{12}=3\sqrt{2}-2\sqrt{3}$
\item $\frac{\sqrt{5}-\sqrt{3}}{\sqrt{5}+\sqrt{3}}=\frac{\sqrt{5}-\sqrt{3}}{\sqrt{5}+\sqrt{3}}\cdot \frac{\sqrt{5}-\sqrt{3}}{\sqrt{5}-\sqrt{3}}=\frac{5-2\sqrt{15}+3}{5-3}=\frac{8-2\sqrt{15}}{2}=4-\sqrt{15}$
\item $\frac{5\sqrt{3}-3\sqrt{5}}{\sqrt{5}-\sqrt{3}}=\frac{5\sqrt{3}-3\sqrt{5}}{\sqrt{5}-\sqrt{3}}\cdot \frac{\sqrt{5}+\sqrt{3}}{\sqrt{5}+\sqrt{3}}=\frac{5\sqrt{15}+15-15-3\sqrt{15}}{5-3}=\frac{2\sqrt{15}}{2}=\sqrt{15}$
\item $\frac{7\sqrt{5}+5\sqrt{7}}{\sqrt{7}+\sqrt{5}}=\frac{7\sqrt{5}+5\sqrt{7}}{\sqrt{7}+\sqrt{5}}\cdot \frac{\sqrt{7}-\sqrt{5}}{\sqrt{7}-\sqrt{5}}=\frac{7\sqrt{35}-35+35-5\sqrt{35}}{7-5}=\sqrt{35}$
\end{enumerate}
\end{solution}

\subsection{Racines $n$èmes}


\begin{solution}
Simplifier les racines suivantes :
\begin{enumerate}
\item $\sqrt[6]{{{a}^{4}}{{b}^{2}}}=\sqrt[3]{{{a}^{2}}b}$
\item $\sqrt[4]{64{{a}^{8}}{{b}^{6}}}=\sqrt[4]{{{2}^{6}}{{a}^{8}}{{b}^{6}}}=\sqrt{{{2}^{3}}{{a}^{4}}{{b}^{3}}}=\sqrt{{{2}^{2}}\cdot 2\cdot {{a}^{4}}\cdot {{b}^{2}}\cdot b}=2{{a}^{2}}b\sqrt{2b}$
\item $\sqrt{\sqrt[3]{4{{a}^{2}}}}=\sqrt[3]{\sqrt{4{{a}^{2}}}}=\sqrt[3]{2a}$
\item $\sqrt[5]{\sqrt[3]{-{{a}^{5}}}}=\sqrt[3]{\sqrt[5]{-{{a}^{5}}}}=-\sqrt[3]{a}$
\item $\sqrt[5]{\sqrt[3]{{{x}^{20}}}}=\sqrt[15]{{{x}^{20}}}=\sqrt[3]{{{x}^{4}}}=x\sqrt[3]{x}$
\item $\sqrt[3]{2a\sqrt{2a}}=\sqrt[3]{\sqrt{8{{a}^{3}}}}=\sqrt{\sqrt[3]{8{{a}^{3}}}}=\sqrt{2a}$
\item ${{\left( \sqrt[12]{a{{b}^{3}}{{c}^{7}}} \right)}^{4}}=\sqrt[3]{a{{b}^{3}}{{c}^{7}}}=b{{c}^{2}}\sqrt[3]{ac}$
\item ${{\left( \sqrt[7]{-a\sqrt{3a}} \right)}^{14}}={{\left( -a\sqrt{3a} \right)}^{2}}={{a}^{2}}\cdot 3a=3{{a}^{3}}$
\item $\frac{{{a}^{2}}}{b}\sqrt[4]{{{b}^{5}}x}=\frac{{{a}^{2}}b}{b}\sqrt[4]{bx}={{a}^{2}}\sqrt[4]{bx}$
\item ${{\left( \sqrt{a\sqrt[5]{{{a}^{8}}}} \right)}^{4}}={{\left( a\sqrt[5]{{{a}^{8}}} \right)}^{2}}={{\left( a\cdot a\sqrt[5]{{{a}^{3}}} \right)}^{2}}={{\left( {{a}^{2}}\cdot \sqrt[5]{{{a}^{3}}} \right)}^{2}}={{a}^{4}}\cdot \sqrt[5]{{{a}^{6}}}={{a}^{4}}\cdot a\sqrt[5]{a}={{a}^{5}}\sqrt[5]{a}$
\item ${{\left( \sqrt[3]{\sqrt[7]{-8{{a}^{3}}}} \right)}^{7}}={{\left( \sqrt[7]{\sqrt[3]{-8{{a}^{3}}}} \right)}^{7}}=\sqrt[3]{-8{{a}^{3}}}=-2a$
\item $\sqrt[3]{12096}=\sqrt[3]{{{2}^{6}}\cdot {{3}^{3}}\cdot 7}=12\sqrt[3]{7}$
\end{enumerate}
\end{solution}

\begin{solution}
Effectuer les opérations suivantes :
\begin{enumerate}
\item $\sqrt[3]{56}+\sqrt[3]{189}+\sqrt[3]{448}=\sqrt[3]{{{2}^{3}}\cdot 7}+\sqrt[3]{{{3}^{3}}\cdot 7}+\sqrt[3]{{{2}^{6}}\cdot 7}=2\sqrt[3]{7}+3\sqrt[3]{7}+4\sqrt[3]{7}=9\sqrt[3]{7}$
\item $\sqrt[3]{54}-\sqrt[3]{16}+\sqrt{12}-\sqrt{3}=\sqrt[3]{{{3}^{3}}\cdot 2}-\sqrt[3]{{{2}^{3}}\cdot 2}+\sqrt{4\cdot 3}-\sqrt{3}=3\sqrt[3]{2}-2\sqrt[3]{2}+2\sqrt{3}-\sqrt{3}=\sqrt[3]{2}+\sqrt{3}$
\item $4\sqrt[3]{24}+\sqrt[3]{-3}+\sqrt[3]{-81}=4\sqrt[3]{{{2}^{3}}\cdot 3}-\sqrt[3]{3}-\sqrt[3]{{{3}^{3}}\cdot 3}=8\sqrt[3]{3}-\sqrt[3]{3}-3\sqrt[3]{3}=4\sqrt[3]{3}$
\item $\sqrt[6]{16}-\sqrt[4]{4}+\sqrt[3]{-4}=\sqrt[6]{{{2}^{4}}}-\sqrt[4]{{{2}^{2}}}-\sqrt[3]{{{2}^{2}}}=\sqrt[3]{{{2}^{2}}}-\sqrt[{}]{2}-\sqrt[3]{{{2}^{2}}}=-\sqrt{2}$
\item $\sqrt{50}-\sqrt[4]{324}-\sqrt[6]{2916}+\sqrt[8]{256}=\sqrt{{{5}^{2}}\cdot 2}-\sqrt[4]{{{2}^{2}}\cdot {{3}^{4}}}-\sqrt[6]{{{2}^{2}}\cdot {{3}^{6}}}+\sqrt[8]{{{2}^{8}}}
=5\sqrt{2}-3\sqrt[4]{{{2}^{2}}}-3\sqrt[6]{{{2}^{2}}}+2=5\sqrt{2}-3\sqrt{2}-3\sqrt[3]{2}+2=2\sqrt{2}-3\sqrt[3]{2}+2$
\item $9\sqrt[3]{2{{a}^{6}}x}+3\sqrt[3]{-16{{a}^{3}}x}+\sqrt[3]{2x}=9{{a}^{2}}\sqrt[3]{2x}-6a\sqrt[3]{2x}+\sqrt[3]{2x}=\left( 9{{a}^{2}}-6a+1 \right)\sqrt[3]{2x}={{\left( 3a-1 \right)}^{2}}\sqrt[3]{2x}$
\end{enumerate}
\end{solution}

\begin{solution}E
Effectuer les opérations suivantes :
\begin{enumerate}
\item $5\sqrt[3]{18}\cdot \sqrt[3]{6}=5\sqrt[3]{{{3}^{2}}\cdot 2\cdot 2\cdot 3}=5\sqrt[3]{{{3}^{3}}\cdot {{2}^{2}}}=15\sqrt[3]{4}$
\item $\sqrt[4]{20}\cdot \sqrt{2}=\sqrt[4]{2{}^{2}\cdot 5}\cdot \sqrt[4]{{{2}^{2}}}=\sqrt[4]{{{2}^{4}}\cdot 5}=2\sqrt[4]{5}\text{   ou   }\sqrt[4]{20}\cdot \sqrt{2}=\sqrt[4]{{{2}^{2}}\cdot 5}\cdot \sqrt{2}={{2}^{\frac{1}{2}}}\cdot {{5}^{\frac{1}{4}}}\cdot {{2}^{\frac{1}{2}}}=2\cdot {{5}^{\frac{1}{4}}}=2\sqrt[4]{5}$
\item $3\sqrt[4]{a}\cdot 7\sqrt[6]{{{a}^{5}}b}=21\sqrt[4]{a}\sqrt[6]{{{a}^{5}}b}=21\sqrt[12]{{{a}^{3}}}\sqrt[12]{{{a}^{10}}{{b}^{2}}}=21\sqrt[12]{{{a}^{3}}\cdot {{a}^{10}}\cdot {{b}^{2}}}=21a\sqrt[12]{a{{b}^{2}}}\text{ }\text{ou   21}\cdot {{\text{a}}^{\frac{\text{1}}{\text{4}}}}\cdot a{}^{\frac{5}{6}}\cdot {{b}^{\frac{1}{6}}}=21{{a}^{\frac{13}{12}}}{{b}^{\frac{1}{6}}}=21a\sqrt[12]{a{{b}^{2}}}$
\item $\sqrt{a}\cdot \sqrt[3]{-a}\cdot \sqrt[4]{a}=-\sqrt{a}\cdot \sqrt[3]{a}\cdot \sqrt[4]{a}=-\sqrt[12]{{{a}^{6}}\cdot {{a}^{4}}\cdot {{a}^{3}}}=-\sqrt[12]{{{a}^{13}}}=-a\sqrt[12]{a} \mbox{ ou }-\sqrt{a}\cdot \sqrt[3]{a}\cdot \sqrt[4]{a}=-{{a}^{\frac{1}{2}}}\cdot {{a}^{\frac{1}{3}}}\cdot {{a}^{\frac{1}{4}}}-=-{{a}^{\frac{13}{12}}}=-a\sqrt[12]{a}$
\item $\sqrt[3]{3+2\sqrt{2}}\cdot \sqrt[3]{3-2\sqrt{2}}=\sqrt[3]{\left( 3+2\sqrt{2} \right)\left( 3-2\sqrt{2} \right)}=\sqrt[3]{9-8}=1$
\item $\left( a\sqrt[3]{a}-\sqrt[3]{{{a}^{2}}}+1 \right)\left( \sqrt[3]{{{a}^{2}}}+1 \right)=a\sqrt[3]{{{a}^{3}}}+a\sqrt[3]{a}-\sqrt[3]{{{a}^{4}}}-\sqrt[3]{{{a}^{2}}}+\sqrt[3]{{{a}^{2}}}+1 ={{a}^{2}}+a\sqrt[3]{a}-a\sqrt[3]{a}-\sqrt[3]{{{a}^{2}}}+\sqrt[3]{{{a}^{2}}}+1={{a}^{2}}+1$
\end{enumerate}
\end{solution}

\begin{solution}
Rendre rationnel le dénominateur des fractions suivantes :
\begin{enumerate}
\item $\frac{\sqrt[3]{12}}{2\sqrt{2}}=\frac{\sqrt[3]{12}}{2\sqrt{2}}\frac{\sqrt{2}}{\sqrt{2}}=\frac{\sqrt[3]{{{2}^{2}}\cdot 3}\cdot \sqrt{2}}{4}=\frac{\sqrt[6]{{{2}^{4}}\cdot {{3}^{2}}\cdot {{2}^{3}}}}{4}=\frac{2\sqrt[6]{2\cdot {{3}^{2}}}}{4}=\frac{\sqrt[6]{18}}{2}$	
\item $\frac{\sqrt[4]{18}}{\sqrt[4]{8}}=\frac{\sqrt[4]{{{3}^{2}}\cdot 2}}{\sqrt[4]{{{2}^{3}}}}\frac{\sqrt[4]{2}}{\sqrt[4]{2}}=\frac{\sqrt[4]{{{3}^{2}}\cdot {{2}^{2}}}}{2}=\frac{\sqrt{6}}{2}$
\item $\frac{4\sqrt[3]{-12}}{2\sqrt{2}}=\frac{-4\sqrt[3]{{{2}^{2}}\cdot 3}}{2\sqrt{2}}\frac{\sqrt{2}}{\sqrt{2}}=-\sqrt[3]{{{2}^{2}}\cdot 3}\cdot \sqrt{2}=-\sqrt[6]{{{2}^{4}}\cdot {{3}^{2}}\cdot {{2}^{3}}}=-\sqrt[6]{{{2}^{7}}\cdot {{3}^{2}}}=-2\sqrt[6]{18}$
\item $\frac{\sqrt{2}}{\sqrt[3]{4}}=\frac{\sqrt{2}}{\sqrt[3]{{{2}^{2}}}}\frac{\sqrt[3]{2}}{\sqrt[3]{2}}=\frac{\sqrt[6]{{{2}^{3}}\cdot {{2}^{2}}}}{2}=\frac{\sqrt[6]{{{2}^{5}}}}{2}=\frac{\sqrt[6]{32}}{2}$
\item $\frac{\sqrt[5]{27}}{\sqrt[4]{9}}=\frac{\sqrt[5]{{{3}^{3}}}}{\sqrt[4]{{{3}^{2}}}}\frac{\sqrt[4]{{{3}^{2}}}}{\sqrt[4]{{{3}^{2}}}}=\frac{\sqrt[20]{{{3}^{12}}\cdot {{3}^{10}}}}{3}=\frac{3\sqrt[20]{{{3}^{2}}}}{3}=\sqrt[10]{3}$
\item $\frac{\sqrt{3}}{\sqrt[3]{-3}}=-\frac{\sqrt{3}}{\sqrt[3]{3}}\frac{\sqrt[3]{{{3}^{2}}}}{\sqrt[3]{{{3}^{2}}}}=-\frac{\sqrt[6]{{{3}^{3}}\cdot {{3}^{4}}}}{3}=-\frac{3\sqrt[6]{3}}{3}=-\sqrt[6]{3}$
\item $5\sqrt[3]{\frac{8}{75}}=5\frac{\sqrt[3]{{{2}^{3}}}}{\sqrt[3]{{{5}^{2}}\cdot 3}}\frac{\sqrt[3]{5\cdot {{3}^{2}}}}{\sqrt[3]{5\cdot {{3}^{2}}}}=5\frac{2\sqrt[3]{5\cdot {{3}^{2}}}}{15}=\frac{2\sqrt[3]{45}}{3}$
\item $4\sqrt[3]{\frac{3}{80}}=4\frac{\sqrt[3]{3}}{2\sqrt[3]{2\cdot 5}}=2\frac{\sqrt[3]{3}}{\sqrt[3]{2\cdot 5}}\frac{\sqrt[3]{{{2}^{2}}\cdot {{5}^{2}}}}{\sqrt[3]{{{2}^{2}}\cdot {{5}^{2}}}}=\frac{\sqrt[3]{3\cdot {{2}^{2}}\cdot {{5}^{2}}}}{5}=\frac{\sqrt[3]{300}}{5}$
\item $\sqrt[3]{\frac{3}{\sqrt[4]{3}}}=\frac{\sqrt[3]{3}}{\sqrt[12]{3}}\frac{\sqrt[12]{{{3}^{11}}}}{\sqrt[12]{{{3}^{11}}}}=\frac{\sqrt[12]{{{3}^{4}}\cdot {{3}^{11}}}}{3}=\frac{\sqrt[12]{{{3}^{15}}}}{3}=\frac{3\sqrt[12]{{{3}^{3}}}}{3}=\sqrt[4]{3}$
\end{enumerate}
\end{solution}
