\begin{solution}
Simplifier les fractions suivantes :
\begin{multicols}{3}
\begin{enumerate}
\item $\frac{2b}{3a}$
\item $\frac{xy}{b}$
\item $\frac{1}{2(a+3)}$
\item $\frac{a(3{{a}^{2}}+4)}{2{{b}^{2}}}$
\item $a+b$
\item $\frac{4(x+y)}{3(x-y)}$
\item $\frac{x-2}{x+2}$
\item $\frac{1}{8{{a}^{3}}-1}$
\item $\frac{5x+2a}{2ax(5x-2a)}$
\item $\frac{3ax}{2x+5}$
\item $\frac{x-8}{x(x+2)}$
\item $\frac{2(x+3)}{x+7}$
\item $\frac{2x-7}{3(4x-3)}$
\item $\frac{5(2x-1)}{3}$
\item $\frac{{{a}^{2}}+ab+{{b}^{2}}}{a-b}$
\item $x+2$
\item $\frac{2x+1}{x+1}$
\item $\frac{x-7}{{{x}^{2}}+x+1}$
\item $\frac{{{(x+1)}^{2}}}{(x-1)(3{{x}^{2}}+9x+8)}$
\item $\frac{1-x+{{x}^{2}}}{1+x+{{x}^{2}}}$
\item $\frac{2x+1}{2x-1}$
\end{enumerate}
\end{multicols}
\end{solution}


\begin{solution}
Effectuer les opérations suivantes et simplifier :
\begin{multicols}{3}
\begin{enumerate}
\item $\frac{5x}{6}$
\item $\frac{12-x}{6}$
\item $\frac{{{(a+b)}^{2}}}{ab}$
\item $\frac{{{a}^{2}}+{{b}^{2}}}{ab}$
\item $\frac{2x}{(x-6)(x-2)}$
\item $\frac{2xy}{{{x}^{2}}-{{y}^{2}}}$
\item $\frac{2{{x}^{3}}}{1-{{x}^{4}}}$
\item $\frac{2+x}{2-x}$
\item $\frac{-5a}{{{a}^{2}}-1}$
\item 1
\item $\frac{5+2x}{1-{{x}^{2}}}$
\item $\frac{{{y}^{5}}}{{{x}^{6}}-{{y}^{6}}}$
\item $\frac{{{a}^{3}}-{{x}^{3}}}{a}$
\item $\frac{2b}{a+b}$
\item $\frac{ax}{a+x}$
\item $2(x+y)$
\item $\frac{6x}{x+2}$
\item $\frac{1}{1-x}$
\item $\frac{2}{(a+4)(a+2)}$
\item $\frac{2({{a}^{2}}+{{x}^{2}})}{{{a}^{2}}-{{x}^{2}}}$
\item $\frac{ax}{{{x}^{2}}-{{a}^{2}}}$
\item $\frac{5x+9}{{{x}^{2}}-9}$
\item $\frac{3ax}{{{x}^{2}}-4{{a}^{2}}}$
\item $\frac{2(1-x)}{1+x}$
\item $\frac{25x-61}{56}$
\item $\frac{{{a}^{2}}+3{{x}^{2}}}{2ax}$
\item $\frac{{{x}^{3}}+{{y}^{3}}}{{{x}^{2}}{{y}^{3}}}$
\item $\frac{2}{x+y}$
\item $\frac{1}{3-a}$
\item $\frac{a-b}{a+b}$
\item $\frac{2b}{a-b}$
\item $\frac{2(a+1)}{a-1}$
\item $\frac{{{a}^{2}}}{a+b}$
\item $-\frac{a+b}{b}$
\item $\frac{3{{a}^{2}}{{m}^{2}}}{a+m}$
\end{enumerate}
\end{multicols}
\end{solution}

\begin{solution}
Effectuer les opérations suivantes et simplifier :
\begin{multicols}{2}
\begin{enumerate}
\item $\frac{14{{a}^{2}}-15ab-9{{b}^{2}}}{12({{a}^{2}}-{{b}^{2}})}$
\item 1
\item $\frac{2{{a}^{2}}+{{b}^{2}}}{a-b}$
\item 2
\item $\frac{1}{1-{{a}^{2}}}$
\item $\frac{2x-a}{x+a}$
\item $\frac{12}{2x-3}$
\item $\frac{2bx}{4{{x}^{2}}-1}$
\item $\frac{1}{x+2}$
\item $\frac{4}{(x+1)(x-2)}$
\item $\frac{3}{(x-1)(x-3)}$
\item $0$
\item $\frac{a-x}{a+x}$
\item $0$
\item $\frac{2}{c-a}$
\item $0$
\end{enumerate}
\end{multicols}
\end{solution}

\begin{solution}
Effectuer les multiplications suivantes et simplifier :
\begin{multicols}{3}
\begin{enumerate}
\item $\frac{ab}{20}$
\item $-\frac{a{{x}^{2}}}{9}$
\item $\frac{8{{a}^{2}}}{21x}$
\item $-\frac{{{a}^{3}}}{10}$
\item $\frac{4x}{3}$
\item $\frac{4(x-4y)}{x+y}$
\item $\frac{2-a}{2+a}$
\item $-{{(a+6)}^{2}}$
\item $\frac{{{x}^{2}}+2}{x+2}$
\item $\frac{x-3}{x+1}$
\item $\frac{{{x}^{2}}(x-y)}{{{y}^{2}}}$
\item $\frac{(a+x)(x-y)}{2}$
\item $-\frac{5ab(b-3)}{2c}$
\item $\frac{y}{{{x}^{2}}-{{y}^{2}}}$
\item $\frac{3x+2y}{3x+y}$
\item $5\left( 1-x \right)$
\item $x-1$
\item $3$
\item $1$
\item ${{x}^{2}}+xy+{{y}^{2}}$
\item $(2x+5a)(x-a)$
\item $x+3$
\end{enumerate}
\end{multicols}
\end{solution}

\begin{solution}
Effectuer les divisions suivantes et simplifier :
\begin{multicols}{2}
\begin{enumerate}
\item $x-y$
\item ${{(a-b)}^{2}}$
\item $(a-2)(b-3)$
\item ${{a}^{2}}(2x+3{{a}^{2}})$
\item $\frac{(a+b)(x+y)}{a-b}$
\item $\frac{5(3b+4)}{{{a}^{2}}}$
\item $\frac{{{a}^{2}}-{{x}^{2}}}{ax}$
\item $-\frac{a+x}{ax}$
\item $\frac{{{a}^{3}}-1}{a}$
\item $\frac{x-1}{a}$
\item ${{x}^{2}}-ax+{{a}^{2}}$
\item $\frac{x(x-a)}{a(x+a)}$
\item $\frac{x-3}{x(1-x)}$
\item $\frac{-{{x}^{2}}(2x+3)}{2+x}$
\item $\frac{1}{x}$
\item $\frac{{{a}^{2}}-{{b}^{2}}}{2}$
\end{enumerate}
\end{multicols}
\end{solution}

\begin{solution}
Calcul des proportions
\begin{enumerate}
\item Application en économie

Le pourcentage : un rabais de $20 \%$

\begin{tabular}{|l|l|l|l|l|l|}
\hline
Rabais & 20  & 40  & 4  & 2  & 840  \\ \hline
Prix   & 100 & 200 & 20 & 10 & 4200 \\ \hline
\end{tabular}

Le taux de change : 1 \euro = 1.17 Fr. (le 11.07.2011)

\begin{tabular}{|l|l|l|l|l|l|}
\hline
\euro   & 1.00 & 2.00 & 2.14 & 250.50 & 41.03 \\ \hline
Fr. & 1.17 & 2.34 & 2.50 & 293.09 & 48.00 \\ \hline
\end{tabular}

Le salaire d’un vendeur en fonction du chiffre d’affaire : $2 \%$ de commission

\begin{tabular}{|l|l|l|l|l|l|}
\hline
Ventes  & 10'000.00 & 50'000.00 & 200'000.00 & 1'000.00 & 277'750.00 \\ \hline
Salaire & 200.00    & 1'000.00  & 4'000.00   & 20.00    & 5'555.00   \\ \hline
\end{tabular}

Le taux d’intérêt d’un compte épargne : $1.25 \%$ par année

\begin{tabular}{|l|l|l|l|l|l|}
\hline
Montant  & 100.00 & 12'500.00 & 38'400.00 & 151'000.00 & 98'764.80 \\ \hline
Intérêts & 1.25   & 156.25    & 480.00    & 1'887.50   & 1'234.56  \\ \hline
\end{tabular}

\item Application en politique

Les votations : la proportion de « oui »$=1.083/(1.083+1.395)=43.7\%$. Pour le « non » $= 56.3\%$
 
\item Application en géographie

Échelle : environ 3.3 cm

\begin{tabular}{|l|l|l|}
\hline
Réel  & 25'000.00 & 821.10 \\ \hline
Carte & 1.00      & 0.03   \\ \hline
\end{tabular}

L'évolution démographique de la Suisse.

\begin{tabular}{|l|l|l|l|}
\hline
            & 2008    & 2009    & Variation \\ \hline
Émigration  & 86'100  & 86'000  & 0.116\%   \\ \hline
Immigration & 184'300 & 160'600 & -12.86\%  \\ \hline
\end{tabular}

\item Application en Histoire

Les morts de la Seconde Guerre mondiale

\begin{footnotesize}
\begin{tabular}{|l|l|l|l|l|l|}
\hline
Pays        & Pertes militaires & Pertes civiles & Pertes totales & \begin{tabular}[c]{@{}l@{}}Population totale\\   d'avant-guerre\end{tabular} & Pertes en \% \\ \hline
Pologne     & 120'000           & 5'300'000      & 5'420'000      & 36'133'000                                                                   & 15.00\%      \\ \hline
Yougoslavie & 300'000           & 1'200'000      & 1'500'000      & 15'000'000                                                                   & 10.00\%      \\ \hline
Allemagne   & 4'000'000         & 3'000'000      & 7'000'000      & 58'333'333                                                                   & 12.00\%      \\ \hline
Japon       & 2'700'000         & 300'000        & 3'000'000      & 75'000'000                                                                   & 4.00\%       \\ \hline
Italie      & 300'000           & 100'000        & 400'000        & 40'000'000                                                                   & 1.00\%       \\ \hline
France      & 250'000           & 350'000        & 600'000        & 40'000'000                                                                   & 1.50\%       \\ \hline
Royaume-Uni & 326'000           & 62'000         & 388'000        & 48'500'000                                                                   & 0.80\%       \\ \hline
États-Unis  & 300'000           & -              & 300'000        & 150'000'000                                                                  & 0.20\%       \\ \hline
\end{tabular}
\end{footnotesize}
D'après Marc NOUSCHI, Bilan de la Seconde Guerre mondiale, Le Seuil, 1996.

\item Application en sociologie

En fonction de la classe.

\item Application en biologie

Qu’est-ce qui est plus grand : un virus (type grippe) ou une bactérie (type bacille) ?

\begin{tabular}{|l|l|l|}
\hline
\begin{tabular}[c]
{@{}l@{}}Virus\\   à l’échelle (cm)\end{tabular} & 1.500  & 0.80  \\ \hline
Réel (nm)                                                          & 46.875 & 25.00 \\
\hline
\end{tabular}

\begin{tabular}{|l|l|l|}
\hline
\begin{tabular}[c]{@{}l@{}}Bactérie\\   à l’échelle\end{tabular} & 3.40     & 4.20     \\ \hline
Réel (nm)                                                        & 1’619.04 & 2'000.00 \\ \hline
\end{tabular}

On a donc la bactérie qui est environ 34.5 fois plus grande que le virus.
\end{enumerate}
\end{solution}