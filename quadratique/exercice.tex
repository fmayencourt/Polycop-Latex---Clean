\subsection{Fonctions quadratiques}

\begin{exercice}
\begin{multicols}{2}
Étudier les signes des trinômes suivants:
\begin{enumerate}
\item ${{x}^{2}}-7x+6$ 
\item $-3{{x}^{2}}+9x-6$ 
\item $4{{x}^{2}}-4x+1$ 
\item ${{x}^{2}}-x+1$ 
\item $-{{x}^{2}}+7x-12$
\item $-25{{x}^{2}}+10x-1$ 
\item $4{{x}^{2}}+16x+7$
\item ${{x}^{2}}$
\item $-{{x}^{2}}-5$
\item $-9{{x}^{2}}-12x-4$
\item $\left( 3x+1 \right)\left( x-2 \right)$
\item $\left( 6x-7 \right)\left( -x-5 \right)$
\item ${{\left( x+3 \right)}^{2}}-25$
\item $\left( 2x-1 \right)\left( x+3 \right)-\left( 2x-1 \right)\left( 3x-5 \right)$
\item ${{\left( 5-8x \right)}^{2}}$
\item ${{\left( x+2 \right)}^{2}}+5$
\end{enumerate}
\end{multicols}
\end{exercice}

\begin{exercice}
Calculer les coordonnées de l'extremum des fonctions suivantes, indiquer s'il s'agit d'un maximum ou d'un  minimum :
\begin{multicols}{2}
\begin{enumerate}
\item $f(x)=3{{x}^{2}}-2x+1$ 
\item $f(x)=-5{{x}^{2}}+x+2$ 
\item $f(x)={{x}^{2}}-x-20$ 
\item $f(x)=-\frac{{{x}^{2}}}{2}-3x+2$
\item $f(x)=-3{{x}^{2}}-8x+1$
\item $f(x)=\frac{{{x}^{2}}}{4}+\frac{x}{2}-3$
\item $f(x)=\frac{2{{x}^{2}}}{3}-\frac{x}{3}+4$
\item $f(x)={{x}^{2}}+x+1$
\item $f(x)=-\frac{3{{x}^{2}}}{25}+\frac{2x}{5}+2$
\end{enumerate}
\end{multicols}
\end{exercice}

\begin{exercice}
Étudier complètement les fonctions suivantes :
\begin{multicols}{2}
\begin{enumerate}
\item $f(x)={{x}^{2}}-4x+3$ 
\item $f(x)=-{{x}^{2}}+2x-8$ 
\item $f(x)={{x}^{2}}+2x+1$ 
\item $f(x)={{x}^{2}}+x+1$
\item $f(x)={{x}^{2}}$
\item $f(x)=-2{{x}^{2}}$
\item $f(x)={{x}^{2}}-4$
\item $f(x)=-{{x}^{2}}+6x-9$
\item $f(x)=-{{x}^{2}}-2$
\item $f(x)=-{{x}^{2}}+6x-5$
\item $f(x)=2{{x}^{2}}-6x+7$
\item $f(x)=2{{x}^{2}}-8x$
\item $f(x)=\frac{{{x}^{2}}}{4}-x-2$
\item $f(x)=\frac{{{x}^{2}}}{16}+\frac{x}{4}+\frac{1}{4}$
\end{enumerate}
\end{multicols}
\end{exercice}

\begin{exercice}
Trouver les coefficients m et n de la fonction $f(x)={{x}^{2}}+mx+n$ :
\begin{enumerate}
\item si elle admet pour $x=-2$ un minimum égal à 3
\item si elle admet 2 pour zéro de fonction et devient minimum pour $x=\frac{5}{4}$
\item si elle admet 1 pour zéro de fonction et un minimum égal à $\beta =-9$
\item si elle admet pour $x=2$ un minimum égal à 0
\item si elle admet pour zéros de fonction ${x}'=2$ et ${x}''=4$
\item si elle admet – 50 pour zéro de fonction et devient minimum pour $x=-55$
\item si elle admet une ordonnée à l'origine égale à 4 et un minimum égal à $\beta =-\frac{7}{4}$
\item si elle admet une ordonnée à l'origine égale à 49 et un zéro de fonction égal à 21
\end{enumerate}
\end{exercice}

\subsection{Problèmes d'optimisation}

\begin{exercice}
Quelle est la valeur maximale du produit de deux nombres si leur somme doit être égale à 35 ?
\end{exercice}

\begin{exercice}
Quelle est la valeur minimale du produit de deux nombres si leur différence doit être égale à 12 ?
\end{exercice}

\begin{exercice}
Sur la limite nord de son terrain, Louis a une montagne de roc qu'il peut utiliser pour former un des côtés d'un enclos rectangulaire pour ses chiens. Pour les trois autres côtés, il dispose de 80 m de clôture. Quelle est l'aire maximale qu'il peut donner à son enclos ?
\end{exercice}

\begin{exercice}
On a une longue pièce de fer-blanc de 30 cm de large. Le long de chacun des rebords, on redresse deux bandes de largeurs égales en les ramenant dans une position verticale formant ainsi une gouttière. Quelle doit être la largeur de ces bandes que l'on relève, si l'on veut que la gouttière ait une capacité maximale (aire d'une coupe transversale maximale) ?
\end{exercice}

\subsection{La fonction homographique}

\begin{exercice}
Calculer les asymptotes horizontales, les asymptotes verticales, les ordonnées à l'origine et les zéros de fonction des fonctions homographiques suivantes :
\begin{multicols}{2}
\begin{enumerate}
\item $f(x)=\frac{2x-3}{x-4}$
\item $f(x)=\frac{5x+2}{x+1}$
\item $f(x)=\frac{4-x}{2x-5}$
\item $f(x)=\frac{3x+5}{2x-1}$
\item $f(x)=\frac{2x}{3x+7}$
\item $f(x)=\frac{1}{x}$
\end{enumerate}
\end{multicols}
\end{exercice}

\begin{exercice}
Étudier complètement les fonctions homographiques suivantes :
\begin{multicols}{2}
\begin{enumerate}
\item $f(x)=\frac{x+1}{x-1}$
\item $f(x)=\frac{2x+1}{-x+2}$
\item $f(x)=\frac{x-3}{x-2}$
\item $f(x)=\frac{2x-4}{-x+4}$
\item $f(x)=\frac{4x+2}{x+1}$
\item $f(x)=\frac{-4x+6}{2x+1}$
\item $f(x)=\frac{6x-12}{-4x+4}$
\item $f(x)=\frac{2x-5}{x}$
\item $f(x)=\frac{1}{x}$
\item $f(x)=\frac{-5}{x}$
\end{enumerate}
\end{multicols}
\end{exercice}